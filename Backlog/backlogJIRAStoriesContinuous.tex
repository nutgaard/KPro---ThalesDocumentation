\documentclass[a4paper, norsk, 12pt]{article}
\usepackage[T1]{fontenc}
\usepackage[utf8]{inputenc}
\usepackage{babel}
\usepackage{amsmath, amsfonts, amssymb}
\usepackage{graphicx}
\usepackage{array}
\usepackage{tabularx}

\author{KPro - Gruppe 15}
\title{Backlog}
\date{\today}

\newcommand{\sprintPrefix}[0]{$\cdot$ }
%\dateFormat{month}{day}{year}{delimiter}
%\newcommand{\dateFormat}[4]{#1#4#2#4#3}
%\newcommand{\dateFormat}[4]{#2#4#1#4#3}
\newcommand{\dateFormat}[3]{#3.#1.#2}
\newcommand{\JIRA}[1]{\\JIRA reference: #1}
\newcommand{\JIRAS}[1]{(#1)}
\newcommand{\SP}[1]{(#1 SP)}

\begin{document}
	\maketitle
	\pagebreak
	\section{Backlog stories in sorted order }
		\begin{enumerate}
			\item {\bf Starting the app\SP{5}} As a user, I should be able to start the program so that i can begin browsing all features of the program
			\item {\bf Create interface between core and gui\SP{13}} As a programmer, I will make the interfaces that is needed so that the service and gui can communicate in a simple way, making it easy for GUI programmers to start using code that has not been written yet.
			\item {\bf Persisting data to phone\SP{17}} As a programmer I will have to implement persistence, so that the application is able to save data and retrieve it whenever it wants.
			\item {\bf Sending a message\SP{13}} As a user, I should be able to click the ``New message'' button, so that I am brought to the new message page, making me able to create a message and pressing ``Send'' to send it to a user
			\item {\bf Browse previously sent messages\SP{13}} As a user, I want to see all messages that I have previously sent, so that I can check their status.
			\item {\bf Browse inbox\SP{13}} As a user, I should be able to review all previous received messages.
			\item {\bf Receiving messages with attachments\SP{15}} As a user I should be able to receive a message with an attachment and open the attachment
			\item {\bf Sending messages with attachments\SP{13}} As a user I should be able to add an attachment to the message I want to send , so that the recipient gets my attachment as well as the message.
			\item {\bf Viewing address book\SP{8}} As a user I should be able to view my address book with all of my contacts, so that I am able to choose a recipient from a list when I want to send a message.
			\item {\bf Answer, delete and forward a message\SP{8}} As a user I want to be able to utilize the ``Answer'', ``Delete'' and ``Forward'' features that is associated with each message so that i am brought to the correct screen for each of these operations.
			\item {\bf Log in to application\SP{13}} As a user, I should be able to log in via the login screen so that after this process I am an authorized user inside the program.
			\item {\bf Secure communications channel\SP{5}} As a programmer I should implement a wrapper that makes the communication with the server secure.
			\item {\bf Extends communication interface\SP{13}} As a programmer I should extend the communication implementation to support more features, so that we can have message priorities, message types, message grading, message status and notification of failed deliveries.
			\item {\bf Implement hardware abstration layer\SP{20}} As a programmer I will implement the hardware abstraction layer, so that I can use the phones physical input\/output devices, as GPS, camera, video and sound. 
			\item {\bf Distribution of addressbook\SP{13}} As a programmer, I should be able to distribute address books between each of the users via LDAP or Control message so that all users have an address book with all users.
			\item {\bf Message templates\SP{17}} As a user, I should be able to choose a template for a message with all relevant information is predefined, as well as creating new ones, so that I can send a message without writing anything at all.
			\item {\bf Add signing to messages\SP{20}} As a programmer I should be able to create or use an existing library for digital signing and verification of messages.
			\item {\bf Settings menu\SP{8}} As a user, I should be able to utilize the settings menu to alter different settings of the app, so that I have the settings set to what I prefer.
			\item {\bf Compression algorithm\SP{13}} As a programmer I want to be able to create or find a library that minimizes data traffic that is needed to send a message so that the messages are sent faster over low speed internet connections.	
		\end{enumerate}
	\pagebreak
	\section{Sprint planning}
	\subsection{Continuous tasks}
	\subsubsection{Sprint 1}
		\begin{tabularx}{\linewidth}{>{\setlength\hsize{1.5\hsize}}X>{\setlength\hsize{.20\hsize}}X>{\setlength\hsize{.1\hsize}}X}
			Continuous tasks & Estimate & Actual\\
			\hline
			\sprintPrefix Setup of Jira & 25h & \\
			\sprintPrefix Report work & 85h & \\
			\sprintPrefix Setup programming enviroment & 6h & \\
			\sprintPrefix Group administraion & 5h & \\
			\sprintPrefix Meetings with Thales & 3h & \\
			\sprintPrefix Meetings with Mohnsen Anvaari & 3h & \\
			\sprintPrefix Internal meetings & 12h & \\
			\sprintPrefix Lectures & 4h & \\
		\end{tabularx}
	\subsubsection{Sprint 2}
		\begin{tabularx}{\linewidth}{>{\setlength\hsize{1.5\hsize}}X>{\setlength\hsize{.20\hsize}}X>{\setlength\hsize{.1\hsize}}X}
			Continuous tasks & Estimate & Actual\\
			\hline
			\sprintPrefix Setup of Jira & 25h & \\
			\sprintPrefix Report work & 85h & \\
			\sprintPrefix Setup programming enviroment & 6h & \\
			\sprintPrefix Group administraion & 5h & \\
			\sprintPrefix Meetings with Thales & 3h & \\
			\sprintPrefix Meetings with Mohnsen Anvaari & 3h & \\
			\sprintPrefix Internal meetings & 12h & \\
			\sprintPrefix Lectures & 4h & \\
		\end{tabularx}
	\subsubsection{Sprint 3}
		\begin{tabularx}{\linewidth}{>{\setlength\hsize{1.5\hsize}}X>{\setlength\hsize{.20\hsize}}X>{\setlength\hsize{.1\hsize}}X}
			Continuous tasks & Estimate & Actual\\
			\hline
			\sprintPrefix Setup of Jira & 25h & \\
			\sprintPrefix Report work & 85h & \\
			\sprintPrefix Setup programming enviroment & 6h & \\
			\sprintPrefix Group administraion & 5h & \\
			\sprintPrefix Meetings with Thales & 3h & \\
			\sprintPrefix Meetings with Mohnsen Anvaari & 3h & \\
			\sprintPrefix Internal meetings & 12h & \\
			\sprintPrefix Lectures & 4h & \\
		\end{tabularx}
	\subsubsection{Sprint 4}
		\begin{tabularx}{\linewidth}{>{\setlength\hsize{1.5\hsize}}X>{\setlength\hsize{.20\hsize}}X>{\setlength\hsize{.1\hsize}}X}
			Continuous tasks & Estimate & Actual\\
			\hline
			\sprintPrefix Setup of Jira & 25h & \\
			\sprintPrefix Report work & 85h & \\
			\sprintPrefix Setup programming enviroment & 6h & \\
			\sprintPrefix Group administraion & 5h & \\
			\sprintPrefix Meetings with Thales & 3h & \\
			\sprintPrefix Meetings with Mohnsen Anvaari & 3h & \\
			\sprintPrefix Internal meetings & 12h & \\
			\sprintPrefix Lectures & 4h & \\
		\end{tabularx}
	\pagebreak
	\subsection{Sprint 1}
		\begin{description}
			\item[Duration:] \dateFormat{08}{27}{2012} - \dateFormat{09}{16}{2012}
			\item[Backlog tasks:] BL1, BL2, BL3, BL4 and BL5
			\item[Goals:] Have a working demo to show the customer and adviser. It should be able to send and receive messages. Everything should be done through a graphical user interface
			\item[Storypoints:] 48
		\end{description}
		%Duration: \dateFormat{08}{27}{2012} - \dateFormat{09}{16}{2012}\\
		%Backlog tasks: BL1, BL2, BL3, BL4 and BL5\\
		%Goal: Have a working demo to show the customer and adviser. It should be able to send and receive messages. Everything should be done through a graphical user interface\\
		%Storypoints in sprint: 48\\\\
		\begin{tabularx}{\linewidth}{>{\setlength\hsize{1.5\hsize}}X>{\setlength\hsize{.20\hsize}}X>{\setlength\hsize{.1\hsize}}X}
			Subtasks for BL1 & Estimate & Actual\\
			\hline
			\sprintPrefix Learn android MVC & 30h & \\
			\sprintPrefix Create a basic android application skeleton & 10h  & \\
			\sprintPrefix Setup MVC for project & 2h  & \\
			\sprintPrefix Create main menu & 72h  & \\
		\end{tabularx}
		\\\\ \\
		\begin{tabularx}{\linewidth}{>{\setlength\hsize{1.5\hsize}}X>{\setlength\hsize{.20\hsize}}X>{\setlength\hsize{.1\hsize}}X}
			Subtask for BL2 & Estimate & Actual\\
			\hline
			\sprintPrefix Design network service interface & 7h &\\
			\sprintPrefix Design persistence service interface & 6h &\\
			\sprintPrefix Design security service interface & 2h &\\
			\sprintPrefix Design hardware abstraction layer service interface & 3h &\\
		\end{tabularx}
		\\\\ \\
		\begin{tabularx}{\linewidth}{>{\setlength\hsize{1.5\hsize}}X>{\setlength\hsize{.20\hsize}}X>{\setlength\hsize{.1\hsize}}X}
			Subtask for BL3 & Estimate & Actual\\
			\hline
			\sprintPrefix Research on structure for persisting objects & 4h & \\
			\sprintPrefix Save data to phone storage & 24h & \\
			\sprintPrefix Load data from phone storage & 4h  & \\		
		\end{tabularx}
		\\\\ \\
		\begin{tabularx}{\linewidth}{>{\setlength\hsize{1.5\hsize}}X>{\setlength\hsize{.20\hsize}}X>{\setlength\hsize{.1\hsize}}X}
			Subtask for BL4 & Estimate & Actual\\
			\hline
			\sprintPrefix Create gui for configuration of message & 3h & \\
			\sprintPrefix Create core bridge & 28h & \\
			\sprintPrefix Make connection between GUI and backend service & 6h & \\
			\sprintPrefix Implement a network class for sending messages & 48h & \\
			\sprintPrefix Implement a network class for receiving messages & 48h & \\
		\end{tabularx}
	\pagebreak
	\subsection{Sprint 2}
		Duration: \dateFormat{09}{17}{2012} - \dateFormat{10}{07}{2012}\\
		Backlog tasks: Remains from Sprint 1, BL5, BL6, BL7, BL8, BL9 and BL10\\
		Goal: Have a working demo to show the customer and adviser. In addition to goals related to sprint 1, it should incorporate an inbox and an outbox,
			it should be possible to send and receive messages with attachments, add recipients from an addressbook and answer, delete or forward messages.
		Storypoints in sprint: 70\\\\
		\begin{tabularx}{\linewidth}{>{\setlength\hsize{1.5\hsize}}X>{\setlength\hsize{.20\hsize}}X>{\setlength\hsize{.1\hsize}}X}
			Subtask remaining from Sprint 1 & Estimate & Actual\\
			\hline
			Will be decided during the sprint planning
		\end{tabularx}
		\\\\ \\
		\begin{tabularx}{\linewidth}{>{\setlength\hsize{1.5\hsize}}X>{\setlength\hsize{.20\hsize}}X>{\setlength\hsize{.1\hsize}}X}
			Subtask for BL5 & Estimate & Actual\\
			\hline
			Will be decided during the sprint planning
		\end{tabularx}
		\\\\ \\
		\begin{tabularx}{\linewidth}{>{\setlength\hsize{1.5\hsize}}X>{\setlength\hsize{.20\hsize}}X>{\setlength\hsize{.1\hsize}}X}			
			Subtask for BL6 & Estimate & Actual\\
			\hline
			Will be decided during the sprint planning
		\end{tabularx}
		\\\\ \\
		\begin{tabularx}{\linewidth}{>{\setlength\hsize{1.5\hsize}}X>{\setlength\hsize{.20\hsize}}X>{\setlength\hsize{.1\hsize}}X}
			Subtask for BL7 & Estimate & Actual\\
			\hline
			Will be decided during the sprint planning
		\end{tabularx}
		\\\\ \\
		\begin{tabularx}{\linewidth}{>{\setlength\hsize{1.5\hsize}}X>{\setlength\hsize{.20\hsize}}X>{\setlength\hsize{.1\hsize}}X}
			Subtask for BL8 & Estimate & Actual\\
			\hline
			Will be decided during the sprint planning
		\end{tabularx}
		\\\\ \\
		\begin{tabularx}{\linewidth}{>{\setlength\hsize{1.5\hsize}}X>{\setlength\hsize{.20\hsize}}X>{\setlength\hsize{.1\hsize}}X}
			Subtask for BL9 & Estimate & Actual\\
			\hline
			Will be decided during the sprint planning
		\end{tabularx}
		\\\\ \\
		\begin{tabularx}{\linewidth}{>{\setlength\hsize{1.5\hsize}}X>{\setlength\hsize{.20\hsize}}X>{\setlength\hsize{.1\hsize}}X}
			Subtask for BL10 & Estimate & Actual\\
			\hline
			Will be decided during the sprint planning
		\end{tabularx}
	\pagebreak
	\subsection{Sprint 3}
		Duration: \dateFormat{10}{08}{2012} - \dateFormat{10}{28}{2012}\\
		Backlog tasks: Remains from Sprint 2, BL11, BL12, BL13, BL14 and BL15\\
		Goal: Have a working demo to show the customer and adviser. In addition to goals related to sprint 2, it should provide a secure login and a secure communications channel.
			Hardware access should be provided to alert the user of flash messages. The messages should support the use of XOMail attributes, and take the correct actions based on those.
			The addressbook should be fetched and updated through the network.\\
		Storypoints in sprint: 64\\\\
		\begin{tabularx}{\linewidth}{>{\setlength\hsize{1.5\hsize}}X>{\setlength\hsize{.20\hsize}}X>{\setlength\hsize{.1\hsize}}X}
			Subtask remaining from Sprint 2 & Estimate & Actual\\
			\hline
			Will be decided during the sprint planning
		\end{tabularx}
		\\\\ \\
		\begin{tabularx}{\linewidth}{>{\setlength\hsize{1.5\hsize}}X>{\setlength\hsize{.20\hsize}}X>{\setlength\hsize{.1\hsize}}X}
			Subtask for BL11 & Estimate & Actual\\
			\hline
			Will be decided during the sprint planning
		\end{tabularx}
		\\\\ \\
		\begin{tabularx}{\linewidth}{>{\setlength\hsize{1.5\hsize}}X>{\setlength\hsize{.20\hsize}}X>{\setlength\hsize{.1\hsize}}X}
			Subtask for BL12 & Estimate & Actual\\
			\hline
			Will be decided during the sprint planning
		\end{tabularx}
		\\\\ \\
		\begin{tabularx}{\linewidth}{>{\setlength\hsize{1.5\hsize}}X>{\setlength\hsize{.20\hsize}}X>{\setlength\hsize{.1\hsize}}X}
			Subtask for BL13 & Estimate & Actual\\
			\hline
			Will be decided during the sprint planning
		\end{tabularx}
		\\\\ \\
		\begin{tabularx}{\linewidth}{>{\setlength\hsize{1.5\hsize}}X>{\setlength\hsize{.20\hsize}}X>{\setlength\hsize{.1\hsize}}X}
			Subtask for BL14 & Estimate & Actual\\
			\hline
			Will be decided during the sprint planning
		\end{tabularx}
		\\\\ \\
		\begin{tabularx}{\linewidth}{>{\setlength\hsize{1.5\hsize}}X>{\setlength\hsize{.20\hsize}}X>{\setlength\hsize{.1\hsize}}X}
			Subtask for BL15 & Estimate & Actual\\
			\hline
			Will be decided during the sprint planning
		\end{tabularx}
	\pagebreak
	\subsection{Sprint 4}
		Duration: \dateFormat{10}{29}{2012} - \dateFormat{11}{18}{2012}\\
		Backlog tasks: Remains from Sprint 3, BL16, BL17, BL18 and BL19\\
		Goal: Have a working demo to show the customer and adviser. In addition to goals related to sprint 3, it should provide signing and verification of messages,
		a settings menu for local application options. It should also incorporate creating, editing, sending and deleting of message templates. And all data being sent should be compressed\\
		Storypoints in sprint: 58\\\\
		\begin{tabularx}{\linewidth}{>{\setlength\hsize{1.5\hsize}}X>{\setlength\hsize{.20\hsize}}X>{\setlength\hsize{.1\hsize}}X}
			Subtask remaining from Sprint 3 & Estimate & Actual\\
			\hline
			Will be decided during the sprint planning
		\end{tabularx}
		\\\\ \\
		\begin{tabularx}{\linewidth}{>{\setlength\hsize{1.5\hsize}}X>{\setlength\hsize{.20\hsize}}X>{\setlength\hsize{.1\hsize}}X}
			Subtask for BL16 & Estimate & Actual\\
			\hline
			Will be decided during the sprint planning
		\end{tabularx}
		\\\\ \\
		\begin{tabularx}{\linewidth}{>{\setlength\hsize{1.5\hsize}}X>{\setlength\hsize{.20\hsize}}X>{\setlength\hsize{.1\hsize}}X}
			Subtask for BL17 & Estimate & Actual\\
			\hline
			Will be decided during the sprint planning
		\end{tabularx}
		\\\\ \\
		\begin{tabularx}{\linewidth}{>{\setlength\hsize{1.5\hsize}}X>{\setlength\hsize{.20\hsize}}X>{\setlength\hsize{.1\hsize}}X}
			Subtask for BL18 & Estimate & Actual\\
			\hline
			Will be decided during the sprint planning
		\end{tabularx}
		\\\\ \\
		\begin{tabularx}{\linewidth}{>{\setlength\hsize{1.5\hsize}}X>{\setlength\hsize{.20\hsize}}X>{\setlength\hsize{.1\hsize}}X}
			Subtask for BL19 & Estimate & Actual\\
			\hline
			Will be decided during the sprint planning
		\end{tabularx}
	\pagebreak
	\section{Changelog}
		\begin{description}
			\item[\dateFormat{09}{12}{2012}] Rewrite of document. Changed from taskdescriptions to stories in backlog, added storypoints for each entry. 
			\item[\dateFormat{09}{06}{2012}] Goals added to each sprint. BL5 (Message template) moved down in backlog as requested from customer.
			\item[\dateFormat{09}{04}{2012}] Sprint planning for sprint 1 added
			\item[\dateFormat{08}{28}{2012}] Backlog pdf created
		\end{description}
\end {document}