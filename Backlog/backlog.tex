\documentclass[a4paper, norsk, 12pt]{article}
\usepackage[T1]{fontenc}
\usepackage[utf8]{inputenc}
\usepackage{babel}
\usepackage{amsmath, amsfonts, amssymb}
\usepackage{graphicx}
\usepackage{array}
\usepackage{tabularx}

\author{KPro - Gruppe 15}
\title{Backlog}
\date{\today}

\newcommand{\sprintPrefix}[0]{$\cdot$ }
%\dateFormat{month}{day}{year}{delimiter}
%\newcommand{\dateFormat}[4]{#1#4#2#4#3}
\newcommand{\dateFormat}[4]{#2#4#1#4#3}

\begin{document}
	\maketitle
	\pagebreak
	\section{Backlog}
		\begin{enumerate}
			\item Create a backend serviceproviderinterface to provide useful services to the application and widget. 
				This should provide serviceinterfaces for sending\&receiving message, signing\&verifying messages, 
				getting date from other applications, CRUD persistent storage (with and without encryption) and 
				controlling the other hardware like lights and sounds.
				
			\item Create a frontend low-fidelity evolutionary prototype with focus on sending, receiving and viewing plain text messages. 
				This must be done in such a manner that it is easy extendable to handle messages containing attachments of different types e.g. images and videos and any future views necessary.
				
			\item	Implement the networkinterfaces for sending messages from 1. 
				Enabling a full vertical low-fidelity prototype of the whole application.
				Investigate server-push solutions like P-IMAP.
				
			\item Implement the persistenceinterface from 1. 
				NB! Encryption is not needed at this stage, but should be easy to implement at a later stage.
				
			\item Extend and modify the existing frontend prototype adding support for message templates. 
				Creation, editing and deleting should be supported.
				
			\item Add frontend support for login screen, addressbook, browsing previous messages.
			
			\item Add frontend support for sending messages with attachments with respect to the current networkconnection. 
				NB! Video suppoprt is not needed at this stage, but should be easy to implement at a later stage.
				
			\item Extends frontend view for message sending to handle message priorities and other special attributes needed to comply with the requirements of XOMail.
			
			\item Implement the hardware abstraction layer from 1. 
				This could include GPS, camera, video, sound, gyro, networkconnection, light ect. Extending the serviceproviderinterface as necessary.
				
			\item Investigate the possibility of fetching the addressbook from the server. 
				If possible implement under the serviceproviderinterface.
				
			\item Extend the existing communicationimplementations to support message priorities, message types, message grading and and failed deliveries.
			
			\item Extend the existing communicationimplementations to support messages containing multimedia and attachments from other applications.
				Explore restrictions based on network connection, and graceful degradation of content. 
			
			\item Extend the frontend view for messages to handle video attachments, and possibly getting the video without storing it to disk first.	
			
			\item Extend the frontend view for messages to incorporate 'answer', 'delete' and 'forward' functionality.
			
			\item Explore the possibilities of getting the addressbook from the server and providing it through the interfaces described in 1.
				One possibility might be DAP/LDAP.
			
			\item Implement the securityprovider according to the interface described in 1. and add the possibility for encryption local data.
				Explore the choice of possible encryption algorithms and the storage of the private key closely. Hereunder the possibility for local login.
			
			\item Implement a wrapper around the communicationsimplementations that is coherent to the interfaces described in 1. that allows for a secure connection to the server.
			
			\item Extend the securityprovider with signing and verification possibilities.
			
			\item Create views for program settings.
			
			\item Add compressalgorithm in order to minimize the datatrafic that is needed.
				Be aware of how the recipient handles this if you are using unorthodox algorithms. Should it be device specific or not?
			
			\item Create a widget for quicksend messages that uses the serviceprovider.
		\end{enumerate}
	\pagebreak
	\section{Sprint 1}
		Duration: \dateFormat{08}{27}{2012}{.} - \dateFormat{09}{16}{2012}{.}\\
		Backlog tasks: BL1, BL2, BL3, BL4, BL5 and BL6\\\\
		\begin{tabularx}{\linewidth}{>{\setlength\hsize{1.5\hsize}}X>{\setlength\hsize{.20\hsize}}X>{\setlength\hsize{.1\hsize}}X}
			Subtask for BL1 & Estimate & Actual\\
			\hline
			\sprintPrefix Setting up the project structure & 10h & N/A\\
			&&\\
			\sprintPrefix Create the interface for the Network module & 7h & N/A\\
			\sprintPrefix Create the interface for the Persistence module & 3h & N/A\\
			\sprintPrefix Create the interface for the Hardware abstraction layer module & 3h & N/A\\
			\sprintPrefix Create the interface for the Security module & 4h & N/A\\
			\sprintPrefix Create factoryclasses for the core modules & 2h & N/A\\
			\sprintPrefix Create dummyclasses for each of the interfaces & 4h & N/A\\
		\end{tabularx}
		\\\\ \\
		\begin{tabularx}{\linewidth}{>{\setlength\hsize{1.5\hsize}}X>{\setlength\hsize{.20\hsize}}X>{\setlength\hsize{.1\hsize}}X}
			Subtask for BL2 & Estimate & Actual\\
			\hline
			\sprintPrefix Create main menu view & 8h & N/A\\
			\sprintPrefix Create send e-mail view & 6h & N/A\\
			&&\\
			\sprintPrefix Create innbox view & 8h & N/A\\
			\sprintPrefix Create view e-mail view & 6h & N/A\\
		\end{tabularx}
		\\\\ \\
		\begin{tabularx}{\linewidth}{>{\setlength\hsize{1.5\hsize}}X>{\setlength\hsize{.20\hsize}}X>{\setlength\hsize{.1\hsize}}X}
			Subtask for BL3 & Estimate & Actual\\
			\hline
			\sprintPrefix Implement a Network class for sending e-mail through gmail's smtp-service & 6h & N/A\\
			\sprintPrefix Implement receiving mail from gmail's imap-service & 7h & N/A\\
			&&\\
			\sprintPrefix Integrate the views created from BL2 to use the implemented Network class & 4h & N/A\\
		\end{tabularx}
		\\\\ \\
		\begin{tabularx}{\linewidth}{>{\setlength\hsize{1.5\hsize}}X>{\setlength\hsize{.20\hsize}}X>{\setlength\hsize{.1\hsize}}X}
			Subtask for BL4 & Estimate & Actual\\
			\hline
			\sprintPrefix Decide structure and format that should be used on disk & 4h & N/A\\
			\sprintPrefix Implement a Persistence class for saving to disk  & 8h & N/A\\
			\sprintPrefix Implement reading from disk & 4h & N/A\\
		\end{tabularx}
		\\\\ \\
		\begin{tabularx}{\linewidth}{>{\setlength\hsize{1.5\hsize}}X>{\setlength\hsize{.20\hsize}}X>{\setlength\hsize{.1\hsize}}X}
			Subtask for BL5 & Estimate & Actual\\
			\hline
			\sprintPrefix Create main menu entry for message templates & 2h & N/A\\
			\sprintPrefix Create ``list of templates''-view & 6h & N/A\\
			\sprintPrefix Create ``new template''-view & 6h & N/A\\
			\sprintPrefix Extend ``new template''-view to be used for editing & 4h & N/A\\
			\sprintPrefix Add delete functionally to ``list of templates''-view & 4h & N/A\\
		\end{tabularx}
		\\\\ \\
		\begin{tabularx}{\linewidth}{>{\setlength\hsize{1.5\hsize}}X>{\setlength\hsize{.20\hsize}}X>{\setlength\hsize{.1\hsize}}X}
			Subtask for BL6 & Estimate & Actual\\
			\hline
			\sprintPrefix Create login view & 8h & N/A\\
			\sprintPrefix Create main menu entry for addressbook & 2h & N/A\\
			\sprintPrefix Create ``list of addresses''-view & 8h & N/A\\
			%\sprintPrefix Create ``new address''-view & 8h & N/A\\
			%\sprintPrefix Extend ``new address''-view to be used for editing & 4h & N/A\\
			\sprintPrefix Connect the ``send mail''-view to easily make use of the addressbook & 8h & N/A\\
		\end{tabularx}
		
	\section{Sprint 2}
		Duration: \dateFormat{09}{17}{2012}{.} - \dateFormat{10}{07}{2012}{.}\\
		Backlog tasks: Remains from Sprint 1, BL7, BL8, BL9, BL10, BL11 and BL12\\\\
		\begin{tabularx}{\linewidth}{>{\setlength\hsize{1.5\hsize}}X>{\setlength\hsize{.20\hsize}}X>{\setlength\hsize{.1\hsize}}X}
			Subtask remaining from Sprint 1 & Estimate & Actual\\
			\hline
			Will be decided during the sprint planning
		\end{tabularx}
		\\\\ \\
		\begin{tabularx}{\linewidth}{>{\setlength\hsize{1.5\hsize}}X>{\setlength\hsize{.20\hsize}}X>{\setlength\hsize{.1\hsize}}X}
			Subtask for BL7 & Estimate & Actual\\
			\hline
			Will be decided during the sprint planning
		\end{tabularx}
		\\\\ \\
		\begin{tabularx}{\linewidth}{>{\setlength\hsize{1.5\hsize}}X>{\setlength\hsize{.20\hsize}}X>{\setlength\hsize{.1\hsize}}X}
			Subtask for BL8 & Estimate & Actual\\
			\hline
			Will be decided during the sprint planning
		\end{tabularx}
		\\\\ \\
		\begin{tabularx}{\linewidth}{>{\setlength\hsize{1.5\hsize}}X>{\setlength\hsize{.20\hsize}}X>{\setlength\hsize{.1\hsize}}X}
			Subtask for BL9 & Estimate & Actual\\
			\hline
			Will be decided during the sprint planning
		\end{tabularx}
		\\\\ \\
		\begin{tabularx}{\linewidth}{>{\setlength\hsize{1.5\hsize}}X>{\setlength\hsize{.20\hsize}}X>{\setlength\hsize{.1\hsize}}X}
			Subtask for BL10 & Estimate & Actual\\
			\hline
			Will be decided during the sprint planning
		\end{tabularx}
		\\\\ \\
		\begin{tabularx}{\linewidth}{>{\setlength\hsize{1.5\hsize}}X>{\setlength\hsize{.20\hsize}}X>{\setlength\hsize{.1\hsize}}X}
			Subtask for BL11 & Estimate & Actual\\
			\hline
			Will be decided during the sprint planning
		\end{tabularx}
		\\\\ \\
		\begin{tabularx}{\linewidth}{>{\setlength\hsize{1.5\hsize}}X>{\setlength\hsize{.20\hsize}}X>{\setlength\hsize{.1\hsize}}X}
			Subtask for BL12 & Estimate & Actual\\
			\hline
			Will be decided during the sprint planning
		\end{tabularx}
		\\\\ \\
	
	\section{Sprint 3}
		Duration: \dateFormat{10}{08}{2012}{.} - \dateFormat{10}{28}{2012}{.}\\
		Backlog tasks: Remains from Sprint 2, BL13, BL14, BL15, BL16\\\\
		\begin{tabularx}{\linewidth}{>{\setlength\hsize{1.5\hsize}}X>{\setlength\hsize{.20\hsize}}X>{\setlength\hsize{.1\hsize}}X}
			Subtask remaining from Sprint 2 & Estimate & Actual\\
			\hline
			Will be decided during the sprint planning
		\end{tabularx}
		\\\\ \\
		\begin{tabularx}{\linewidth}{>{\setlength\hsize{1.5\hsize}}X>{\setlength\hsize{.20\hsize}}X>{\setlength\hsize{.1\hsize}}X}
			Subtask for BL13 & Estimate & Actual\\
			\hline
			Will be decided during the sprint planning
		\end{tabularx}
		\\\\ \\
		\begin{tabularx}{\linewidth}{>{\setlength\hsize{1.5\hsize}}X>{\setlength\hsize{.20\hsize}}X>{\setlength\hsize{.1\hsize}}X}
			Subtask for BL14 & Estimate & Actual\\
			\hline
			Will be decided during the sprint planning
		\end{tabularx}
		\\\\ \\
		\begin{tabularx}{\linewidth}{>{\setlength\hsize{1.5\hsize}}X>{\setlength\hsize{.20\hsize}}X>{\setlength\hsize{.1\hsize}}X}
			Subtask for BL15 & Estimate & Actual\\
			\hline
			Will be decided during the sprint planning
		\end{tabularx}
		\\\\ \\
		\begin{tabularx}{\linewidth}{>{\setlength\hsize{1.5\hsize}}X>{\setlength\hsize{.20\hsize}}X>{\setlength\hsize{.1\hsize}}X}
			Subtask for BL16 & Estimate & Actual\\
			\hline
			Will be decided during the sprint planning
		\end{tabularx}
	
	\section{Sprint 4}
		Duration: \dateFormat{10}{29}{2012}{.} - \dateFormat{11}{18}{2012}{.}\\
		Backlog tasks: Remains from Sprint 3, BL17, BL18, BL19, BL20 and BL21\\\\
		\begin{tabularx}{\linewidth}{>{\setlength\hsize{1.5\hsize}}X>{\setlength\hsize{.20\hsize}}X>{\setlength\hsize{.1\hsize}}X}
			Subtask remaining from Sprint 3 & Estimate & Actual\\
			\hline
			Will be decided during the sprint planning
		\end{tabularx}
		\\\\ \\
		\begin{tabularx}{\linewidth}{>{\setlength\hsize{1.5\hsize}}X>{\setlength\hsize{.20\hsize}}X>{\setlength\hsize{.1\hsize}}X}
			Subtask for BL17 & Estimate & Actual\\
			\hline
			Will be decided during the sprint planning
		\end{tabularx}
		\\\\ \\
		\begin{tabularx}{\linewidth}{>{\setlength\hsize{1.5\hsize}}X>{\setlength\hsize{.20\hsize}}X>{\setlength\hsize{.1\hsize}}X}
			Subtask for BL18 & Estimate & Actual\\
			\hline
			Will be decided during the sprint planning
		\end{tabularx}
		\\\\ \\
		\begin{tabularx}{\linewidth}{>{\setlength\hsize{1.5\hsize}}X>{\setlength\hsize{.20\hsize}}X>{\setlength\hsize{.1\hsize}}X}
			Subtask for BL19 & Estimate & Actual\\
			\hline
			Will be decided during the sprint planning
		\end{tabularx}
		\\\\ \\
		\begin{tabularx}{\linewidth}{>{\setlength\hsize{1.5\hsize}}X>{\setlength\hsize{.20\hsize}}X>{\setlength\hsize{.1\hsize}}X}
			Subtask for BL20 & Estimate & Actual\\
			\hline
			Will be decided during the sprint planning
		\end{tabularx}
		\\\\ \\
		\begin{tabularx}{\linewidth}{>{\setlength\hsize{1.5\hsize}}X>{\setlength\hsize{.20\hsize}}X>{\setlength\hsize{.1\hsize}}X}
			Subtask for BL21 & Estimate & Actual\\
			\hline
			Will be decided during the sprint planning
		\end{tabularx}

\end {document}