
\chapter{Usability testing}\label{ch:usatest}
	\section{Test execution}
		\subsection{Preparation for the test leader}
			\begin{itemize}
				\item{}Make sure you have the latest version of XOXOmail (updated 08.11) installed and Internet connection on your phone/computer
				\item{}Have the tasks ready on paper or screen
				\item{}Have the SUS questionnaire ready on paper or screen
				\item{}Have pen and the observation form on paper or a computer ready to make notes on
				\item{}Preconditions for the tests: Make sure there are no mails on the account (kprotesting@gmail.com/kprotest) with flash/override as priority. Other messages are OK.
				\item{}Put an ID (e.g. the test leader's name) on both the observation form and the SUS questionnaire. 
				\item{}Note that task 5 is meant to open the message the test person sent to himself/herself. If the test person sent the message to another address or failed to send a message, you need to send a new message with flash/override priority to kprotesting@gmail.com.
			\end{itemize}
		
		\subsection{Information about the test given to the user}
			\begin{itemize}
				\item{}This is a test to find out if the application is intuitive and user friendly, and not a test of you and your skills.
				\item{}The test consists of six (6) tasks and will take approximately 20 minutes.
				\item{}Read the instructions for each task and perform the tasks one by one.
				\item{}Each task has the same structure: A task number, a task name, a description of what you should do and input data that is needed to solve a task. If you find input fields in the application that do not have a value listed, you can enter an arbitrary value.
				\item{}If you cannot figure out how to solve a task this is not your fault, but the application that is not designed in a user friendly way. You can then move on to the next task, but notify the test leader.
				\item{}You can ask questions before and after the test, but we cannot help you during the test.
				\item{}You can quit the test anytime you want.
				\item{}It would be helpful if you could try to think aloud during the test. Try to explain what you see and why you make your choices. This makes it easier for us to figure out how users think and what could be done better in the design.
				\item{}After the test we would appreciate if you could fill out a questionnaire and give feedback if you felt that something worked well or not so well.
			\end{itemize}
\subsection{Information about the application given to the user}
			XOXOmail is a mail client application for Android phones. It is intended for use in the military or other similar organizations. The application is very much like a regular mail client application, but the difference is that a mail in this application can be given three special  attributes: security label, priority and type. Otherwise the application has the ability, like regular mail clients, to send and receive messages with and without attachments. 
			\newline
			\newline
			There is also a distinction between a regular mail and a so-called instant message. An instant message is meant for quicker sending, where you can predefine a receiver and values of the three attributes in the settings.  A last thing to note is that messages with high priority are handled differently from regular messages, as they will take over the screen and force an action to either open the message or cancel to carry on with what you were doing.
	     	\subsection{SUS form}
			The SUS form can be found in figure \ref{fig:SUS_form} on page \pageref{fig:SUS_form}
			\begin{figure}[htp]{
			\includegraphics[page=4, scale=0.8]{./SUS_info.pdf}}
			\caption{SUS form}\label{fig:SUS_form}
			\end{figure}
		\subsection{Observation form}
			The observation form template can be found in figure \ref{fig:observation_form} on page \pageref{fig:observation_form}
			\begin{figure}[htp]{
			\includegraphics[scale=0.8]{./observation_form.pdf}}
			\caption{Observation form}\label{fig:observation_form}
			\end{figure}

	\section{The results}
	\includepdf[pages={1,2,3,4,5}]{./observation_form_results.pdf}
