\section{Fulfillment of requirements}

\subsection{Functional quality requirements}
All of the functional requirements listed in section ??? (Business requirements sec.1@googledocs), require two things from the application and architecture, a frontend graphical user interface for the user to interact with, and backend service providing the frontend with data and functionality. The fulfillment of functional requirement will therefore not be investigated any further.

\subsection{Non-functional quality requirements}

\subsubsection{Usability}
The architecture has in no way changed due to the usability requirements. 

\subsubsection{Security}

\paragraph{Accessing locally stored data outside of app}
“No data exposed to the user or app, as long as the user do not have root access.” S1 - Quality requirements
\newline
\newline
This requirement is fulfilled by the PersistenceImpl, (figure in graphical view of architecture), which keeps track of all the models currently in use in the application. If the malicious user or application does not have root access, then the data will be secure just by using android provided features like sandboxing processes and private storage. We can not however guarantee that this will always be the case, and another layer of security is therefore needed. 
\newline
\newline
There are two possible scenarios when looking at this requirement. First off, a malicious user/application find a locked/encrypted device, where the owner is not logged in to the device. In this case, the android hardware encryption will successfully mitigate any malicious intent.
Secondly, a malicious user/application find a unlocked devices, but the owner is not logged into the XOXOMail. Under normal circumstances, without an additional layer of security, the malicious user could gain access to the application and application data, circumventing the android private storage access control due to the users root privileges. In order to mitigate this, all XOXOMail specific data will be encrypted before stored into the private storage section. 
Third, a malicious user/application find an unlocked device, where the owner is logged into the XOXOMail application. In this case the malicious user is given some possible ways of exploiting the application by sending predefined flash messages. But it would still require the malicious user to type in the owner password in order to browse any other messages. 

\paragraph{Trying to use app with wrong privileges}
“No features exposed to the user - stopped by login screen” S2 - Quality requirements
\newline
\newline
This requirement is fulfilled by the ServiceProvider along with the graphical user interface. There are two possible scenarios when looking at this requirement. First off, the service backend is not started and a user starts the app. In this case the no privileges will be given until the user types in the correct username and password. In fact, the application itself does not have the possibility to decrypt any data without this information.
	Secondly, the service is started, but not open. In this case a malicious user could send predefined flash messages, but not view any messages, or get access to the system as a whole. 

\paragraph{Trying to access the apps external data trafic}
“No useful data exposed to the user” S3 - Quality requirements
\newline
\newline
This requirement is fulfilled by the third party library, JavaMail, which ensures a secure communication channel to the mail server. JavaMail  will setup a SSL/TLS connection, mitigating the possibility of looking into the data traffic generated by the application.

\subsubsection{Performance}

\paragraph{Latency}
“With a latency of maximum 3 seconds, the user should be able to read the message after it is received. This is the latency when we subtract the download time of message, which is dependent of the network connection.” P1 - Quality requirements
\newline
\newline
This requirement is fulfilled by the IMAPIdle module, (figure in graphical view of architecture), which keeps an open connection from the device to the IMAP server. The IMAP server will not respond to the IDLE request until a new message is received or a message is deleted. Once the client receives an answer from the server it can determine the correct course of action and issue a new IDLE request, causing it to go back into a waiting stage. 