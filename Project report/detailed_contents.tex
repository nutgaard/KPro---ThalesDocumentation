\chapter*{Detailed contents}

\section*{Goal of the report}
This report documents the entire process behind the application. It describes what was done, why it was done and when it was done. The report contains summaries of all the pre-studies the group has been doing throughout the project, as well as the decisions made. It will describe the reason behind the actions and decisions.
The report can also be used to monitor the development process of making the application the customer wants. In short, it describes how the prioritizations was made, and sums up what went well and what went not so well during the project.

\section*{What this report covers}
Chapter 2 - Project directive describes all the involved parties in this project. \\ 
Chapter 3 - Project planning gives an overview of how this task have been planned, how the team is organized and the quality assurance measures of the project. \\
Chapter 4 - Preliminary studies gives summaries of all the preliminary studies done in this project, as well as the decisions that have been made. \\
Chapter 5 - Requirements contains user stories and business requirements that detail how the application should look and behave. \\ 
Chapter 6 - Test plan explains how the code have been tested in this project. \\
Chapter 7 - Architectural description describes the architecture of the application, both backend and frontend. \\
Part II - Scrum process describes all of the sprints very carefully. It also includes the product backlog and the changelog. \\
Part III - Conclusion \& Reflection includes reflections on how the team has worked together, and describes disputes and agreements.

\section*{How this report is structured}
The report is divided into parts as well as chapters. This gives a much better overview of the report, and it makes it a lot easier to find its way around it. It is divided into four parts; planning, preliminary studies, sprints and reflection.