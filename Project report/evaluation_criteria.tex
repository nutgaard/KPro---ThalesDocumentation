\section{Evaluation and conclusion}

HERE WE WILL HAVE A SUPER DUPER CONCLUSION OF ALL THE CHAPTERS

\subsection{Remote vs. local service conclusion}
There was little that supported the use of a remote service for this application, so the natural choice is the local service, both for its ease of use and performance.

\newpage

\subsection{Secure communication conclusion}
There exist several methods which can be used to secure a communication channel, some more prominent than others. Secure communication could in theory be implemented at every level of the TCP/IP 5-layer reference model \cite{bib:cn}/"Internet model" \cite{bib:rfc1122}. The first option was to create our own secure protocol at the application layer that would provide a transparent wraparound of the transport layer, but this is rarely an optimal solution and will probably result in a poor and insecure imitation of an existing solution.
\newline
\newline
Second, we choose a lower level, as to implement encryption on the network or link layer \cite{bib:techtarget}. For network layer encryption it would be possible to use \gls{ips}, this is however not possible through Java or Android \cite{bib:ispec} but would require interfering with the operating system beneath.  The same can be said about link layer solutions; it would require interference with the operating system. But compared the link layer encryption relies on the security of each link host, something that cannot be guaranteed when sending over the Internet.
\newline
\newline
Third we have the option of using a pre-existing application/transport layer protocol like \gls{ssl1}/\gls{tls}. In order to maximize security gain it is recommended to use \gls{tls} 1.2 \cite{bib:ssl}  which has improved security relative to earlier version of \gls{tls} and \gls{ssl1}.

\subsection{Secure storage conclusion}
If we chose to create custom account types it would involve a lot of server side implementation that would be time consuming and problematic. So we should try to avoid this solution. The two other possibilities are both valid ones if we make certain assumptions.
\newline
\newline
If we are going to implement the version where username and password is saved in the local storage we would need to assume that no unwanted physical access to the phone will occur. Given this assumption this would be the best and easiest solution to implement.
\newline
\newline
To implement the secure storage with derived keys we need to assume there will be no problem for the user to log in every time he starts the application. This will make sure the information is secure even though the phone should get misplaced or stolen, but this is more complicated to implement than saving the credentials locally.

\newpage

\subsection{Wireshark conclusion}
Sending and receiving messages with our application over GSM seems to work in a respectable manner. The time it takes to receive and send messages is pretty good considering this was done over 14.4 kbits/s. The TLS protocol also seems to be working as intended for both sending and receiving messages. The initial handshake is completed and the following data is encrypted as it should be. 
\newline
\newline
The only problem we could find is that the amount of data sent when considering both the sending and receiving functionality appears to be much larger than the actual size of the message when you inspect it in clear text. The receiving part of the application had an increase of 66.47% whilst the sending part had an increase of 42.7%. 
\newline
\newline
At first we thought this could be a result of the TLS encryption of the data, but if the encryption is working as intended the only part of the encryption that could create some overflow is the initial handshake. This was not included in the calculation, only the actual application data sent or received was included. The actual encrypted TLS packages are based on relatively efficient symmetric ciphers and should not take up a lot more space than unencrypted packages.  
\newline
\newline
It is difficult to pinpoint the exact reason for this large increase in data transferred at this point, but it was sent in a reasonable time and according to protocol, and those are the most important attributes for our client.  

\subsection{Compression of data conclusion}
We have now gone through a lot of methods for both lossy and lossless compression and it is evident that there are many choices. The intention of this compression study was merely to give an overview of what exists. We have not listed all options, but just peeked into some of the most interesting methods. In order to give a qualified answer to what is the best algorithms, a more thorough study has to be done in order to find the requirements on compression time in relation with connection speed, what Android devices will be using the algorithms and what algorithms that have a working implementation for android.  