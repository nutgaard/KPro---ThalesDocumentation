\pagebreak
\section{Evaluation and conclusion of project study}
Through the project study we had looked at the desired functionality of the app we were requested to make. In many ways, it was to function as a regular mail client, but with additional specific functionality related to miltary attributes and, of course, security. We conducted a study to see if any similar solutions existed. This kind of application is targeted at a very specific group, and we realized early on that we could not find any existing applications that could solve our problem. We found out that there exist a lot of other mail clients as well as applications and libraries that support signing and verification, but not the full package. Some of the ideas regarding design and solution could be used in our application, but not directly.
\newline\newline
This project had a lot of uncertainties, and choosing our development model was important. The choice fell on agile development in the form of Scrum. This method fits this kind of project where uncertainty and regular revisions are central parts of development. Using a waterfall model would be too rigid for a project like this.
\newline\newline
One of the most important decisions we did at the beginning of the project was to decide on what tools we wanted to use during the development process and what programming language(s) we should use. The first and obvious choice was to use Java, as this is the de facto standard for Android programming. Even though we did an investigation to see if other programming languages could do the work better for us in uncertain settings, this turned out to not be the case. All other languages would give us more problems than advantages, so the natural choice was Java. We also had to choose a development enviroment, where the choice was between Eclipse and Netbeans, and the latter won.  This was in some ways an unfortunate choice regarding Android development, as we were sometimes missing some tools in Netbeans, but in the long run it was not a big problem.
\newline\newline
We also made some important decisions about the application. The first choice was to have the service part of the application locally instead of on a server, since Android supports local services easily.
\newline\newline
One of the most important aspects of the application was to find out what security features that were facilitated by Android and what needed to be implemented by us. In order to do this we had to make some assumptions about the phone, the users of the application and the network. After this was done we could discuss what is secure and what is not secure under these assumptions. We ended up with different solutions on how a secure communication channel could be made and that local storage of the current user is sufficiently protected under the assumption that no one will get unwanted access to the phone. We also assumed that the user would be able to log in to the application everytime it started without any problems.
\newline\newline
Since the application should do encryption before sending messages, we wanted to check that the application performed well under conditiions where the network connection is limited. We saw that this worked well without any problems.
\newline
\newline
When it comes to compression, we saw that this was a very complicated field of study. What we ended up with was a listing of a lot of current techniques that is possible. Deciding on a compression technique for the different data types is beyind the scope of this project.
\newline\newline
At the end we listed up all relevant licences for tools and software that we had used, and ended with a summary of the different licences.