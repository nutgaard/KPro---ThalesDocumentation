\subsection{Markup languages}

\subsubsection{\gls{xml}}
Extensible Markup Language, also known as \gls{xml}, is a markup language that defines a set of rules for encoding documents in a format that is both human-readable and machine-readable \cite{bib:xml}. Its purpose is to have a simple way of representing data in a textual format which is easy to learn and use. It is often used in programming as a medium for persisting data from objects into a file format which then can the be used later on. Many libraries have been written to make it easier to generate \gls{xml} data.

\subsubsection{\gls{json}}
JavaScript Object Notation, also known as \gls{json}, is a text-based open standard designed for human-readable data interchange. This makes it quite like \gls{xml}. The main difference is that its structure is much simpler than \gls{xml}. The grammar is smaller and maps directly onto the data structures used in modern programmming languages \cite{bib:json}.

\newpage

\subsection{Markup languages conclusion}
\gls{xml} is a widespread exchange format which has a lot of libraries that can be used to generate \gls{xml} data. Even though, it is known that \gls{xml} is primarily a document exchange format, where as \gls{json} is better suited for data exchange. This means that \gls{json} often is much more readable, as the mapping between objects in programming and \gls{json} representation is much more alike. As a result, the \gls{json} code is easier for machines to read and write. \gls{json} is becoming more and more common, and is now widely adopted by the computer industry. Still, the format is irrelevant for us, as we will be using a serializer and deserializer that ensures that syntax is correct, so we will use a \gls{xml}-serializer. 
