\subsection{Markup languages}

\subsubsection{XML}
Extensible Markup Language, also known as XML, is a markup language that defines a set of rules for encoding documents in a format that is both human-readable and machine-readable\cite{bib:xml}. Its purpose is to have a simple way of representing data in a textual format which is easy to learn and use. It is often used in programming as a medium for persisting data from objects into a file format which then can the be used later on. Many libraries have been written to make it easier to generate XML data.

\subsubsection{JSON}
JavaScript Object Notation, also known as JSON, is a text-based open standard designed for human-readable data interchange. This makes it quite like XML. The main difference is that its structure is much simpler than XML. The grammar is smaller and maps directly onto the data structures used in modern programmming languages\cite{bib:json}.


