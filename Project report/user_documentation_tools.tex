\section{Documentation tools}

\subsection{LaTeX}
LaTeX is a free high-quality typesetting system designed for writing technical and scientific documentation. It is most often used for medium-to-large documents, but can be used for almost any form of publishing. 
\newline
\newline
LaTeX is based on the idea that it is better to leave document design to document designers and to let authors get on with writing documents. So it encourages authors not to worry too much about the appearance of their documents, but to concentrate on getting the content right \cite{bib:latex}.
\newline
\newline
\textbf{Features of LaTeX:}
\begin{itemize}
\item{}Typesetting journal articles, technical reports, books, and slide presentations
\item{}Control over large documents containing sectioning, cross-references, tables and figures
\item{}Advanced typesetting of complex mathematical formulas with AMS-LaTeX
\item{}Automatic generation of bibliographies and indexes
\item{}Multilingual typesetting
\item{}Using PostScripts or Metafont fonts
\item{}Inclusion of artwork, and process or spot colour
\end{itemize}

\subsection{Google Docs}
Google Docs is a service for creating, storing and sharing files. The service is online and you can access your files from anywhere. Google Docs is free, unless you need more than 5GB of storage space. You can access your files everywhere you are; on the web, at home, at the office and on the go. It is great for sharing files with, for example, all the members of a project group. Google Docs is available for PC, Linux and Mac, Chrome OS, iPhone and iPad and other Android devices.

\subsection{Jira}
Jira is a project tracker for team planning, building and launching products. It is great for capturing and organizing issues, working through action items, and staying up-to-date with team activity. Jira is a proprietary issue tracking product commonly used for bug tracking, issue tracking, and project management \cite{bib:atlassian}.
\newline
\newline
Jira does only have a one month free trial, after that you have to pay a monthly fee to keep using it. How much depends on the maximum number of users \cite{bib:jira}.

\subsection{Greenhopper}
GreenHopper unlocks the power of Agile, whether you're a seasoned Agile expert or just getting started. Creating user stories, estimating those stories for a sprint backlog, identifying team velocity, visualizing team activity and reporting progress has never been so easy.
\newline
\newline
GreenHopper leverages JIRA's customizable issues and flexible workflow to provide an Agile project management tool that adapts as a team evolves \cite{bib:green}.
\newline
\newline
\textbf{Features:}
\begin{itemize}
\item{}Build a backlog: Quickly build a product backlog by creating user stories. Specify components, success criteria, business value or any other field the team uses to plan and execute work.
\item{}Order the backlog: Order the user stories and bugs in the product backlog by dragging and dropping the issues. Put those stories that deliver the most customer value at the top of the backlog.
\item{}Estimate: Add estimates to user stories (and bugs too if you wish) while in a planning meeting. Capture the estimates as you play planning poker.
\item{}Visually update progress: Team members can update the status of stories by dragging and dropping them or edit their details in the integrated pop up detail view. You can see the status of everything at a glance and you can do away with physical cardwalls.
\item{}Customize issue types, fields, and workflows to match your existing workflows and quickly adapt to changes as your processes evolve.
\item{}Continuous improvement: Visualize team process and identify bottlenecks as they emerge to respond quickly and address the problem.
\end{itemize}