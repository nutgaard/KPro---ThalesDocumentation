\section{Scrum}
Scrum is used in agile software development. Rather than beeing a full description of the process of program development, it is a framework setting the boundaries for the software development team [9]. The reason this is done is because the team knows best how to solve the task they are presented with.
\newline
\newline
Scrum relies on a self-organizing, cross functional team. This means that there is no team leader who decides who will do what. This also sets a boundary on the maximum size of team, which is about eight to ten persons. If the group consists of more persons than this, it should be divided in to 2 or more functional teams consisting of three or more persons.
\newline
\newline
The development cycle is created by basic units, called a sprint. These sprints last between one week and one month [10]. In the start of each sprint, there is a planning meating, where tasks are defined and goals are made. The tasks are defined from the backlog, and are refined into a task specification which can be performed by a programmer. During each sprint, a part of the completed product is made. It is not unusual to create a basic version of the complete software during the first sprint, and then add more functions as we go, during the later sprints.
\newline
\newline
Each day during a sprint starts with a Scrum meeting. The main purpose of this meeting is to give everyone a status report of what is going on. Each member of the group summarizes what he has done, what he is about to do, and what stands in his way of doing his tasks. These meetings have a maximum duration of 15 minutes, and should be done standing, as this keeps the talks short and effective.
\newline
\newline
The organization of the groups tasks is done by using the scrum task board. Here, one can see which tasks are unassigned, in progress, in testing and done. As the group consists of few persons, most of organization can be done via direct communication from person to person or during the meeting.

\newpage
We have chosen Scrum as our group organization model, as it fits this projects size and timeframe. There is a lot of uncertainty in our project regarding how we should organize the project from start to finish, as well as a limited time frame. The limited time frame forces us to make choices on which features we are able to implement.  But, as we are able to see what needs to be done in the near future (three weeks to one month), we can divide the project into sprints of this size and create a more detailed description of each task as we go.
\newline
\newline
Scrum is certainly a good choice of model.