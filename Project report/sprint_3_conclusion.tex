\section{Sprint 3 - Conclusion}
After finishing the third sprint we ended up with an application that satisfied most of the initial requirements presented by Thales. We were reasonably pleased with the way the application worked both at the backend and frontend. We considered XOXOmail user friendly and with a solid backend that supported modifiability. 
\newline
\newline
During the sprint we had many programmatic problems to solve, some of them were solvable in a convenient way, but some ended up being time consuming and difficult to solve. Even though a tasks could be simple in theory there, was a chance that bugs and unforeseen problems arised. An example is when a group member tried to fetch gps coordinates from the Android framework. This was initially functioning without any noticeable problems, but after testing it on different phones it was revealed that it did not function on all the Android devices we had available. This and other similar issues caused us to exceed the time planned on some of the tasks. 
\newline
\newline
There was a lot of work to be done in this sprint and due to the fact that programming tasks can be hard to give a time-span we ended up spending more time than we planned. The end result however, was more or less what we pictured when we started the sprint and we were pleased by the way it performed both visually and functionally. We ended up in the place we wanted to be before the final sprint.