

\subsection{Risk management}

\textbf{R1. Illness / Unavailability:} A member or members of the group become ill or unavailable. \\
\textbf{R2. Internal conflicts/disputes:} A conflict/dispute between two or more group members. \\
\textbf{R3. Problems with internal or external communication:} Misunderstandings between group members or with the customer. \\
\textbf{R4. Bad technical solution:}A bad technical solution/implementation is selected for the problem.. \\
\textbf{R5. Too much priority given to a certain task:} Too much effort is spent working on a task. \\
\textbf{R6. Technical incompetence:} The group does not possess the technical experience to solve a task. \\
\textbf{R7. Inexperienced with software development method:} The group does not have the needed experience with the software development method. \\
\textbf{R8. Hardware failure:} A hardware failure halts the projects progress. \\
\textbf{R9. Chosen tools do not deliver:} Development tools cause the project to slow down. \\

Explanation of abbreviations used in table% \ref{tab:risks}:
\begin{itemize}
\item{}Probability:
\begin{itemize}
\item{}L = low
\item{}M = medium
\item{}H = high
\end{itemize}
\item{}Strategy and actions:
\begin{itemize}
\item{}Reduce = how to reduce the consequence
\item{}Accept = accept that the risk has occourred
\item{}Avoid = how to avoid that the risk happens
\item{}Transfer = how to transfer the consequence from a risk that has happened, so that the consequent gets minimized
\end{itemize}
\end{itemize}

See table \ref{tab:risks} at page \pageref{tab:risks}







