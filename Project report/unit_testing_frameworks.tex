\section{Unit testing frameworks}

Unit testing is a method where individual units of source code are tested to determine if they are fit for use. These units consist of sets of one or more computer program modules together with their associated control data, usage procedures, and operating procedures \cite{bib:kolawa}. This section will describe the different unit testing frameworks available.

\subsection{JUnit}
JUnit is a testing framework for Java programming language, and is a member of the xUnit collective. Therefore a JUnit test fixture is a Java object, and all the test method must be annotated by the @Test annotation. JUnit is linked as a JAR at compile time. With JUnit you must write your own tests, and you can run many tests at the same time. JUnit is a great tool to use for testing Java code \cite{bib:junit}.

\subsection{Mockito}
Mockito is an open source testing framework for Java that allows creation of Test Double objects called “Mock Objects”. Mock Objects are simulated objects that mimic the behavior of real objects in a controlled way \cite{bib:mock}.
\newline
\newline
Mockito distinguishes itself from other mocking frameworks by allowing developers to verify the behavior of the system under test (SUT), without establishing any expectations beforehand \cite{bib:mockito}.
One of the criticisms of mock objects are that there is a tight coupling of the test code to the system under test \cite{bib:mocks}.

\subsection{GreenMail}
GreenMail is the first and only library that offers a test framework for both receiving and retrieving emails from Java.It is an open source, intuitive and easy-to-use suite of email servers used for testing purposes. It supports SMTP, POP3 and IMAP with SSL socket support \cite{bib:greenmail}.


\subsubsection{Sending and Retrieving}
GreenMail can easily be configured to use all or a combination of ports, protocols, and bind addresses. It  is possible to run GreenMail on SMTP, POP3, POP3S and IMAPS ports as easily as only with SMTP. Many systems might already be running these servers or don’t allow non root users to open the default ports which is why GreenMail ships with a special configuration for testing.

\newpage