
\chapter{Reflection}
\section{Tools we used}

\begin{tabularx}{\linewidth}{>{\setlength\hsize{.5\hsize}}X|>{\setlength\hsize{0.3\hsize}}X|>{\setlength\hsize{1\hsize}}X}
\textbf{Tool} & \textbf{Purpose} & \textbf{Way we used the tool} \\ \hline \hline

Gmail & Group communication &We used gmail for sending important messages to all the group members by using a gmail mailgroup.\\ \hline

Facebook & Group communication &We used facebook for fast communication about less important issues, such as where to work and small questions regarding a task or a document written by someone else.\\ \hline

GoogleDocs &Cooperative editing of documents &We started using google docs to make it easy to commonly edit different report documents. It worked pretty good, and a great thing with it is that many people can change one document at the same time. After a short period of time, we also discovered that it had a function for letting us see what changes had been made lately, and by whom.\\ \hline

DropBox & File sharing &Dropbox we decided to use for especially picture sharing, since sharing pictures aren’t that easy in GoogleDocs. In Dropbox we made a folder that all our group members had access to, and then we synced it.\\ \hline

GitHub &  Revision control&We used git for revision control of all our code and latex documents. Worked great throughout the entire project and enabled us to cooperate efficiently when writing code and latex documents.\\ \hline

TeXWork & Document writing &We used texwork for writing the final report in. We collected all our documents in a folder and included them all in a main file. It was pretty self explanatory, which was great since only one in our group had used it before.\\ \hline

NetBeans & IDE & We chose NetBeans as our IDE in this project and it worked without any significant problems during the entire project span. We were pleased with the services offered by this IDE.\\ \hline

Jira& Effort and tasks managing &Jira we decided to use for making tasks, assigning tasks and follow the tasks lifecycle. We also used Jira for effort registration. Jira was a bit time consuming, and we unfortunately used some more time at making tasks than necessary the first to sprints. But after using it for some time, Jira turned out pretty good. The effort management in Jira was very good all along. \\ \hline

Android Emulator &App testing & Slow but functional. Accompanied by testing on actual phones.\\ \hline



\end{tabularx}
	


\section{Internal Evaluation}
We have learned a lot from this project. Which is to say, we were a lot less efficient in the first few weeks than we were closer to the end. We all agreed on using an agile development model called scrum, and only a few of us were familiar with that way of working. It was pretty hard to estimate the effort we were supposed to put into each task during the planning, but we became better at it during the sprints. We stumbled a bit before we found our stride, and this section is the record of our improvement.  We have written this chapter together, to better reflect our different experiences and contributions.

\subsection{Teamwork}
While the norwegian school system provides its students with a near infinite supply of group projects, there is always more to learn about teamwork and group dynamics. Especially as few school projects are as long and comprehensive as this one. Clocking in at over three months and counting for half our grade this semester, it is of respectable size. In addition, it covers the full process of a workplace project far closer than a typical project, providing valuable experience on the required paperwork.
\newline
\newline
It has been a lot of trial and error during this project. None of us had any real experience with leading projects of this size. As the first week went by, the project leader was unofficially selected. This was approved. We agreed that the project leader would not be the boss, but have the responsibility of keeping the group together and resolve disputes between members. As commented by the project leader:
\newline
\newline
''Being a group leader has been a good experience. At first I saw this as a practical manager role; arranging room reservations, making choices on important issues, being customer and advisor contact and delegating tasks.
\newline 
\newline
What we saw during the first was that not all delegations of tasks were as good as they should be. The first Document responsible was not interested in being this so a new one was chosen.
\newline 
\newline
After a few meetings we saw that it was more natural that Agenda and Minutes of meetings responsible had the direct customer contact, to avoid the flow of connection going through an extra person. A responsible for booking rooms was also appointed.­­­­
\newline
\newline
Needless to say, we have had a lot of discussions. Unfortunately we had a lot of, what I would call blowouts. This is something I fully understand. Irritation has been expressed without the person in question present. This is something I am strongly against. Discussions should first be taken up with the person in question and then in plenum if needed.
\newline
\newline
Some of the complaints have been expressed about my leadership, but not to me in person. Nothing is more frustrating than seeing that something is destroying the group dynamics and not knowing it is I. This is why I initiated a talk with one of the group members, whom I felt, had a lot on their mind. This was a constructive discussion where many problems were solved and others arose. This led to another group member contacting me to tell what was on their mind. At this point I know that I had done something right. For me it was this I wanted all along: group members coming to me with thoughts and problems. This was not the case in the start. I clearly stated that all problems regarding the group should be taken with me first, via SMS or phone. Still, I got messages on Facebook and mail. This created a lot of communication problems.
\newline
\newline
The trend in the last phase of the project (sprint 3 and sprint 4) has been that every time I see that someone is frustrated, I ask: “What is bothering you?” In my opinion this has solved a lot of problems. It has made me able to see where the problems are and try to correct them before they get too big. I have certainly learned a lot during this project.''
\newline
\newline
We also had a person responsible for keeping track of the report. The report responsible had an overview of what was done, what had to be done and the general layout of the document. The following expresses some of the thoughts of the report responsible regarding teamwork:
\newline
\newline
''We were a group consisting of six very different people, both regarding work ethics and personality. That we had such a variety of people I think made the group better, but the different work ethics was very frustrating at times. It was not hard to see who were interested in working hard, and who weren’t. The effort spent on this project divided pretty fast into who were doing very much, who were doing only what’s expected, and who were doing not nearly enough. This trend unfortunately continued through the entire project, but the people who at the start did not do nearly enough started putting in some more effort later on. But till this day on, there are still those who work very very much, and I think that the difference in effort spent have made some discord in the group.''
\newline
\newline
We also had a product responsible that was supposed to try to see the customers view in choices made along the way. He also had some thoughts on teamwork:
\newline
\newline
''When first starting this project I did not have any expectations, but I was eager to get started. I knew little of what it involved to participate in a “real” project with an actual client, and looked forward to gain some experience in this setting.
\newpage
When I look back on this project I realize that I have gained a lot of insight into how it is to work on a “real” project with a group put together at random. I also had my first experience with practical agile development and gained a lot of insight into how it is to develop systems with this methodology. All in all I am very pleased with how this project turned out even though we had some problems in the beginning. The way the group evolved during the project was an important experience and brought to light many problems regarding group dynamics that should be avoided in future projects.''

\subsection{Reaching the project goal}
The goal of this project was to provide Thales with a working prototype and, less importantly for them, a set of documents describing the prototype and why we built it as we did. This we managed more than adequately. The prototype works as advertised and the documents are comprehensive. Even so, we had a number of difficulties getting to this point.
\newline
\newline
In the very start of the project we decided to give each group member different responsibilities. After approximately one week we made some changes to these responsibilities, but after that everybody had a responsibility that suited their personality.
\newline
\newline
At the start of the first sprint we used a lot of time on planning and setting up the programming environment. We saw that if we were to finish the project we had to do programming alongside with documentation. During this sprint it was a lot of trial and error. No one knew how the flow of management would be, so the sprint plan was so good as it should be. This is something we wanted to change in sprint two.
\newline
\newline
At the start of sprint two we had a much better sprint plan. We used the backlog and corresponding stories to create tasks and estimates. This made everyone more clear on what needed to be done during this sprint.
\newline
\newline
What became clear during the next sprints was that a lot of the estimates were way too low. We tried to do the estimates as accurate as possible based on the story points, but since we did not have enough experience a lot of the estimates were bound to be wrong. In a real business environment this would be a disaster for the company, but this is only a school project. The result was that a lot of extra effort had to be put into the project to be able to finish the sprints. It is this extra effort that has made it possible to finish the project with a good result. 
\newpage
The product owner had some thoughts regarding the ''road to the goal'':
\newline
\newline
''We started to work on the project shortly after the first introductory lecture and quickly agreed to adopt the SCRUM methodology as our work ethic. In the starting phase of the project we started to work on the application almost straight away. We quickly realized as the project went on that we should halt this part of the project for a bit, and focus more on the planning part. This was beneficial in the later phases of the project and gave us a better overview of what had to be done.
There was also a noticeable progress in how to group cooperated and communicated as the project went on. Routines and work methods became more or less second nature, and increased the productivity of the group. When the project neared it’s end, we ended up with an application prototype that contained more features than I hoped we were able to implement, and a report that contained most of the elements we wanted to include.
\newline
\newline
The main thing we should have done differently in this project is that we should have performed the pre study more thoroughly at an earlier stage in the project span. I’m not sure if this would have saved us some time, but it would make the final product much more clearer at an earlier stage in the development and maybe some of the programmatic choices we made along the way could have been done in a better way.
''
\newline
\newline
We did not finish all requirements that was requested.  

\section{Relations and communications with the customer}
We were lucky to have a customer that had participated in this kind of project before and so had a very good idea of what to expect. They were careful to communicate clearly with us about what they wanted out of the the project. They were helpful and understanding about the difficulties involved and offered possible solutions whenever we ran into a wall.
\subsection{Relations}
We worked well with our customer, agreeing on most issues. We generally had a good idea of where the project was headed and what they felt about our current progress. There were a few hiccups, but we generally functioned well. The group dynamic between us and Thales was good from the start.

\subsection{Communication}
Communicating efficiently is one of the most difficult things in the world, and this project was no exception. We had weekly meetings planned with the customer but in practice we ended up having meetings only every other week. Mostly due to the fact that we didn’t have enough questions to require weekly meetings.
\newline
\newline
That aside, the communication with Thales has been good and fruitful. As the representatives of Thales have a programming background, we often got into very technical discussions. For many of our group members this was from time to time a bit tedious as they could not follow the discussion. Still, this gave all of us a technical inside and everyone got a bit more involved in all parts of the application. It has never been a problem understanding what Thales wanted, and they have been very understanding and good at restraining themselves on what tasks we should implement. They have been very clear on what features that they wanted for the demo and what we should just skip because it had no relevance to what they were curious about in a prototype.
\subsection{The project}
This was an interesting project, quite outside our areas of expertise. Though most of us had worked with the Android platform before, the worlds of security and communications were all new to us. It took a bit of effort to familiarize ourselves with the required protocols and standards.
\newpage
A lot of us were surprised that the task would be so big. It all seemed so big and insuperable at first. As we dig deeper into the task, we saw that this was an interesting task that was really hands on. We all felt relieved that it was a project that was easy to see the path ahead, as we all knew what an email client is. It was all about creating a specialized version of one. 

\section{Lessons learned}
\subsection{Setting up our tools}
In this project we have used an extensive array of tools, most of them software based. Getting this set of tools to work on 6 different computers and three separate operating systems (not counting variations in version) was a living hell for the one group member who had experience with this.
\newline
\newline
We lost a large number of man hours in this setup phase, and it may not have been worth the increased efficiency we got out of the toolset. In a longer project, with a more experienced team, it certainly would have been, but things are more ambiguous in our case. It will however probably help us in the long run, as many of us will probably use those tools again.
\newline
\newline
On the other hand, these tools are useful for us; they handle everything from importing outside classes, to testing and version control. There is a lot of later work that would continue to bog us down that we have been saved by having more trouble in the beginning. It was probably worth it, but the technical responsible who spent the first three weeks helping us all set up might feel differently.
\subsection{Group communication on time use}
At the end of the first sprint we had a discussion on group effort and time use, spurred on by an uneven distribution of labor in the group. Over the course of the three week sprint, we have managed to arrive in a situation where the top two workers in the group had worked an average of twice the hours of the bottom two workers. This clear mismatch of commitment was thoroughly discussed over a couple of days.
\newline
\newline
We learned a lot from those two days. We agreed on a revised work schedule, better taking into account our individual ability to invest into the project. By getting everyone to commit to an individual weekly minimum hours of work, the average work per week went up significantly, even as members stopped feeling pressured to work more than they were capable of.
\newline
\newline
This illustrated the need for communication within the group. When some people feel pressured into working more than they are comfortable with while others feel disheartened because they have a perceived higher workload than the rest, it is clear that resentment will start to grow on both sides. By discussing it early, before tensions had gotten too high, we saved ourselves from facing a far greater challenge later.
\newline
\newline
However, while the problem has been thoroughly discussed, the underlying cause is still there. As can be seen from the work log of the later sprints, there is still a large deviation in hours of work for various group members. It remains to be seen if this will affect group morale as crunch time approaches.
\newline
\newline
Commented by the group leader: ''During the whole project I have been relying on the comments on my work from the other group members. I had no experience at the start, so in that way I was a wild card. But this was something I really wanted to do. What were most difficult to react to were all the different personalities in the group. We have people in the group who does not see the Customer Driven Project as important at all as well people who does what it takes to get a good result. After a few weeks we saw that the workload had to be better specified, so that the persons working more than the verbally agreed 20 hours didn’t get irritated. It is easier to accept what is agreed and written down. One group member has agreed to work 17 hours per week, and this is okay. We know that that is what he can contribute with.  This solved a lot of disputes, but it has been a recurring problem during the whole project. The weeks some persons at the most works over 30 hours, it is frustrating to see that others work just over 12 hours, even if they commit to 20 per week, just because they had other tasks to do.''

\section{Relations and communications with the advisor}

\subsection{Relations}
The relation with the advisors was not as extensive as the one we had with the customer. It was more about pointing us is the right direction with tips and general guidelines for performing such a project. 

\subsection{Communication}
The communication with the advisor has not been without problems. The first few weeks we had just some short meetings of 10-15 minutes. This was all we needed because we were working with a general plan for the whole project.
\newline
\newline
Unfortunately, we did not hear from our advisor for about 1,5-2 weeks because he did not receive our mails because of a mail-forwarding problem. This resulted in one advisor meeting being canceled. We tried to call him, but that did not work either.
\newline
\newline
After some time we saw that we would have benefited more from having more guidance from the advisor. We did not know what we could expect from him, so we sent a mail to course responsible, Reidar Conradi, and he said that he wanted to take part in the next two weeks sessions we had with our advisor.
\newline
\newline
These two weeks we got a lot out of the meetings. We got a fully reviewed version of our report with proposals on structure changes. Even though, we felt that course responsible and our advisor were working against each other. We were unable to have a structured presentation of the agenda, because they both asked questions to parts of the group. We were all happy that we now got thoughts and suggestions on our report, but the meetings were difficult to follow.
\newpage
\subsection{Supervision}
The supervision of our project was in the starting phases not as extensive as it could have been. This was mostly because of a misunderstanding involving how the meetings with the advisor were to be executed. The group was quite passive in the first few meetings and we had not prepared enough questions and things we were unsure of. After having a couple of meetings we realized that we needed to be more active and thus resulting in much more advice being given.
\section{Continuation of the project}
Our prototype, though it has the required functionality, can hardly be called complete. Most of the advanced security features are lacking, or at best partway implemented. The code is buggy and unstable, though less so every day we work on it. In the meantime, there are a number of ways the product could be improved further.
\subsection{Signing and encryption}
The biggest problem on the security side right now is that we have no verification of the identity of the mail sender. The code for it has been written, but has not been integrated or tested so significant work remains. For a company that already has experience with security programing this should not be significantly complicated, but was a bit more time consuming than we had anticipated.


\section{Suggestions for improvements for the project}
In any subject there is a possibility of improvement. There are always things that can be done better, clearer or more efficiently. So naturally we have some suggestions for how to run this course next year.
\newline
\newline
Firstly, it should be better guidelines on what to expect from the advisor. We had no experience in having these kind of meetings, and they often gave us nothing at all.
\newline
\newline
Second: Make sure all deadlines are correct and not lagging from previous year.


