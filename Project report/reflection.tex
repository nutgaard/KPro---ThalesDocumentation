

\chapter{Reflection}

This chapter is a reflection part that will describe what went well and what we could have done better in the project. Section \ref{sec:refl_tools} lists the different tools we have used as well as reflections on what these tools gave us of benefits and problems. Section \ref{sec:refl_internal} will give our thoughts on the teamwork and how we have worked to reach the goal. Section \ref{sec:refl_customer} will reflect on the relation and communication with Thales. Section \ref{sec:refl_advisor} will describe our relation and communication with our advisor. Section \ref{sec:refl_lessons} reflects on what lessons we have learned from the project. Section \ref{sec:refl_improve} gives suggestions for improvements of the course.

\section{The tools we used}\label{sec:refl_tools}

\begin{tabularx}{\linewidth}{>{\setlength\hsize{.3\hsize}}X|>{\setlength\hsize{0.5\hsize}}X|>{\setlength\hsize{1\hsize}}X}
\hline
\textbf{Tool} & \textbf{Purpose} & \textbf{Way we used the tool} \\ \hline \hline

Gmail & Group communication &We used Gmail for sending important messages to all the group members by using a Gmail mailgroup.\\ \hline

Facebook & Group communication &We used Facebook for fast communication about less important issues, such as where to work and small questions regarding a task or a document written by someone else.\\ \hline

Google Docs &Cooperative editing of documents &We started using Google Docs to make it easy to simultaneously edit different report documents. It worked pretty well, and a great thing with it is that many people can change one document at the same time. After a short period of time, we discovered that it also has a built-in version control.\\ \hline

DropBox & File sharing &We decided to use Dropbox for picture sharing, since sharing pictures is not that easy in Google Docs. In Dropbox we made a folder that all our group members had access to, and then we synced it.\\ \hline

GitHub &  Revision control&We used Git for revision control of all our code and Latex documents. This tool worked great throughout the entire project and enabled us to cooperate efficiently when writing code and Latex documents.\\ \hline

TeXWork & Document writing &We used TeXWork for writing the final report. We collected all our documents in a folder and included them all in a main file. It was pretty self explanatory, which was great since only one in our group had used it before.\\ \hline

NetBeans & IDE & We choose NetBeans as our IDE in this project and it worked without any significant problems during the entire project span. We were pleased with the services offered by this IDE.\\ \hline

Jira& Effort and tasks management&We decided to use Jira for making tasks, assigning tasks and follow the tasks lifecycle. We also used Jira for effort registration. Jira was a bit time consuming, and we unfortunately used more time at making tasks than necessary the first two sprints. But after using it for some time, Jira turned out to function pretty good. The effort management in Jira was very good all along. \\ \hline

Android Emulator &Application testing & Slow, but functional. Accompanied by testing on actual phones.\\ \hline

\end{tabularx}
	
\section{Internal Evaluation}\label{sec:refl_internal}
The learning curve of the project has been steep for all of us. Fortunately, we became a lot more efficient in the last part of the project than we were in the beginning. The agreed work methodology was scrum, and only a few of us were familiar with this method. Especially the estimating of the effort needed to complete each task during the planning phases was hard, but we got increasingly better also at this point. One could say that we stumbled a bit before we found our stride, and this section is the record of our improvement. We have all contributed to this chapter, to better reflect our different experiences and contributions.

\subsection{Teamwork}
While the Norwegian school system provides its students with a near infinite supply of group projects, there is always more to learn about teamwork and group dynamics. Especially as few school projects are as long and comprehensive as this project has been. Clocking in at over three months and counting for half our grade this semester, this project is of respectable size. In addition, it covers the full process of a workplace project far closer than a typical project, providing valuable experience on the required paperwork.

\newpage

We started off the project by talking through the background of each group member and their ambitions for the project. Some had their minds set on the best grade possible, whereas other would settle with a mediocre grade. We also discussed each person's experience with project work and technology. Based on these discussions we carved out the different roles needed and assigned these to the people that were most interested in the different roles. In retrospect, we realized that we could maybe have waited with delegating these roles and included more people in each work area. From this we might have distributed work loads better and given more persons a feel of what was going on in the different work areas. 
\newline
\newline
There has been a lot of trial and error during the course of this project. None of us had any real experience with leading projects of this size. As the first week went by, the project leader was unofficially selected. We agreed that the project leader would not be the boss, but have the responsibility of keeping the group together and resolve disputes between members. At first the leader position had a practical manager role; arranging room reservations, making choices on important issues, being customer and advisor contact and delegating tasks. What we saw during the first weeks was that not all delegations of tasks were as good as they should be. The first Document responsible was not really interested in having this role, so a new one was chosen. After a few meetings we saw that it was more natural that Agenda and Minutes of meetings responsible had the direct customer contact, to avoid the flow of connection going through an extra person. A responsible for booking rooms was also appointed.
\newline
\newline
We have had a lot of discussions in the during the project, and a lot of these were constructive. However, some of the group members did talk about their frustration and irritation without the person in question present. We discussed this and agreed to always broach issues with the person in question first and then in plenum if needed. The group leader became increasingly better at discovering when people had problems and asking them what was bothering them. The approach was to try to discover and solve the problems before they got too big. We also struggled a little with the digital communication, as some of the group members were available on Facebook and mail, whereas others prefered SMS or a phone call. In urgent cases this could create communication problems.
\newline
\newline
We were a group consisting of six very different people, both regarding work ethics and personality. This probably made the group better, but the different ambitions and willingness to put effort into the project created a lot of tension and was frustrating for some of the group members. Some people easily took responsibilities and ended up with a lot of work in that responsibility area. Some of the group members  were more passive and often needed others to tell them what to do. This trend contunied through the entire project, but people increased their efforts as we started the 3rd and 4th sprint. The differences in efforts spent has to a certain extent made some discord in the group. 
\newline
\newline
The differences in personality and morale quickly became evident, and manifested themselves often when deadlines came closer and the pressure became higher. However, we feel that the group as a unit has managed these disputes in an acceptable fashion. The tools we chose turned out to cause certain problems, and there was only one group member that had experience with the tools. This caused everyone coming to this person with technical questions and probably strangling this person's productivity. In retrospect, we do however feel like the stress was worth it, and that the tools did in fact return with profit.

\subsection{Reaching the project goal}
The goal of this project was to provide Thales with a working prototype and, less importantly for them, a set of documents describing the prototype and why we built it as we did. We came a long way in reaching this goal. The prototype contains the most important requirements and the documentation is comprehensive. Even so, we had a number of difficulties getting to this point.
\newline
\newline
We started to work on the project shortly after the first introductiry lecture and quickly agreed to adopt the Scrum methodology as our work ethic. In the starting phase of the project we started to work on the application almost straight away. We quickly realized as the project went on that we should halt this part of the project for a while, and focus more on the planning part. This was beneficial in the later phases of the project and gave us a better overview of what had to be done.
\newline
\newline
At the start of the first sprint we used a lot of time on planning and setting up the programming environment. We saw that we were to finish the project we had to do programming and documentation tasks simultaneously and continuously. During the first sprint it was a lot of trial and error. No one knew exactly how the flow of management should be, so the sprint plan was as good as it could be at this point.
\newline
\newline
At the start of the second sprint we were able to create a much better sprint plan. We used the backlog and corresponding stories to create tasks and estimates. This made it more clear for all group members what needed to be done during this sprint. 
\newline
\newline
What became clear during the next sprints was that a lot of the estimates we created were way too low. We always tried to create estimates as accurate as possible based on the story points, but due to the lack of experience these were bound to be wrong. In a real business environmen this would be a disaster for the company, but in our case we just had to spend more hours completing tasks and hence not having time to implement everything we wanted. We had to put a lot of extra effort into each sprint to be able to finish what we had planned. This extra effort has strongly contributed to finishing the project with a good result.
\newline
\newline
There was a noticeable progress in how the group cooperated and communicatd as the project went on. Routines and work methods became more or less second nature, and increased the productivity of the group. When the project came close to the end, we ended up with an application prototype that contained most of the features we had planned to implement, and a report that contained most of the elements we wanted to include.

\newpage

\section{Customer relations}\label{sec:refl_customer}
We were lucky to have a customer that had participated in this kind of project before and so had a very good idea of what to expect. They were careful to communicate clearly with us about what they wanted out of the the project. They were helpful and understanding about the difficulties involved and offered possible solutions whenever we ran into an obstacle.
\subsection{Relations}
We worked well with our customer, agreeing on most issues. We generally had a good idea of where the project was headed and what they felt about our current progress. There were a few hiccups, but we generally functioned well. The group dynamic between us and Thales was good from the start.

\subsection{Communication}
Communicating efficiently can be one of the most difficult things in the world, and this project was no exception. We had weekly meetings planned with the customer, but in practice we ended up having meetings only every other week. This was mostly due to the fact that we did not have enough questions to require weekly meetings.
\newline
\newline
That aside, the communication with Thales has been good and fruitful. As the representatives of Thales have a programming background, we often got into very technical discussions. For many of our group members this was from time to time a bit tedious as they could not follow the discussion. Still, this gave all of us a technical insight and everyone got a bit more involved in all parts of the application. It has never been a problem understanding what Thales wanted, and they have been very understanding and good at restraining themselves on what tasks we should implement. They have been very clear on what features that they wanted for the demo and what we should just skip because it had no relevance to what they were curious about in a prototype.

\subsection{The project}
This was an interesting project, quite outside our areas of expertise. Though most of us had worked with the Android platform before, the worlds of security and communications were all new to us. It took a bit of effort to familiarize ourselves with the required protocols and standards.
\newline
\newline
A lot of us were surprised that the task would be so comprehensive, as it all seemed so big and insuperable at first. As we dug deeper into the task, we saw that this was an interesting task that was really hands on. We all felt relieved that it was a project in which it was easy to see the path ahead, as we all knew what an email client is. It was all about creating a specialized version of one. 

\newpage

\section{Advisor relations}\label{sec:refl_advisor}
This section will elaborate about the relations we had with our advisor.

\subsection{Relations}
The relation with the advisors was not as extensive as the one we had with the customer. It was more about pointing us is the right direction with tips and general guidelines for performing such a project. 

\subsection{Communication}
The communication with the advisor has not been without problems. The first few weeks we had just some short meetings of 10-15 minutes. This was all we needed because we were working with a general plan for the whole project.
\newline
\newline
Unfortunately, we did not hear from our advisor for about 1,5-2 weeks because he did not receive our mails because of a mail-forwarding problem. This resulted in one advisor meeting being canceled. We tried to call him, but were not able to reach him.
\newline
\newline
After some time we saw that we could have benefited from having more guidance from the advisor. We did not know what we could expect from him, so we sent a mail to course responsible, Reidar Conradi, and he said that he wanted to take part in the next two weeks sessions we had with our advisor.
\newline
\newline
These two weeks we got a lot out of the meetings. We got a fully reviewed version of our report with proposals on structure changes, and we were happy that we got thoughts and suggestions on how to improve the report. After these weeks we also got more out of the meetings with our initial advisor, as we had learned more about what we could expect from him.

\subsection{Supervision}
The supervision of our project was in the starting phases not as extensive as it could have been. This was mostly because of a misunderstanding involving how the meetings with the advisor were to be executed. The group was quite passive in the first few meetings and we had not prepared enough questions and things we were unsure of. After having a couple of meetings we realized that we needed to be more active and thus resulting in much more advice being given.

\newpage

\section{Lessons learned}\label{sec:refl_lessons}
This section will discuss what lessons we have learned from this project and what we might have done differently if we were to start a new project. These lessons come in addition to the experience we have learned from using the technical tools and programming experience.

\subsection{Setting up our tools}
In this project we have used an extensive array of tools, most of them software-based. Getting this set of tools to work on six different computers and three separate operating systems (not counting variations in version) created a lot of work for the one group member who had experience with this.
\newline
\newline
We lost a large number of man hours in this setup phase, and it might not have been worth the increased efficiency we got out of the toolset. In a longer project, with a more experienced team, it certainly would have been, but things are more ambiguous in our case. It will however probably help us in the long run, as many of us will probably use those tools again.
\newline
\newline
On the other hand, these tools were useful for us; they handle everything from importing outside classes, to testing and version control. There is a lot of later work that would continue to bog us down that we have been saved by having more trouble in the beginning. It was probably worth it, but the technical responsible who spent the first three weeks helping us all set up did probably not always feel that way.

\subsection{Group communication and effort}
At the end of the first sprint we had a discussion on group effort and time use, spurred on by an uneven distribution of labor in the group. Over the course of the three week sprint, we have managed to arrive in a situation where the top two workers in the group had worked an average of twice the hours of the bottom two workers. This clear mismatch of commitment was thoroughly discussed over a couple of days.
\newline
\newline
We learned a lot from those two days. We agreed on a revised work schedule, better taking into account our individual ability to invest into the project. By getting everyone to commit to an individual weekly minimum hours of work, the average work per week went up significantly, even as members stopped feeling pressured to work more than they were capable of.
\newline
\newline
This illustrated the need for communication within the group. When some people feel pressured into working more than they are comfortable with while others feel disheartened because they have a perceived higher workload than the rest, it is clear that resentment will start to grow on both sides. By discussing it early, before tensions had gotten too high, we saved ourselves from facing a far greater challenge later.
\newline
\newline
However, while the problem has been thoroughly discussed, the underlying cause is still there. As can be seen from the work log of the later sprints, there is still a large deviation in hours of work for various group members. It remains to be seen if this will affect group morale as crunch time approaches.

\subsection{Project organization}
The main thing we should have done differently in this project is that we should have performed the pre-study more thoroughly at an earlier state in the project span. We are not sure if this could have saved us any time, but it would make the final product much more clear at an earlier stage in the development and maybe some of the programmatic choices we made along the way could have been done in a better way. We also did the mistake of starting the sprints at Mondays and then not having the customer meetings until Wednesday. The hours in between were spent partly on fixing things from the previous sprint and partly on trying to plan ahead for the newly started sprint, although this was not always easy as we had discussed it with Thales.

\section{Suggestions for course improvements}\label{sec:refl_improve}
In any subject there are possibilties for improvement. There are always things that can be done better, clearer or more efficiently. In this section we will give our thoughts on this matter.
\newline
\newline
Firstly, we think that the course could give better guidelines as to what to expect from the advisor. We had no experience in having these kinds of meetings, and in the beginning the meetings often were not so helpful. 
\newline
\newline
Secondly, we would have appreciated if the lectures had taken place in the original course time, as this would have made it easier to plan.
\newline
\newline
Thirdly, we have discovered that some of the deadlines mentioned in the compendium were not always correct and probably lagging from previous year.
\newline
\newline
We would also have liked to have a fixed meeting room for all our meetings. Having a room with a board that we could use as a scrum board would have helped us a lot.
