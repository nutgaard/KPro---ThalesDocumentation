\section{Sprint 1 - Backlog}

\begin{itemize}
\item{}\textbf{Setup of Jira:}we decided to use Jira to get an overview over all the work that we need to do. Jira also gives us an overview over what tasks are supposed to be done in which sprint. We also use Jira to assign the tasks to each of the team members.
\item{}\textbf{Report work:} the report that is to be delivered at the end of the project needs a lot of work and we have decided that we will use time on the report in every sprint.
\item{}\textbf{Set up programming environment:} working with setup of software in an administrative sense.
\item{}\textbf{Group Administration:} to administrate on bhalf of the entire group, with booking group rooms and sending e-mails to the customer, the advisor and the team to call in for meetings.
\item{}\textbf{Meetings:} arranging meetings and the time we use on meetings.
\begin{itemize}
\item{}\textbf{Meetings with Thales:} we have weekly meetings with our customer so that we can get rapid feedback on what we do. To know if they agree with our decisions and that we haven’t misunderstood the tasks they have given us.
\item{}\textbf{Meetings with Mohnsen Anvaari:} we have regular meetings with our advisor so that he can give us feedback on how we do our work, and make sure that we are doing what are expected of us in the course.
\item{}\textbf{Internal meetings:} we have almost daily meetings to learn what everybody has been doing lately, and how far we have come in our tasks; what is left and what is done.
\item{}\textbf{Lectures:} there are some lectures during the semester, and we are adviced to participate in these. In this sprint there are 3 lectures and courses that we have decided to attend.
\end{itemize}
\item{}\textbf{Create interfaces between Core and GUI:} code interfaces that let GUI and core communicate in the simplest way. We need this code to be able meet our Sprint 1 milestone. We also need these interfaces to make a simple model.
\begin{itemize}
\item{}\textbf{Persistence Service Interface:} preliminary design of interfaces for the Persistence Service module.
\item{}\textbf{Hardware abstraction layer interface:} preliminary design of interfaces for the HAL Service module.
\item{}\textbf{Network Service Interface:} preliminary design of interfaces for the Network Service module.
\item{}\textbf{Security Service Interface:} preliminary design of interfaces for the Service Service module.
\end{itemize}
\item{}\textbf{General setup of tools:} general setup of different tools we use.
\item{}\textbf{Meeting and agenda document writing:} to write all the meeting agendas and minutes using the right template.
\item{}\textbf{Starting App:} making our program good enough so that it is possible to start the program and begin browsing all features of the program.
\begin{itemize}
\item{}\textbf{Create Main menu:} make a simple main menu with a header showing the name of the program and a list containing all views it is possible reach from the menu.
\item{}\textbf{Create a basic android app skeleton:} make an Android app project so that we have a running application with nothing in it.
\item{}\textbf{Learn how Android MVC works:} find out how Android MVC works and get a general feeling of how the layout of the Android project will be.
\end{itemize}
\item{}\textbf{Sending a message:} the user should be able to click the “New message” button, be brought to the new message page, create a message and send it pressing the “Send” button.
\begin{itemize}
\item{}\textbf{Adding subject and text to a message:} create a simple GUI to make the user able to create a message with a title and a text and sending it to a recipient. At first it is enough to send to a predefined receiver until the address book is made.
\item{}\textbf{Make connection between “Send Message” button and backend:} make the GUI communicate with the backend responsible for sending the actual message.
\item{}\textbf{Implement a Network class for sending e-mail through gmail’s smtp-server:} implement an instance of the NetworkService interface which sends mail via gmail’s mail servers.
\item{}\textbf{Create the new message view:}make it possible for the user to get a view showing all fields relevant to creating a message by clicking “New message”.
\item{}\textbf{Implement receiving mail from gmail’s imap-service:} make the app able to receive the mail automatically from Gmail’s IMAP, as soon as a message is received at the account. This must be done via push to client, not constant pulling.
\item{}\textbf{Create core bridge:} make the connection from GUI to core and implement return value from interface on core side.
\end{itemize}
\item{}\textbf{Persisting data to phone:} Implement persistence, so that the app is able to save data and retrieve it whenever it wants.
\begin{itemize}
\item{}\textbf{Research on structure for saving and reading data}: find a structure for saving data to the phone. What is the best way of organizing data with respect to files created by app, file settings and other files?
\item{}\textbf{Save data to phone storage:} ensure proper saving of data to phone.
\item{}\textbf{Read data from phone:} ensure proper saving of data to phone.
\end{itemize}
\end{itemize}