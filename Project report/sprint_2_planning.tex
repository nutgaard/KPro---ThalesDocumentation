\chapter{Sprint 2}

\section{Sprint 2 - Planning}
Finishing the first sprint left us with a working demo that could send and receive e-mails. After meeting with Thales and showing them the demo, we came to the conclusion that we needed to perform a thorough study regarding the security aspect of the application. The security issues that were most vital to the functionality of the application were local storage of program data and secure sending with SSL. We also realized that we needed more concise documentation of the architecture, which was a bit vague at this point. This was something which we did intentionally, as we weren’t sure what we could implement and what would be left as a pre- and poststudy. 
\newline
\newline
We knew that the second sprint would be hectic. We did not have much room for extra workloads, as the regular documentation work takes a lot of time. Thales request for a security prestudy did not make it easier, but we agreed to continue at a higher pace with both the frontend and backend part of the application. The GUI of the frontend was almost non existing, so it had to be developed from scratch within just one sprint. One of the biggest goals was to implement the ability to add attributes to outgoing messages, so that they could be prioritized on these attributes, as well as getting a GUI with a more complete look and feel.
\newline
\newline
Sprint two had a lot of time set aside for further documentation. This was crucial for being able to get better progress in the later sprints regarding programming. Most members of the project group were set to work on the documentation while two of the developers continued to work with the application. 