\chapter{Introduction}

This chapter is a technical introduction to the project. The purpose of this chapter is to give an overview of the problem that we solved and the most important terms used in the report.

\subsection*{The existing product}
Thales has developed an information handling and transfer system called XOmail \cite{bib:xomail}, which has been in operational use for more than 20 years. In many ways, it could be compared to the regular, well-known email system. The first and biggest difference between regular email systems and XOmail is that the latter is developed with large formal organizations and military messaging procedures in mind. This is the reason why Thales has been delivering XOmail as a complete product to several European \gls{nato} member countries.  
\newline
\newline
The messages transferred in the system are defined to have certain special attributes, where the most important is a security label. A security label is a declaration of the required clearance level the receivers of a message need to have. The security label specifies that only people or groups with the required or higher security clearance can see the message. Security is an important feature of XOmail, and it has never lost a message in 20 years of operation.

\subsection*{What Thales wanted us to make}
Thales’ existing product is a complete messaging platform for messaging in large organizations, but the current system is primarily made for use on computers with relatively large screens. What Thales wanted to know was if it was possible to create a mobile messaging platform building on the existing infrastructure already in use by XOmail. It is important to note that this mobile platform would  only have a subset of the functions that the original system has and the main focus was on security, efficiency and ease of use.
\newline
\newline
Since Thales had not tried to implement XOmail on mobile platforms, there was no existing framework to start with. Thales wanted a prototype that demonstrated a possible solution to their problem. Security was of great importance, but the main focus was placed on implementing functionality that was critical for showing the concept software.  

\newpage

Even though XOmail primarily targets the military and other large organizations where security and security declarations are important, we had to keep in mind that this new software also should be well suited for other groups or organizations that have a need for a user-friendly portable system for sending urgent messages. A few of the possible candidates include paramedics, security firms and joint civilan-military operations or exercises.
\newline
\newline
For a summary of what Thales already had and what they wanted, see the table below.

\begin{table}[h!]
\begin{tabularx}{\linewidth}{>{\setlength\hsize{.5\hsize}}X|>{\setlength\hsize{0.5\hsize}}X}\hline
\textbf{What Thales has} & \textbf{What Thales wants}\\ \hline \hline
An information handling and transfer system developed for large formal organizations and military messaging procedures&A system for handheld devices with a subset of the existing functions\\ \hline
A large number of ways of giving a message attributes&Focus on security label, priority and type\\ \hline
Extensive focus on security and reliability&Focus on security and usability\\ \hline
Support for a broad range of attachments&Support for images, text and video\\ \hline
\end{tabularx}
\caption{Thales' current and wanted situation}
\label{tab:introcomparison}
\end{table}

\subsection*{A short overview of the standards of XOmail}
XOmail is based on the \gls{nato} military messaging standard \gls{stanag}. This standard is used for both Strategic and Tactical messaging \cite{bib:stanag}. It has a number of special protocols used to support tactical messaging to support links with very low bandwidth.
\newline
\newline
The standard is based on binary encoded data, and there are limited libraries freely available for these protocols. Special attributes like information sensitivity and priority are defined by extensions to the Internet Mail RFCs. Our connection to Thales’ system goes through a XOmail \gls{smtp} gateway. \gls{smtp}, which is shorthand for Simple Mail Transfer Protocol \cite{bib:smtp}, is an Internet standard for electronic mail transmission over \gls{ip} networks. Messages to other users are sent by sending an email to the server with an address, and then the server handles the message headers and other important attributes. The messages are then pushed to the correct recipients via \gls{pami} \cite{bib:imap}, which is an abbreviation for Internet Message Access Protocol, and is one of the most prevalent Internet standard protocols for email retrieval. 


