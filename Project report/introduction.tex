\chapter{Introduction}

This chapter is a technical introduction to the project. The purpose of this chapter is to give an overview of the task to be solved and the most important terms used in the report.

\subsection*{The existing product}
Thales has developed an information handling and transfer system called XOmail[2], which has been in development and use for more than 20 years. In many ways, it could be compared to a regular, well-known email system. The first and biggest difference between regular email and and XOmail is that XOmail is developed to handle groups of people, not just individuals. This feature makes the system ideal for large organizations where a message is intended for several receivers, such as a message from the manager to the employees of his/her department. An example of such an organization is the military. This is the reason why Thales has been delivering XOmail as a complete product to several European NATO member countries.  
\newline
\newline
The messages that are sent are defined to have a security label, which is a declaration of the required clearance level the receivers must have. The security label specifies that only people or groups with the required (or higher) security clearance can see the messages. Security is an important feature of XOmail, and it has never lost a message in 20 years of operation. 

\subsection*{What Thales wants us to make}
Thales’ existing product is a complete messaging platform for messaging in large organizations, but the current system is primarily made for use on typical computers with relatively large screens. What Thales wants to know is if it is possible to create a mobile messaging platform building on the existing infrastructure already in use by XOmail. It is important to note that this mobile platform will only have a subset of the functions that the original system has and the main focus is on security, efficiency and ease of use.
\newline
\newline
Since Thales has not tried to implement XOmail on mobile platforms, there is no existing framework to start with. Thales wants a prototype that demonstrates a possible solution to their problem. Security is of great importance, but the main focus should be on implementing functionality that is critical for showing the concept software.  
\newline
\newline
Even though XOmail primarily targets the Military and other large organizations where security and security declarations are important, it is important to keep in mind that this software also should be well suited for other groups or organizations that have a need for a user-friendly portable system for sending urgent messages. A few of the possible candidates include paramedics, security firms etc. 

\subsection*{A short overview of the standards of XOmail}
XOmail is based on the military messaging standard STANAG 4406. This standard is used for both Strategic and Tactical messaging[1]. It has a number of special protocols used to support tactical messaging to support links with very low bandwidth.
\newline
\newline
As XOmail is a large software project developed in-house by Thales, there are a lot of business secrets around, and a lot of processing happens behind closed curtains. Our connection to Thales’ system goes through a XOmail SMTP gateway. SMTP which is shorthand for Simple Mail Transfer Protocol [4], is an Internet standard for electronic mail transmission over IP networks. Messages to other users are sent by sending an email to the server with an address, and then the server handles the message headers and other important attributes. The messages are then pushed to the correct recipients via IMAP [3], which is an abbreviation for Internet Message Access Protocol and is one of the most prevalent Internet standard protocols for email retrieval. 

