\section{Sprint 2 - Ordered sprint backlog}

\begin{itemize}
\item{}\textbf{Report work:} The report that is to be delivered at the end of the project needs a lot of work, and we have decided that we will use time on the report in every sprint.
\item{}\textbf{Group Administration:} To administrate on behalf of the entire group, to book group rooms and send emails to Thales, the advisor and the team to call in for meetings.
\item{}\textbf{Meetings:} Arranging meetings and the time we use on meetings.
\begin{itemize}
\item{}\textbf{Meetings with Thales:} We have weekly meetings with Thales so that we can get rapid feedback on what we do. To know if they agree with our decisions and that we haven't misunderstood the task thay have given us.
\item{}\textbf{Meetings with Mohnsen Anvaari:} We have regular meetings with our advisor so that he can give us feedback on how we are doing our work, and make sure that we do what are expected of us in the course.
\item{}\textbf{Internal meetings:} Every time we meet we first have a status meeting were we share what we have done and how far we have come with our tasks.
\item{}\textbf{Lectures:} There are some lectures during the semester, and we are adviced to paticipate in these. In this sprint there are 2 lectures and courses that we have decided to attend.
\end{itemize}

\newpage

\item{}\textbf{Browse previously sent messages:} Make it possible to see all messages that have been previously sent, so that it is possible for the user to check that they actually were sent.
\begin{itemize}
\item{}\textbf{Show basic listing of all elements in list:} A basic listing of all elements should be implemented, and each data element in the list should have all necessary fields that are required.
\item{}\textbf{Style elements in list:} Use design theory to make all the elements look presentable.
\end{itemize}
\item{}\textbf{Meeting and agenda document writing:} Write all the meetings agendas and minutes using the right template.
\item{}\textbf{Sending a message:} The user should be able to click the \textsc{New message} button, be brought to the new message page, create a message and send it pressing the \textsc{Send} button.
\begin{itemize}
\item{}\textbf{Implement metadata structure and show it:} Implement all the fields that are required to send a message; from, to, classification, label, priority, and make them look presentable.
\end{itemize}
\item{}\textbf{Browse inbox:} The user should be able to show a list of all received messages, both read and unread.
\begin{itemize}
\item{}\textbf{Show basic listings of all messages received:} Implement a basic list of all elements, and make sure that all necessary fields that are required are displayed.
\item{}\textbf{Style elements in list:} Style each element of the list, so that it is pleasing to look at.
\end{itemize}
\item{}\textbf{Read message:} Making the user able to read a message.
\begin{itemize}
\item{}\textbf{Showing all metadata:} Implement all fields that are related to a message, and display them.
\item{}\textbf{Showing message subject and text:} Show all fields that are related to a message; date received, from.
\item{}\textbf{Reply/Delete/Forward:} Implement the buttons so that the user can reply to a message, delete a message and forward a message.
\end{itemize}
\item{}\textbf{Browse outbox:} The user should be able to show a list of all messages that are waiting to be sent, or haven’t been sent yet.
\begin{itemize}
\item{}\textbf{Show basic listing of all elements in list:} Implement a basic listing of all elements, and that each data element in the list has all necessary fields that are required.
\item{}\textbf{Style list for pretty viewing:} Make the list pleasing to look at.
\end{itemize}
\item{}\textbf{Bottom menu bar:} Implement the bottom menu bar so that a user can switch between pages using the menu bar.
\begin{itemize}
\item{}\textbf{Create bottom menu bar:} Implement a menu bar and make it presentable and useful by applying design theory.
\end{itemize}

\newpage

\item{}\textbf{Security Pre-study:} Dig deeper into the security aspects of Android, in order to make well informed decisions.
\begin{itemize}
\item{}\textbf{Secure sending:} Write security documentation on secure sending.
\item{}\textbf{Secure storage on phone:} Write documentation regarding secure storage of data and password.
\item{}\textbf{Signing and verification:} Find out how secure signing works, and write documentation on it.
\item{}\textbf{Limitations and facilities of Android:} What are the limitations regarding Android security, and what does Android facilitate when it comes to encapsulation? Is data in memory secure? Find the answers to the questions and document it.
\item{}\textbf{Security requirements:} Write a summary of the documentation linked to by Tellefsen.
\end{itemize}
\item{}\textbf{Architecture documentation:} Make an architectural description document, so that we know that the code fulfills all the requirements.
\begin{itemize}
\item{}\textbf{Fulfillment of requirements:} Describe how our architecture fulfills the security, usability and latency requirements.
\item{}\textbf{Graphical view of architecture:} Create a graphical view of the architecture, both backend and frontend.
\item{}\textbf{Frontend:} Document the frontend properly.
\item{}\textbf{Backend:} Document the backend properly.
\end{itemize}
\end{itemize}
