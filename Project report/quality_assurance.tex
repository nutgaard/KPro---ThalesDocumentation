\section{Quality Assurance}

\subsection{Time of response}
\begin{itemize}
\item{}Agenda and questions must be sent to customer within 24 hours before the customer meeting 
\item{}Minutes of meeting should be sent to customer within 48 hours after the meeting
\item{}Approval of minutes of customer meeting should be done within X hours
\item{}Feedback on phase documents the customer would like for review should be given within X hours
\item{}Approval of phase documents should be done within X hours
\item{}Time to get agreed documents is X hours
\end{itemize}


\subsection{Routines for producing high quality internally}
To secure high quality internally we made sure every item produced was reviewed by a team member that did not work on set item. By doing this the quality of items would be higher and the group members would get better control of the entire project and not just the parts they worked on themselves. 

\subsection{Routines for approval of phase documents}
To make sure the different documents throughout the phases had the quality we were looking for, we had a group member that had the overall responsibility of putting together the different parts of the documents and at the same time reviewing the quality of the documents. We also made sure the customer and advisor got a chance to view the important documents that would affect the direction of our project. In this way we would receive vital feedback and make the necessary changes according to the response of the advisor or customer. The visibility of our documents and progress is a part of the SCRUM method.   

\subsection{Calling for a meeting with the customer}
When calling a meeting we should specify:
\begin{itemize}
\item{}The date, time and place of the meeting.
\item{}The reason for the meeting.
\item{}The agenda we are going to go through.
\item{}All the documents that are vital for the meeting.
\item{}The preparations needed to be done either by the group or the customer.
\end{itemize}
We agreed with the customer to call meetings at least 24 hours ahead of the time scheduled. The call for a meeting should also include the minutes from the last meetings. 

\subsection{Minutes of a customer meeting}
The minutes of the weekly meeting with the customer should concede of:
\begin{itemize}
\item{}The name of the project.
\item{}Where the meeting takes place.
\item{}The time and date of the meeting.
\item{}The name of the meeting responsible.
\item{}The name of all attendees.
\item{}Topics of the meeting.
\item{}Decisions made.
\item{}Actions agreed upon.
\item{}Clarifications.
\item{}Time, date and place of next meeting.
\end{itemize}
The minutes of the weekly meeting should be sent to the customer by 12:00 o’clock the following day and then be approved. If the minutes are not approved they should be rewritten and sent again for approval. 

\subsection{Calling for the weekly advisor meeting between the group and its advisor}
The meeting with the advisor should be every tuesday at 11:00 o’clock if no cancellation has been made.\\
\\
A meeting with the advisor should be scheduled 48 hours before the time of the meeting and the calling of the meeting should include:
\begin{itemize}
\item{}The name of the project.
\item{}Date and time of the meeting.
\item{}The agenda of the meeting.
\item{}A status report.
\item{}A table of reported working hours.
\item{}Minutes from the last meeting with the advisor.
\item{}Other documents vital for the meeting.
\end{itemize}

\newpage
\subsection{Agenda for the weekly meeting with the advisor}
The course planner gave us a template that is to be followed:
\begin{enumerate}
\item{}Approval of agenda
\item{}Approval of minutes of meeting from last advisor meeting
\item{}Comments to the minutes from last customer meeting or other meetings
\item{}Approval of the status report, which may be structured as follows:
\begin{enumerate}
\item{}Summary
\item{}Work done in this period
\subitem{}Status of the documents that are being created
\subitem{}Meetings
\subitem{}Other activities
\item{}Problems – what is interfering with the progress or taking resources? Problems are often risks that have taken effect.
\item{}Planning of work for the next period
\subitem{}Meetings
\subitem{}Activities
\item{}Other
\end{enumerate}
\item{}Review/approval of attached phase documents
\item{}Other issues are listed here…
\item{}Other issues
\end{enumerate}

\newpage
\subsection{Minutes of the weekly meeting with the advisor}
The minutes of the weekly meeting with the advisor should concede of:
\begin{itemize}
\item{}The name of the project.
\item{}Where the meeting takes place.
\item{}The time and date of the meeting.
\item{}The name of the meeting responsible.
\item{}The name of all attendees.
\item{}Topics of the meeting.
\item{}Decisions made.
\item{}Actions agreed upon.
\item{}Clarifications.
\item{}Time, date and place of next meeting.
\end{itemize}
The minutes of the weekly meeting should be sent to the advisor by 12:00 o’clock the following day and then be approved. If the minutes are not approved they should be rewritten and sent again for approval. 


