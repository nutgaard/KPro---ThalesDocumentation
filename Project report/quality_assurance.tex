\section{Quality assurance}
This section explains what we have planned to do to ensure quality in the project. Quality assurance is important to establish routines, not duplicate work or never lose any of the work. We want to ensure that all group members know how to do things, use the correct format and keep the deadlines.

\subsection{Time of response}
Agenda and questions for the meetings must be sent to the customer no later than 24 hours before the customer meeting is scheduled. 
Minutes of meeting should be sent to customer within 48 hours after the meeting, but preferably as soon as possible.
Approval of minutes of customer meeting should be done by the rest of the group before sending it to the customer. 
The customer should comment and approve of the minutes within 48 hours after receiving it from the group. 
Agenda and weekly documents should be sent to the advisor before Mondays at 14:00.
We should get approval and feedback of the documents sent to the advisor within 72 hours of having sent him the documents.
\newline
\newline
For a table of the agreed response times, see \ref{tab:responsetable} at page \pageref{tab:responsetable}.

\begin{table}[hbt]
\begin{center}
\begin{tabular}{l|l} \hline
\textbf{What} & \textbf{Time of response} \\ \hline \hline
Approval of minutes of customer meeting & 48 hours \\
Feedback on weekly documents & 72 hours \\
Approval of weekly documents & 72 hours \\
Answer to a question & 48 hours \\
Producing requested documents & 48 hours after approving request \\ \hline
\end{tabular}
\end{center}
\caption{Time of response table}\label{tab:responsetable}
\end{table}


\subsection{Routines for producing high quality internally}
To ensure high quality internally we will make sure every item produced is reviewed by a team member who did not work on the given item. We have planned to do this to increase the quality of the items and give the group members a better feel of the entire project and not just the parts they work on themselves.
\newline
\newline
Most of the working hours are planned together with other group members, either the entire team or in groups of two or three people. We want to ensure that there should always be someone to ask if any questions comes up, either in the 18 hours we have scheduled to meet physically each week or by communicating electronically. 

\newpage

\subsection{Routines for approval of phase documents}
Making sure that the different documents throughout the phases have the desired quality is a high priority for us. Therefore we have allocated one group member with the overall responsibility of putting together the different document parts and simultaneously reviewing of the documents. 
\newline
\newline 
We also want to make sure that the customer and advisor get opportunities to view the important documents that would affect the direction of our project. This way we will receive vital feedback and make the necessary changes according to the response of the advisor or customer. 
\newline
\newline
The visibility of the documents and progress is a part of the scrum method, meaning that the customer always has insight into what the group is doing and how far the group has progressed.

\subsection{Calling for a meeting with the customer}
In advance of all meetings with the customer we will send a call for the meeting, specifying time, place, intention, agenda, and background documents. It is important to specify preparations that have to be made by both the customer and the group prior to the meeting.
\newline
\newline
A calling for a meeting should be sent before Monday at 14:00 the week the meeting took place.


\subsection{Minutes of a customer meeting}
The group should write a summary of each meeting with the customer. The most vital points are decisions, actions (what, who and deadline) and clarifications that are important for further work on the project. The customer should approve the minutes of meeting to make sure there are no misunderstandings of decisions made. The minutes of meetings are part of the "contract" with the customer, not uncommon in normal settings.
\newline
\newline
The minutes of meeting should be sent to the customer as soon as possible and always before Friday the week the meeting took place. If the minutes are not approved by the customer, they should be rewritten and resent for approval.

\subsection{Calling for the weekly advisor meeting}
Calling for meetings should be sent before Monday at 14:00 the week the meeting was scheduled. The meeting with the advisor will normally takes place every Tuesday at 11:15 if no cancellation or exceptions are made.

\subsection{Agenda for the weekly meeting with the advisor}
We were given a template from the course administrative that was to be followed. This template can be found in appendix \ref{se:adag}.

\subsection{Minutes of the weekly meeting with the advisor}
Minutes from the last meeting will be attached to the next meeting call and is a fixed subject on the agenda.


