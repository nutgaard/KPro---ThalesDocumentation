\section{Quality Assurance}

\subsection{Time of response}
Agenda and questions must be sent to customer between 24 and 48 hours before the customer meeting takes place. 
Minutes of meeting should be sent to customer within 48 hours after the meeting, but preferably as soon as possible.
Approval of minutes of customer meeting should be done by the rest of the group before sending, and within 48 hours of receiving it for the customer. Agenda and weekly documents should be sent to the advisor before mondays at 14:00.
We should get approval and feedback at the documents sent to the advisor at least within 72 hours of having sent him the documents.
\newline
\newline
See \ref{tab:responsetable} at page \pageref{tab:responsetable}.

\begin{table}
\begin{tabular}{l|l}
Approval of minutes of customer meeting & 48 hours \\
Feedback on weekly documents & 72 hours \\
Approval of weekly documents & 72 hours \\
Answer to a question & 48 hours \\
Producing requested documents & 48 hours after approving request
\end{tabular}
\caption{Time of response table}\label{tab:responsetable}
\end{table}

\subsection{Routines for producing high quality internally}
To secure high quality internally we made sure every item produced was reviewed by a team member who did not work on the given item. This should increase the quality of items and give the group members a better feel of the entire project and not just the parts they worked on themselves. 
\newline
\newline
Most of the working hours are done together with other group members, either in the entire team or in groups of two or three people. We have scheduled 18 hours to work together with the whole group each week, and thus there should always be someone to ask if we have any questions.

\subsection{Routines for approval of phase documents}
Making sure that the different documents throughout the phases had the quality we were looking for was a high priority for us. So we had a group member with the overall responsibility of putting together the different document parts and simultaneously reviewing the quality of the documents.
\newline
\newline 
We also made sure that the customer and advisor got a chance to view the important documents that would affect the direction of our project. In this way we would receive vital feedback and make the necessary changes according to the response of the advisor or customer. 
\newline
\newline
The visibility of our documents and progress is a part of the SCRUM method, meaning that the customer will have insight into what we are doing and how far we have progressed.   

\subsection{Calling for a meeting with the customer}
For all meetings with the customer we will send a call for the meeting, specifying time, place, intention (result), agenda, and background documents. It is important to specify preparations that has to be made by both the customer and the group prior to the meeting.
\newline
\newline
A call for a meeting should be sent before monday at 14:00 in the week the meeting takes place.
\newline
\newline
This information should be included in the meeting call:
\begin{itemize}
\item{}The date, time and place of the meeting.
\item{}The reason for the meeting.
\item{}The agenda we are going to go through.
\item{}All the documents that are vital for the meeting.
\item{}Preparations required by either by the group or the customer.
\end{itemize}

\subsection{Minutes of a customer meeting}
The group must write a summary of the meetings with the customer. It is vital that we write down decisions, actions (what, who, and deadline), clarifications etc. that are important for further work in the project. The customer must approve the minutes of the meeting, to make sure there are no misunderstanding of decisions made etc. The minutes of meetings are part of the “contract” with the customer. In a normal setting it is not uncommon that the minutes of meeting are part of the contract document with the customer.
\newline
\newline
In the project plan we will specify when the summary of the meeting should be done, when it is to be given to the customer for approval, how to distribute (e-mail, fax etc.) and expected time of response from the customer. It ought to be written the following day and should be distributed as soon as possible when you are done with internal approval of the minutes within the group. It is vital that we get the minutes approved as soon as possible to avoid misunderstandings.
\newline
\newline
The minutes of the weekly meeting with the customer should concede of:
\begin{itemize}
\item{}The name of the project.
\item{}Where the meeting takes place.
\item{}The time and date of the meeting.
\item{}The name of the meeting responsible.
\item{}The name of all attendees.
\item{}Topics of the meeting.
\item{}Decisions made.
\item{}Actions agreed upon.
\item{}Clarifications.
\item{}Time, date and place of next meeting.
\end{itemize}

The minutes of the weekly meeting should be sent to the customer as soon as possible or at least before friday the same week as the meeting took place. If the minutes are not approved they should be rewritten and sent again for approval. 

\subsection{Calling for the weekly advisor meeting between the group and its advisor}
A call for a meeting should be sent before monday at 14:00 in the week the meeting takes place. The meeting with the advisor will normally take place every tuesday at 11:00 if no cancellation has been made.
\newline
\newline
This information should be included in the meeting call:
\begin{itemize}
\item{}The name of the project.
\item{}Date and time of the meeting.
\item{}The agenda of the meeting.
\item{}A status report.
\item{}A table of reported working hours.
\item{}Minutes from the last meeting with the advisor.
\item{}Minutes from the last meeting with the customer.
\item{}Other documents vital for the meeting.
\end{itemize}

\subsection{Agenda for the weekly meeting with the advisor}
We are given a template from the course administrative that is to be followed.
\newline
\newline
\begin{enumerate}
\item{}Approval of agenda
\item{}Approval of minutes of meeting from last advisor meeting
\item{}Comments to the minutes from last customer meeting or other meetings
\item{}Approval of the status report, which may be structured as follows:
\begin{enumerate}
\item{}Summary
\item{}Work done in this period
\begin{enumerate}
\item{}Status of the documents that are being created
\item{}Meetings
\item{}Other activities
\end{enumerate}
\item{}Problems – what is interfering with the progress or taking resources? Problems are often risks that have taken effect.
\item{}Planning of work for the next period
\begin{enumerate}
\item{}Meetings
\item{}Activities
\end{enumerate}
\item{}Other
\end{enumerate}
\item{}Review/approval of attached phase documents
\item{}Other issues are listed here…
\item{}Other issues
\end{enumerate}

\subsection{Minutes of the weekly meeting with the advisor}
Minutes from the last meeting attached to the next calling for meeting and is a fixed subject on the agenda.
\newline
\newline
The minutes of the weekly meeting with the advisor should concede of:
\begin{itemize}
\item{}The name of the project.
\item{}Where the meeting takes place.
\item{}The time and date of the meeting.
\item{}The name of the meeting responsible.
\item{}The name of all attendees.
\item{}Topics of the meeting.
\item{}Decisions made.
\item{}Actions agreed upon.
\item{}Clarifications.
\item{}Time, date and place of next meeting.
\end{itemize}

The minutes of the weekly meeting should be sent to the customer as soon as possible or at least before friday the same week as the meeting took place. If the minutes are not approved they should be rewritten and sent again for approval. 


