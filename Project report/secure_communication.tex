\subsection{Secure communication}

\subsubsection{Drivers}
Communication channels are probably the easiest way to gain unauthorized access or internal information from the system since it does not require any physical interaction with the device. For example a packet sniffer would easily pick up on unprotected data sent through a wireless network. In an effort to mitigate this seccurity issue, network encryption is widely in use today, which is the topic of discussion in this section.

\subsubsection{Requirements}
In order to discuss and evaluate the different possible opportunities regarding end-to-end encryption of the communication channels it is needed to have some features to compare.

\subsubsection{Transparency}
In order to provide flexibility and modifiability one should prefer a transparent implementation of the security protocol from the application’s point of view. By using a protocol that is transparent to the application it allows the developers to focus on product development, and opens up for the possibility of changing the protocol at a later stage.

\subsubsection{Level of security vs. performance}
Different security protocols have different levels of security and performance. It is therefore important to recognize the level of security needed and the performance penalty associated with the chosen protocols. In general a higher level of security would always be used when performance is not an issue.

\subsubsection{Implementation complexity}
The implementation complexity of today's secure protocols is very high and not feasible during this project assignment. There exists, however, implementations of several secure protocols on the Android platform as it is. 

\subsubsection{Secure communication discussion}
There exist several methods which can be used to secure a communication channel, some more prominent than others. Secure communication could in theory be implemented at every level of the TCP/IP 5-layer reference model \cite{bib:cn}/"Internet model" \cite{bib:rfc1122}. The first option was to create our own secure protocol at the application layer that would provide a transparent wraparound of the transport layer, but this is rarely an optimal solution and will probably result in a poor and insecure imitation of an existing solution.
\newline
\newline
Second, we choose to implement the encryption at a lower level, as the network or link layer \cite{bib:techtarget}. For network layer encryption it would be possible to use \gls{ips}, this is however not possible through Java or Android \cite{bib:ispec} as it would require interfering with the operating system beneath.  The same can be said about link layer solutions; it would require interference with the operating system. But the link layer encryption relies on the security of each link host, something that cannot be guaranteed when sending over the Internet.
\newline
\newline
Third we have the option of using a pre-existing application/transport layer protocol like \gls{ssl1}/\gls{tls}. In order to maximize security gain it is recommended to use \gls{tls} 1.2 \cite{bib:ssl}  which has improved security relative to earlier version of \gls{tls} and \gls{ssl1}.

\subsubsection{Secure communication conclusion}
After this study we found that the only reasonable and feasible solution is to use \gls{ssl1}/\gls{tls}. This was mainly due to the transparency of the implementation and the lack of complexity since it is already implemented. This notion was only strengthened by the requirements from the customer. 

