\section{Secure communication}

\subsection{Drivers}
Communication channels are probably the easiest way to gain unauthorized access or internal information from the system since it does not require any physical interaction with the device.  For example a packet sniffer would easily pick up on unprotected data sent through a wifi net. In an effort to mitigate this seccurity issue, encryption is widely in use today, which is the topic of discussion in this section.

\subsection{Requirements}
In order to discuss and evaluate the different possible opportunities regarding end-to-end encryption of the communication channels it is needed to have some features to compare.

\subsubsection{Transparency}
In order to provide flexibility and modifiability one should prefer a transparent implementation of the security protocol from the application’s point of view. By using a protocol that is transparent to the application it allows the developers to focus on product development, and opens up for the possibility of changing the protocol at a later stage.

\subsubsection{Level of security vs. performance}
Different security protocols have different levels of security and performance. It is therefore important to recognize the level of security needed and the performance penalty associated with the chosen protocols. In general a higher level of security would always be used when performance is not an issue.

\subsubsection{Implementation complexity}
The implementation complexity of today's secure protocols is very high and not feasible during this project assignment. There exists, however, implementations of several secure protocols on the Android platform as it is. 

