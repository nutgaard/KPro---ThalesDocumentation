
\section{Black-box testing}
Black-box testing \cite{bib:blackbox} \cite{bib:bbfun} is a software testing method to test the functionality of the program. The tests focus on how the software is supposed to behave, but not how the software does it behind the scenes. The internal of the software is not known to the tester, hence the software can be called a black box. The tester provides input (touch, keystroke etc.) and compare the output to the expected result. Black-box testing is useful in all levels of software testing, from unit testing to acceptance testing, but it becomes increasingly more useful with higher levels. 
\newline
\newline
We have chosen to use two forms of functional testing, namely functional testing and usability testing. 

\subsection{Functional testing}\label{subsec:functionaltesting}

\paragraph{Method}\hfill
\newline
Functional testing \cite{bib:funtest} is a type of black-box testing that is based on test cases derived from the specification of the software. One tests by providing input, as touch or keyboard gestures, and compare the actual to the expected result. 

\paragraph{Template for functional testing}\hfill
\newline		
The template for functional testing can be found in table \ref{tab:casetemp1} on page \pageref{tab:casetemp1}.	

\begin{table}[h!]
\begin{center}
\begin{tabular}{l|p{10cm}}\hline
\textbf{Item} & \textbf{Description} \\ \hline \hline
Test case ID & id \\ \hline
Name & Shortname for the test\\ \hline
Requirement & Requirement reference\\ \hline
Description & Description of test\\ \hline
Preconditions & The conditions of the application before the tests are started\\ \hline
Flow of events & The list of steps necessary to complete the test \\ \hline
Expected results & The end result if everything works as expected\\ \hline 
Actual results & The end result\\ \hline
Comments & What went wrong, what could be the cause\\ \hline
Status &OK (no errors), OK- (minor issues) and Failed (not working)\\ \hline
\end{tabular}
\end{center}
\caption{Test case template} \label{tab:casetemp1}
\end{table}
\paragraph{The tests}\hfill
\newline

The test cases 1-25 are meant to verify that the application works correctly to normal use, whereas the test cases 26-28 are meant to that the application behaves well with incorrect input. The test cases can be found in appendix \ref{ch:funtest}. The specific test should be carried out whenever a feature is finished, and all the tests should be carried out at the end of the project.
			
\subsection{Usability testing}\label{subsec:usabilitytesting}
\paragraph{Method}\hfill
\newline
Usability testing is testing of the application on possible end-users. This is also a kind of black-box testing meant to observe how users use the application to find unknown errors and get an evaluation of the design and usability of the application. We created a test with six cases that mirrored the most important actions in the application. The test person tried to solve the tasks while the test leaders measured the time spent on each task and noted problems/comments/thoughts. After the test the persons filled out a SUS form.

\paragraph{SUS}\hfill
\newline
SUS is an abbreviation for System Usability Scale \cite{bib:sus} and is a simple Likert scale of ten questions where the test persons shall give a score from 1 to 5 where 1 is strongly disagree and 5 is strongly agree. The purpose is to obtain a global view of subjective evaluation of the usability of the software.
\newline
The SUS score is calculated in the following way:
\begin{enumerate}
\item{}Take the odd numbered questions and subtract 1 from these scores.
\item{}Take the even numbered questions and subtract the score from 5.
\item{}Add these scores together. 
\item{}Take this sum and multiply by 2.5.
\item{}Then you have a SUS score between 0 and 100.
\end{enumerate}

\paragraph{Usability goals}\hfill
\newline
Before conducting the usability tests, we have set some usability goals, to have something to assess the usability against.
\begin{enumerate}
\item{}The users should not spend more than 5 minutes on a task
\item{}The application should not crash during the usability tests
\item{}The users should not make more than 1 error during the tasks
\item{}The users should solve task 6 faster than task 1 (memorability)
\item{}The average SUS score should be above 70 \footnote{Studies have shown that 68 is average \cite{bib:susavg}}
\end{enumerate}

\paragraph{Templates}\hfill
\newline
SUS form can be found in appendix \ref{ch:usatest}.
\newline\newline
Observation form can be found in appendix \ref{ch:usatest}.

\paragraph{The tests}\hfill
\newline
We will have five people perform the tests. The reason behind this number is the theory of Jakob Nielsen \cite{bib:useit}, who states that a test with five persons should discover about 85 \% of the usability problems. This approach suggested that we should test and redesign three times, but due to time constraints we probably will not conduct more than one iteration.  
\newline

The tests can be found in table \ref{tab:usabilitytask1} - \ref{tab:usabilitytask6} on page \pageref{tab:usabilitytask1} - \pageref{tab:usabilitytask6}.

\begin{table}[h!]
\begin{center}
\begin{tabular}{>{\bfseries}l|l} \hline
Task no.&1\\ \hline
Task name&Log in\\ \hline
Description&Log in to the application with username and password\\ \hline
Input&Username="kprotesting" \\
&Password="kprotest"\\ \hline
\end{tabular}
\end{center}
\caption{Usability test - task 1} \label{tab:usabilitytask1}
\end{table}

\begin{table}[h!]
\begin{center}
\begin{tabular}{>{\bfseries}l|l}\hline
Task no.&2\\ \hline
Task name&Send message\\ \hline
Description&Send a regular message\\ \hline
Input&To="kprothales@gmail.com" \\
&Subject="Usability test"\\
&Security label="UNCLASSIFIED"\\ 
&Priority="Routine"\\
&Type="Operation"\\
&Message text=Enter your own text here\\ \hline
\end{tabular}
\end{center}
\caption{Usability test - task 2} \label{tab:usabilitytask2}
\end{table}

\begin{table}[h!]
\begin{center}
\begin{tabular}{>{\bfseries}l|l}\hline
Task no.&3\\ \hline
Task name&Settings\\ \hline
Description&Change the settings for predefined attributes for instant message\\ \hline
Input&Standard receiver="kprotesting@gmail.com" (yourself)\\
& Security label="RESTRICTED",\\
&Priority="Flash"\\
& Type="Drill"\\ \hline
\end{tabular}
\end{center}
\caption{Usability test - task 3} \label{tab:usabilitytask3}
\end{table}

\begin{table}[h!]
\begin{center}
\begin{tabular}{>{\bfseries}l|l}\hline
Task no.&4\\ \hline
Task name&Send instant message with attachments\\ \hline
Description&Send instant message with an image\\ \hline
Input &Image=Take one with the camera\\
&Message text=Enter you own text here\\ \hline
\end{tabular}
\end{center}
\caption{Usability test - task 4} \label{tab:usabilitytask4}
\end{table}

\begin{table}[h!]
\begin{center}
\begin{tabular}{>{\bfseries}l|l}\hline
Task no.&5\\ \hline
Task name&Read high-priority message with attachments\\ \hline
Description&Open a received high-priority message and view the attachments\\ \hline
Input&\\ \hline
\end{tabular}
\end{center}
\caption{Usability test - task 5} \label{tab:usabilitytask5}
\end{table}

\begin{table}[h!]
\begin{center}
\begin{tabular}{>{\bfseries}l|l}\hline
Task no.&6\\ \hline
Task name&Send message to a contact from the address book\\ \hline
Description&Send message to a contact by choosing from from the address book\\ \hline
Input&To="kprothales@gmail.com"\\
&Security label="UNCLASSIFIED"\\ \hline
\end{tabular}
\end{center}
\caption{Usability test - task 6} \label{tab:usabilitytask6}
\end{table}
