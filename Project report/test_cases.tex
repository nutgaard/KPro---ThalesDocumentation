
\section{Black-box testing}
		Black-box testing is a software testing method to test the functionality of the program. The tests focus on what the software is supposed to behave, but not how the software does it. The internal of the software is not known to the tester, hence it can be called a black box. The tester should provide input (touch, keystroke etc.) and compare the output to the expected result. Black-box testing is useful in all levels of software testing, from unit testing to acceptance testing, but it becomes increasingly more useful with higher levels. 
%Kilder: http://en.wikipedia.org/wiki/Black-box_testing (09.11)
%Kilder: http://softwaretestingfundamentals.com/black-box-testing/ (09.11)
	
		\subsection{Functional testing}
			\paragraph{Method}
				Functional testing is a type of black-box testing that is based on test cases derived from the specification of the software. One tests by providing input, as touch or keyboard gestures, and compare the actual to the expected result. 
%Kilder: http://en.wikipedia.org/wiki/Functional_testing (10.11)
			\paragraph{Template for functional testing}		
			The template for functional testing can be found in table \ref{tab:casetemp1} on page \pageref{tab:casetemp1}.	
				\begin{table}
					\begin{tabular}{l|p{10cm}}\hline
						Test case ID & id \\ \hline
						Name & Shortname for the test\\ \hline
						Requirement & Requirement reference\\ \hline
						Description & Description of test\\ \hline
						Preconditions & The conditions of the application before the tests are started\\ \hline
						Flow of events & The list of steps necessary to complete the test \\ \hline
						Expected results & The end result if everything works as expected\\ \hline 
						Actual results & The end result\\ \hline
						Comments & What went wrong, what could been the cause\\ \hline
						Status & \\ \hline
					\end{tabular}
				\caption{Test case template} \label{tab:casetemp1}
			\end{table}
			\paragraph{The tests}
				The test cases can be found in the APPENDIX REFERANSE HER.
				The test cases 1-25 are meant to verify that the app works correctly to normal use, whereas the test cases 26-28 are meant to that the app behaves well with incorrect input.
			\paragraph{Results}
				A summary of the test results can be found in table \ref{tab:caseresults} on page \pageref{tab:caseresults}.
				\begin{table}
					\begin{tabular}{l|l|l}\hline
						Test case ID &Test name&Result \\ \hline
						1&Login&OK\\
						2&Send regular message&OK\\
						3&Send message to contact from the address book&OK\\
						4&Send full message&OK\\
						5&Sent messages folder&OK\\
						6&Read and browse messages&OK\\
						7&Send attachments (camera)&OK\\
						8&Send attachments (gallery)&OK\\
						9&Send attachments (GPS)&OK\\
						10&Attachments received&OK\\
						11&Instant message settings&OK\\
						12&Message retrieval strategy settings&Failed\\
						13&Security labels settings&Failed\\
						14&Send instant message&OK\\
						15&Receive flash/override message&OK\\
						16&Send instant message with attachments&GPS failed, images OK\\
						17&Receive instant message outside the app&OK\\
						18&Widget for instant message&OK\\
						19&Reply&OK-\\
						20&Forward&OK-\\
						21&Delete&Failed\\
						22&Delivery report and receipt notification&Failed\\
						23&Status of delivery report and receipt notification&Failed\\
						24&Sort messages&OK\\
						25&Search in inbox&Failed\\	
						26&Login incorrect input&OK\\
						27&Receiver incorrect input&OK\\
						28&Security label incorrect input&OK\\ \hline
					\end{tabular}
				\caption{Functional test result summary} \label{tab:caseresults}
			\end{table}
			\paragraph{Discussion}
			The results from the functional testing showed that 19 out of the 28 test cases passed the test,  six failed and three almost passed. Out of the six that failed, three were due to not having time to start with the implementation and three were known bugs that we did not get the time to fix. The three that almost passed were issues that we know how to fix but had not seen until the test.
	\subsection{Usability testing}
		\paragraph{Method}
			Usability testing is testing of the app on possible end-users. This is also a kind of black-box testing meant to observe how users use the app to find unknown errors and get an evaluation of the design and usability of the app. We created a test with 6 cases that mirrored the most important actions in the app. The test persons tried to solve the tasks while the test leaders measured the time spent on each task and noted problems/comments/thoughts. After the test the persons filled out a SUS form.
		\paragraph{Template}
			SUS form REFERANSE TIL APPENDIX
			\newline\newline
			Observation form REFERANSE TIL APPENDIX
		\paragraph{The tests}
			Testene bør inn her eller?
		\paragraph{Results}
			\subparagraph{SUS scores}
			REFERANSE TIL APPENDIX 
			The average SUS score was X. 
			\newline\newline
			\subparagraph{Time spent}
			REFERANSE TIL APPENDIX
			\subparagraph{Observation forms}
			REFERANSE TIL APPENDIX
		\paragraph{Summary}
			There were many problems that reoccured in the test results. Below is a short summary of what were the problems:
			\begin{itemize}
				\item{}The testers were annoyed that the application does not remember login information
				\item{}Some were confused by odd choices of words
				\item{}The testers were annoyed that tilting of the phone sets you back to the inbox
				\item{}Almost all complained that the instant message and settings tab icons were not intuitive and spent some time finding the right tab.
				\item{}Some complained about small fonts and small toasts.
				\item{}Many were annoyed that the text boxes behaved weird, e.g. that the letters were not capitalized after a period and the keyboard did not always close. 
				\item{}The settings were unstable and the app stopped during the task that tested the settings.
				\item{}Many of the testers complained about the lack of confirmation after actions in the app.
			\end{itemize}
		\paragraph{Discussion}
			Most of the feedback we got in the usability tests were issues that we had not thought about. Even though we did not have the time to make improvements of the user interface and functionality, we learned a lot about how do better do things in later projects.