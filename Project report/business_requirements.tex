

\section{Business requirements}

\subsection{Quality requirements}

\subsubsection{Functional requirements}
See table \ref{tab:functionalreq} at page \pageref{tab:functionalreq}.

\begin{table}
\begin{tabularx}{\linewidth}{>{\setlength\hsize{.2\hsize}}X|>{\setlength\hsize{1.8\hsize}}X|>{\setlength\hsize{0.3\hsize}}X}
\textbf{Req.} & \textbf{Description} & \textbf{Priority} \\ \hline \hline
\textbf{FR1} & \textbf{Starting application and logging in:} the user has to be able to start the apllication and authorize himself against an authorizing mechanism & High \\ \hline
\textbf{FR2} & \textbf{Send a message to another user:} the user has to be able to send a simple message via regular mail protocols to a recipient of own choice & High \\ \hline
\textbf{FR3} & \textbf{Browse received messages:} the user has to be able to browse all messages he has recieved & High \\ \hline
\textbf{FR4} & \textbf{Browse sent messages:} the user has to able to browse all messages he has sent, and seeing the status of it, e.g "unread", "read", "action taken" & High \\ \hline
\textbf{FR5} & \textbf{Viewing address book:} the user has to be able to view the address book with all contacts, so that he is able to choose a recipient from a list when he want to send a message & High \\ \hline
\textbf{FR6} & \textbf{Sending and receiving message with attachments:} the user has to be able to add an attachment to the message, so that the recipient gets the attachment as well as the message. By opening the message, the attachment will also show & Medium \\ \hline
\textbf{FR7} & \textbf{Answer, delete and forward messages:} the user can, by clicking on a message, choose if he want to answer, delete or forward the message, and be brought to the correct screen for doing selected action & Medium \\ \hline
\textbf{FR8} & \textbf{Marking messages with a grade of precedence:} ?? & Medium \\ \hline
\textbf{FR9} & \textbf{Settings menu:} the user can choose to alter the following settings ??? via the settings menu & Low
\end{tabularx}
\caption{Functional requirements} \label{tab:functionalreq}
\end{table}

\subsubsection{Quality requirements}

\paragraph{Usability}
\subparagraph{U1 Ease of use}\hfill
\newline
The user should be able to use the program without a lot mistakes, e.g. pressing the wrong button  because button labels are ambiguous or too small.
\newline
\newline
See table \ref{tab:easeofuse} at page \pageref{tab:easeofuse}.
\begin{table}
\begin{tabularx}{\linewidth}{>{\setlength\hsize{.3\hsize}}X|>{\setlength\hsize{0.7\hsize}}X}
\textbf{Portion of scenario} & \textbf{Values} \\ \hline \hline
Source & End user \\ \hline
Stimulus & Use system without problems \\ \hline
Artifact & System \\ \hline
Environment & At run time \\ \hline
Response & Provide clean and easy to understand interface \\ \hline
Response measure & Only 5\% of user operations were mistakes, and not what the user wanted to do
\end{tabularx}
\caption{Ease of use} \label{tab:easeofuse}
\end{table}

\subparagraph{U2 Efficent use}\hfill
\newline
The user should be able to use the main features of the program with as few clicks as possible. Streamlined design is therefore of great importance.
\newline
\newline
See table \ref{tab:efficentuse} at page \pageref{tab:efficentuse}.
\begin{table}
\begin{tabularx}{\linewidth}{>{\setlength\hsize{.6\hsize}}X|>{\setlength\hsize{1.4\hsize}}X}
\textbf{Portion of scenario} & \textbf{Values} \\ \hline \hline
Source & End user \\ \hline
Stimulus & Use system efficiently \\ \hline
Artifact & System \\ \hline
Environment & At run time \\ \hline
Response & Organize application so that important functions are easy to reach from main menu \\ \hline
Response measure & The user should never use more than 1 minute finding what he looks for
\end{tabularx}
\caption{Efficent use} \label{tab:efficentuse}
\end{table}

\newpage
\paragraph{Performance}
\hfill
\newline
See table \ref{tab:performance} at page \pageref{tab:performance}.
\begin{table}
\begin{tabularx}{\linewidth}{>{\setlength\hsize{.6\hsize}}X|>{\setlength\hsize{1.4\hsize}}X}
\textbf{Portion of scenario} & \textbf{Values} \\ \hline \hline
Source & End user \\ \hline
Stimulus & Wants to do normal GUI operations \\ \hline
Artifact & System \\ \hline
Environment & Under normal operations \\ \hline
Response & The operation is performed \\ \hline
Response measure & With a latency of maximum 3 seconds
\end{tabularx}
\caption{P1 Latency} \label{tab:performance}
\end{table}

\paragraph{Security}
\hfill
\newline
See table \ref{tab:s1}, \ref{tab:s2} and \ref{tab:s3} at pages \pageref{tab:s1}, \pageref{tab:s2} and \pageref{tab:s3}.
\begin{table}
\begin{tabularx}{\linewidth}{>{\setlength\hsize{.6\hsize}}X|>{\setlength\hsize{1.4\hsize}}X}
\textbf{Portion of scenario} & \textbf{Values} \\ \hline \hline
Source & Unauthorized user or app \\ \hline
Stimulus & Wants to access data saved by our app \\ \hline
Artifact & System \\ \hline
Environment & Under normal operations \\ \hline
Response & User is blocked from accessing data \\ \hline
Response measure & No data exposed to the user or app
\end{tabularx}
\caption{S1 Accessing locally stored data outside of app} \label{tab:s1}
\end{table}


\begin{table}
\begin{tabularx}{\linewidth}{>{\setlength\hsize{.6\hsize}}X|>{\setlength\hsize{1.4\hsize}}X}
\textbf{Portion of scenario} & \textbf{Values} \\ \hline \hline
Source & Unauthorized user \\ \hline
Stimulus & Wants to use app features \\ \hline
Artifact & System \\ \hline
Environment & Under normal operations \\ \hline
Response & User is blocked from using functions \\ \hline
Response measure & No features exposed to the user - stopped by login screen
\end{tabularx}
\caption{S2 Trying to use app with wrong privileges} \label{tab:s2}
\end{table}

\begin{table}
\begin{tabularx}{\linewidth}{>{\setlength\hsize{.6\hsize}}X|>{\setlength\hsize{1.4\hsize}}X}
\textbf{Portion of scenario} & \textbf{Values} \\ \hline \hline
Source & Unauthorized user or app \\ \hline
Stimulus & Wants to access data stream \\ \hline
Artifact & System \\ \hline
Environment & Under normal operations \\ \hline
Response & Is unable to get useful information because of SSL encryption \\ \hline
Response measure & No useful data exposed to the user
\end{tabularx}
\caption{S3 Trying to access the apps external data traffic} \label{tab:s3}
\end{table}

\subsubsection{Component and Technology related constraints}
Vanilla Android is the platform of choice, compatible down to Android API 8

\paragraph{Programming language} \hfill
\newline
Developing for Android will in this project mean code will be written in Java for the Dalvik Virtual Machine. This means there is no support for multiple inheritance, and limits design choices to some degree.

\paragraph{Mobile devices} \hfill
\newline
Android is a mobile OS, so the app will run on phones, as well as tablets.

\paragraph{Input methods} \hfill
\newline
Android phones have small screens and usually no keyboard. This means the user interface must be designed to work with small touch screens as the only input method.

\paragraph{Memory}\hfill
\newline
Phones have limited memory. Typically 256-512 MB. The app will probably never be close to breaching this limit, as the most memory consuming components are a couple of textures, all smaller than 1280x800 pixels.

\paragraph{Screen resolution/aspect ratio} \hfill
\newline
As the app will support different Android devices, it also has to support different resolutions as well as aspect ratios. The graphics must be programmed to scale to different formats without degrading quality or proportions.

\paragraph{Bandwidth} \hfill
\newline
As the app probably will be downloaded from an app store over a wireless Internet connection, the package size have to be acceptable. (5-15 MB?)
\newline
Since the program is network based, it is important to pay attention to bandwidth usage. By compressing large data files before they are sent, we will reduce this to a minimum.

\paragraph{Commercial off-the-shelf (COTS)} \hfill
\newline
As this project has sec


