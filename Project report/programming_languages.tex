\newpage

\section{Programming languages}
We have decided to write the program in Java, and used \gls{xml} for creating parts of the supporting structure of our project. A summary of these choices are presented at the end of this section.

\subsection{Java}

Java is a programming language originally developed by James Gosling at Sun Microsystems \cite{bib:java}. It is a class based, object oriented language that incorporates concurrency and it is a general purpose language. The main reason Java was developed, was to create a programming language which had a simpler object model and provided fewer low-level facilities than C or C++. Even though, much of its syntax is derived from these languages.
\newline
\newline
Writing in Java, means that we do not have to recompile our code for every kind of Android device we want the application to run on. Our program will run on top of the Dalvik Virtual Machine, which encapsulates the Android application. This makes programming a lot easier, as we do not have to worry about where our program runs - the Dalvik \gls{vm} takes care of this.
\newline
\newline
One of the major benefits from writing a program in Java, is that it is compiled into bytecode that can run on any Java Virtual Macine. This means, that the program can run on any architecture that Java can run on.

\subsection{Native code}
It is possible to implement parts of the application using native-code languages such as C and C++ by using the Java Native Development Kit (\gls{jndk}) \cite{bib:andk}. This may be beneficial if there are good libraries written in these languages, or if we need the extra performance you get by not going through the Dalvik VM. We have chosen not to do this as as it is stated on the Android developer page for \gls{ndk}: “... you should understand that the \gls{ndk} will not benefit most apps.”

\subsection{Scripting}
By using the Scripting Layer for Android (\gls{sl}), it is possible to edit and execute scripts and interactive interpreters directly on the Android device \cite{bib:slfa}. By using the APIs available to the full-fledged Android application, it is indeed possible to run software using scripting. Currently available programming languages are Locale, Python, Pers, JRuby, Lua, BeanShell, JavaScript, Tcl and shell. We are not using this method, as stated by Google: “SL4A is designed for developers and is alpha quality software.” We will probably be running in to a lot of problems that is not well enough documented.

\subsection{PHP}
Programming can be done in PHP by installing the PHP for Android application and PHP library. This includes a lot of scripting to do common Android tasks, but it is certainly feasible. Unfortunately, documentation is a bit tricky to find, and the user community is small \cite{bib:php}.

\subsection{C\#}
It is possible to write Android software using C\#. By installing MonoDroid created by Xamarin \cite{bib:mbx}, you can develop C\# software for Android. The advantages of using C\# may be many, depending on the developer. If you are familiar with Visual Studio and the .NET framework, you will certainly feel at home here. The framework is well documented, so there should not be any problem developing XOXOmail using MonoDroid.
\newline
\newline
The drawbacks of programming with MonoDroid are a few. The first drawback is the pricing, which of current date (10. september 2012) is 2305 NOK for a professional licence. While MonoDroid does support a vast majority of the .NET framework, there are still some features missing, so one should be aware of this before deciding on using MonoDroid.

\subsection{Programming languages conclusion}
As we have seen, there are many ways to create software for Android. One can rely on only one language for programming, or use a combination of languages. We have chosen to develop purely on a Java platform. Why is this relevant for us, as we are creating an Android application?
\newline
\newline
First, it is the most used developing language for Android, which means we are using a well tested and documented development framework. It is unlikely that we will run into big problems. Second, we are all familiar with using Java, so the learning curve will only be a bit steep for those in the group that are new at Android development.  Another benefit of using Java is that, if we decide on implementing a server part, we can create the server first and then decide if we want it to run on the phone or on a computer, be that a Linux or Windows computer or something else. This flexibility is something we only can get from Java.
\newline
\newline
We chose Java as our main programming language as this is the most widespread language and has most support online. It is the language which has been pushed forward the most from Google, and is the recommended language to use when developing Android applications.





