\subsection{Tool selection}
This section will describe the tools we chosed to use during the project.
\subsubsection{Git \& GitHub}
\begin{wrapfigure}{r}{0.2\textwidth}
  \vspace{-40pt}
  \begin{center}
    \includegraphics[width=60px,height=80px]{GitHub}
  \end{center}

\end{wrapfigure}
\gls{git} is an extremely fast, efficient and distributed version control system ideal for the collaborative development of software. It can be used to manage all your public and private repositories, and makes it easy to share and work with the same source code and documents \cite{bib:git}. \gls{github} is a hosting system that implements git and is free to use if you choose to share your source code with others. 

\subsubsection{NetBeans IDE}
\begin{wrapfigure}{r}{0.2\textwidth}
  \vspace{-65pt}
  \begin{center}
    \includegraphics[width=0.2\textwidth]{NetBeans}
  \end{center}

\end{wrapfigure}
NetBeans \gls{ide} is an integrated development environment for developing with Java. It is written in Java and runs on Windows, Linux and \gls{osx}. NetBeans \gls{ide} is one of the most popular \gls{ide}s. Using a common \gls{ide} helps speed up the development process.   

\subsubsection{Google Docs}
\begin{wrapfigure}{r}{0.2\textwidth}
  \vspace{-115pt}
  \begin{center}
    \includegraphics[width=0.2\textwidth]{GoogleDocs}
  \end{center}

\end{wrapfigure}
Google Docs is a free web based service that offers users the ability to create and edit documents online. Google Docs lets us share documents easily  and \newline collborate on the same document simultaneously.	

\subsubsection{Apache Maven}
\begin{wrapfigure}{r}{0.2\textwidth}
  \vspace{-35pt}
  \begin{center}
    \includegraphics[width=0.2\textwidth, height=50px]{Maven}
  \end{center}

\end{wrapfigure}
\gls{maven} is a build automation tool which is typically used with Java projects. It uses an \gls{xml} file to describe the project, the project dependencies on other external modules and components, the build order, directories, and required plug-ins.   

\subsubsection{JIRA with GreenHopper extension}
\begin{wrapfigure}{r}{0.2\textwidth}
  \vspace{-40pt}
  \begin{center}
    \includegraphics[width=80px,height=30px]{Jira}
  \end{center}

\end{wrapfigure}
JIRA is an issue tracking system developed by Atlassian that is used for bug tracking, issue tracking and project management. When used along with the GreenHopper extension it adds support for agile development.\cite{bib:atlassian}

\subsubsection{Wireshark}
\begin{wrapfigure}{r}{0.2\textwidth}
  \vspace{0pt}
  \begin{center}
  \vspace{-30pt}
    \includegraphics[width=0.2\textwidth]{Wireshark}
  \end{center}

\end{wrapfigure}
Wireshark is a tool used for network troubleshooting and analysis. It captures network traffic and displays it to you through a graphical user interface. Wireshark is great for understanding how your application acts on a lower network layer.

\newpage

\subsubsection{LaTeX}
\begin{wrapfigure}{r}{0.2\textwidth}
  \vspace{-35pt}
  \begin{center}
    \includegraphics[width=0.2\textwidth, height=50px]{Latex}
  \end{center}

\end{wrapfigure}
LaTeX is a free high-quality typesetting system designed for writing technical and scientific documentation. It is most often used for medium-to-large documents, but can be used for almost any form of publishing. 
\newline
\newline
LaTeX is based on the idea that it is better to leave document design to document designers and to let authors get on with writing documents. So it encourages authors not to worry too much about the appearance of their documents, but to concentrate on getting the content right \cite{bib:latex}.

\subsubsection{Fluid UI}

\begin{wrapfigure}{r}{0.2\textwidth}
  \vspace{-35pt}
  \begin{center}
    \includegraphics[width=0.2\textwidth, height=50px]{fluidui}
  \end{center}

\end{wrapfigure}

Fluid UI \cite{bib:fui} is an online mobile app prototyping tool for Android and iOS application. Fluid UI was chosen because it gave us a quick and easy way of creating interactive prototypes to share with the customer.

\subsubsection{Violet}

\begin{wrapfigure}{r}{0.2\textwidth}
  \vspace{-125pt}
  \begin{center}
    \includegraphics[width=0.2\textwidth, height=50px]{violetuml}
  \end{center}

\end{wrapfigure}
Violet is a UML editor which is good at drawing diagrams. It is \newline
basically intended for developers and students that needs to produce UML \newline
diagrams \cite{bib:violet}. \gls{uml} is a standarized modeling language for developing\newline
of a software program \cite{bib:lmu}. We have used this program to make our use cases.

\subsubsection{Software Ideas Modeler}

\begin{wrapfigure}{r}{0.2\textwidth}
  \vspace{-35pt}
  \begin{center}
    \includegraphics[width=0.2\textwidth, height=50px]{ideasmodeler}
  \end{center}

\end{wrapfigure}

Software Ides Modeler is also a \gls{uml} tool. But it is a more advanced tool than Violet, and it supports all the different diagram types specified in \gls{uml}. This tool does even support ERD diagrams, BPMN, CRC, flowcharts and data flow diagrams \cite{bib:sim}.

\subsection{Organizational demands}
There are no organizational demands from Thales, but we have agreed on a scrum based approach. We decided that scrum was a good lifecycle-model to use since it could give us a runnable product early that could be revised during the project. This is, after all, a pilot project and the specifications are not fully completed, so many revisions are anticipated.

\subsection{Resources}
The group consists of six members with different personalities and interests. We are all students working on a master's degree in computer science, but the programming experience of each member varies. Two of the group members have programmed since they were about 13 years old, while the rest started when they came to NTNU. All of us have at some point had a programming job, and hence have some experience with working in teams.

\newpage

Fortunately we all have different interests regarding what we want to contribute with in the project. We have all agreed that everybody has to participate in both planning, implementation and report work, but we have also tried to make use of our special interests. As some of us enjoy report work and others have had much experience in the setup of programming related tasks, we have allocated the main roles accordingly.
\newline
\newline
Having a group consisting of people with different personalities could always be a challenge with regards to the group dynamics. By conducting a daily standup where everyone shares what they have done, what they will do the current day and what they need to get it done, we plan to avoid communication problems and misunderstandings. We want to share a common understanding of what needsto be done and get everyone in sync. 

\subsection{Schedule of results}
We have chosen to use scrum because of the flexibility it offers. It is a lightweight solution with very little extra overhead for us to worry about. This will allow us to focus on the actual work rather than having to waste our limited resources on superfluous coordination.

\subsubsection{Milestones}
See table \ref{tab:milestones} below for an overview of the milestones in our project.
\begin{table}[h!]
\begin{center}
\begin{tabular}{l|l} \hline
\textbf{Milestone} & \textbf{Date} \\ \hline \hline
Project start &  21.08.2012\\ 
Pre-delivery of project report & 14.10.2012\\ 
Final delivery of project report & 22.11.2012\\
Project presentation and demonstration & 22.11.2012\\ \hline
\end{tabular}
\end{center}
\caption{Milestones} \label{tab:milestones}
\end{table}


\subsubsection{Sprints}
See table \ref{tab:sprints} below for an overview of the sprints that this project consisted of.
\begin{table}[h!]
\begin{center}
\begin{tabular}{l|l} \hline
\textbf{Sprint} & \textbf{Duration} \\ \hline \hline
Sprint 1 &  27.08.2012 - 16.09.2012\\
Sprint 2 & 17.09.2012 - 07.10.2012\\
Sprint 3 & 08.10.2012 - 28.10.2012\\
Sprint 4 & 29.10.2012 - 18.11.2012\\ \hline
\end{tabular}
\end{center}
\caption{Sprints} \label{tab:sprints}
\end{table}

