\section{Parsers libraries \& tools}

\subsection{XStream}
XStream is a simple Java library to serialize objects to \gls{xml} and back. Using XStream, you can serialize most Java objects without any mapping. Object names become element names in the \gls{xml} produced, and the strings within classes form the element content of the \gls{xml}.
\newline
\newline
The classes that you serialize with XStream don't need to implement the Serializable interface, thanks to XStream handling all serializations. XStream is a serialization tool and not a data binding tool, which means that it doesn't perform class generation from an XML Schema Definition (\gls{xsd}) file \cite{bib:xstream} \cite{bib:ibm}.
\newline
\newline
Another feature of XStream is that it has the capability to serialize to and from \gls{json} as well.

\textbf{Features:}
\begin{itemize}
\item{}Ease of use: A high level facade is supplied that simplifies common use cases.
\item{}No mappings required: Most objects can be serialized without a need for specifying mappings.
\item{}Performance: Speed and low memory footprint are a crucial part of the design, making it suitable for large object graphs or systems with high message throughput.
\item{}Clean \gls{xml}: No information is duplicated that can be obtained via reflection. This results in \gls{xml} that is easier to read for humans and more compact than native Java serialization.
\item{}Requires no modifications to objects: Serializes internal fields, including private and final. Supports non-public and inner classes. Classes are not required to have default constructor.
\item{}Full object graph support: Duplicate references encountered in the object-model will be maintained. Supports circular references.
\item{}Integrates with other \gls{xml} \gls{api}s: By implementing an interface, XStream can serialize directly to/from any tree structure (not just \gls{xml}).
\item{}Customizable conversion strategies: Strategies can be registered allowing customization of how particular types are represented as \gls{xml}.
\item{}Error messages: When an exception occurs due to malformed \gls{xml}, detailed diagnostics are provided to help isolate and fix the problem.
\item{}Alternative output format: The modular design allows other output formats. XStream ships currently with \gls{json} support and morphing.
\end{itemize}

\textbf{Limitations:}
\begin{itemize}
\item{}If using the enhanced mode, XStream can re-instantiate classes that do not have a default constructor. However, if using a different \gls{jvm} like an old JRockit version, a \gls{jdk} 1.4 or you have restrictions because of a SecurityManager, a default constructor is required.
\item{}The enhanced mode is also necessary to restore final fields for any \gls{jdk} < 1.5. This implies deserialization of instances of an inner class.
\item{}Auto-detection of annotations may cause race conditions. Preprocessing annotations is safe though.
\end{itemize}