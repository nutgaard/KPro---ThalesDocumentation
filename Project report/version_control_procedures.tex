

\subsection{Version control procedures}
Our group has created a systematic procedure for version control for all textual  and code documents. 

\subsubsection{Version control for code}
For source code management there are several viable options on the market, the big ones being \gls{git} and \gls{svn}. Although they do pretty much the same thing, there are some key differences in how they work which makes \gls{git} favorable over \gls{svn}. One key feature is performance, where \gls{git} greatly outperforms \gls{svn}. \gls{git} gives you the possibility of a hierarchy of repositories, which offers great flexibility. One upside being the local repository every user has on their computer. This enables the developer to work even though he does not have access to the remote repository. Lastly, and perhaps the most important difference comes with the branch/merge differences between the two tools. \gls{git} does this incredible well, while the general consensus is that \gls{svn}’s merge tool is tedious to use.  
\newline
\newline
In addition to technical differences between \gls{git} and \gls{svn}, it was the experiences of the developer team that finally made the decision to use \gls{git} for this project. To learn how to use \gls{git} check out the Git Book at \cite{bib:gitbook}.

\subsubsection{Version control procedure}
All procedures assumes the use of a terminal similar to Bash, and that the current directory is the workspace of your \gls{ide}.
\newline
\newline
[]: optional arguments \& <>: argument

\paragraph{Setup}
\begin{description}
\item[Description:] Setup of project (usually just once)
\item[Command:] git clone <git remote repository address>
\end{description}

\paragraph{On programming session start}
\begin{description}
\item[Description:] Whenever a developer starts programming
\item[Command:] git pull [<remote repository> <branch>]
\end{description}

\paragraph{On change}
\begin{description}
\item[Description:] Whenever a developer changes some small functionality, feature or structure.
\item[Command:] git add <file> \#Repeat for all changed files \\
\hspace*{3em} git commit -m "<commit message describing the changes made>"\\
\hspace*{3em} [git pull \&\& git push] \#should only be done for if project builds
\end{description}

\paragraph{On complete}
\begin{description}
\item[Description:]  Whenever a developer finishes with some functionality, feature,  structure or changes 
\hspace*{4em}which influence any of the above.
\item[Command:]  git pull [<remote repository> <branch>] \\ 
\hspace*{3em} \#Fix any merge conflicts \\
\hspace*{3.5em}git push [<remote repository> <branch>]
\end{description}

\subsubsection{Version control for report documents}
Version control for documents is divided into two bulks. In the development stage of a document it resides within Google Drive (Google Docs). This gives us a rapid way of writing, editing and sharing all documents related to the project. Google Drive also provides the necessary version control needed at this point. When a document or part of a document is finished it is then typesett in LaTeX and pushed to \gls{git}. Once on \gls{git}, the documents work in the same way as the code.  


