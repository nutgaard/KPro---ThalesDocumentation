\subsection{Templates and standards}
The group have made templates and standards for the most relevant document types. Even though it took some time to create these in the beginning, we believe it will benefit us over time.

\subsubsection{Templates}
We have made templates for agendas and minutes of meetings, as well as for the weekly documents. These are located in appendix \ref{ch:temp}. The template will hopefully make the work for our agenda and minutes responsible a lot easier.

\subsubsection{Standards}

\paragraph{Organizing files}
\subparagraph{Google Docs}\hfill
\newline
We will use Google Docs to organize our files so that our entire team can view and collaborate on the documents simultaneously. We have one folder that contains our entire project and in that folder we have subfolders to make the best overview possible. We will try not to have too many files in each folder so that finding a specific document becomes as easy as possible.

\subparagraph{Git}\hfill 
\newline
We will also use Git so that it is easy to share all the code and documents as well as to keep a version control of the files. When using \gls{git} all the team members will have access to change, read and use all the code and document files whenever they feel like it.

\newpage

\subparagraph{Saving files on gitHub}\hfill
\newline
All documents and Java files must compile before they are pushed to the repository.
\newline
\newline
\textbf{Code repository structure}
\begin{verbatim}
kpro-app/
  res/ -- application specific resources -- resources used in the application
    drawable/ -- images for the application
    layout/ -- layouts for the the activies
    values/ -- common values for the application
  src/ -- java source folder
    main/java/no/ntnu/kpro/app -- the differente activites
  target/ -- build folder
kpro-instrumentation/
  src/ -- java source folder
    main/java/no/ntnu/kpro
	    app/ -- gui tests folder
		  core/service/ -- service tests folder
  target/ -- build folder
kpro-lib/
  res/ -- application specific resources
  src/ -- java source folder
    main/
      java/no/ntnu/kpro/core
        model/ -- application specific models
        service/ -- service related package
          factories/ -- factory classes
          implementations/  -- implementating classes
          interfaces/ -- all relevant interfaces
        utilities -- common javabased utilities
    test/
      java/no/ntnu/kpro/core/
        service/
          implementation/
  target/ -- build folder
\end{verbatim}

\newpage

\textbf{LaTeX repository structure}
\begin{verbatim}
Backlog/		-- contains the teams backlog
Meetings/		-- agendas, feedback and minutes from every meeting
  Internal/
    Agendas/
    Feedback/
    Minutes/
  Mohsen Anvaari/
    Agendas/
    Feedback/
    Minutes/
  Thales/
    Agendas/
    Feedback/
    Minutes/
Project report/		-- the tentative complete report
Templates and standards/ -- templates in use
Weekly status reports/ -- weekly timesheets

\end{verbatim}

\paragraph{Naming of files}\hfill
\newline
All the files in our project shall be given a name that starts with the date they are created so that they will be easier to find. The name should also contain one or two words that explains what the documents contain.

\subparagraph{Agendas}\hfill
\newline
"YYYY-MM-DD-Agenda-[Customer/Advisor/Internal]"

\subparagraph{Minutes of meeting}\hfill
\newline
"YYYY-MM-DD-Minutes of Meeting-[Customer/Advisor/Internal]"

\newpage

\paragraph{Coding style}\hfill
\newline
See table \ref{tab:namingconventions} below for an overview of the naming conventions for the UI elements.
\begin{table}[h!]
\begin{center}
\begin{tabular}{l|l|l} \hline
\textbf{Element} & \textbf{Convention} & \textbf{Example} \\ \hline \hline
Button & btn & btnSubject \\
ImageButtonr & imb & imbSubject \\
TextView & lbl & txtSubject \\
ImageView & imv & imvSubject \\
EditText & txt & txtSubject \\
Spinner & spr & sprSubject \\
ListView & lst & lstSubject \\
CheckBox & chb & chbSubject \\
RadioButton & rbt & rbtSubject \\
ToggleButton & tbt & tbtSubject \\
Layout & lay & laySubject \\ \hline 
\end{tabular}
\end{center}
\caption{Naming conventions - UI elements}\label{tab:namingconventions}
\end{table}

