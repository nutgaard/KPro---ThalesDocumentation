\section{Templates and standards}
The group has decided to make templates and standards for the most relevant document types. Even though it will take some time to create these in the beginning, we believe to benefit on it over time.

\subsection{Templates}

\subsubsection{Phase documents}

\paragraph{Sprint template}\hfill
\begin{enumerate}
\item{}Sprint Planning
\item{}Sprint Duration
\item{}Sprint Goal
\item{}Sprint Backlog
\item{}System Design
\item{}Customer Feedback
\item{}Conclusion
\end{enumerate}

\subsubsection{Agenda for meetings}
We have made an template of an agenda that we use when we are planning an advisor meeting. The template make the work for our agenda responsible a lot easier.

\paragraph{Agenda for advisor meetings}\hfill
\newline
\begin{enumerate}
\item{}Approval of agenda 
\item{}Approval of minutes of meeting from last advisor meeting
\item{}Comments to the minutes from last customer meeting or other meetings
\item{}Approval of the status report, which may be structured as follows:
\begin{enumerate}
\item{} Summary
\item{} Work done in this period
\begin{enumerate}
\item{}Status of the documents that are being created
\item{}Meetings
\item{}Other activities
\end{enumerate}
\item{}Problems – what is interfering with the progress or taking resources? Problems are often risks that have taken effect.
\item{}Planning of work for the next period
\begin{enumerate}
\item{}Meetings
\item{}Activities
\end{enumerate}
\item{}Other
\end{enumerate}
\item{}Review/approval of attached phase documents
\item{}Other issues
\end{enumerate}

\paragraph{Agenda for customer meetings} \hfill
\newline
\begin{itemize}
\item{}Date
\item{}Project name
\item{}Calling by
\item{}Time and date 
\item{}Place 
\item{}Attendees 
\item{}Referent
\end{itemize}

\begin{enumerate}
\item{}Approval of agenda
\item{}Approval of minutes of meeting from last customer meeting
\item{}Case name
\item{}Other issues
\item{}Next meeting
\end{enumerate}

\paragraph{Agenda for internal meetings} \hfill
\newline
\begin{itemize}
\item{}Date
\item{}Project name
\item{}Calling by
\item{}Time and date 
\item{}Place 
\item{}Attendees 
\item{}Referent
\end{itemize}

\begin{enumerate}
\item{}Approval of agenda
\item{}Approval of minutes of meeting from last internal, customer and advisor meeting
\item{}Case name * X
\item{}Other issues
\item{}Next meeting
\end{enumerate}

\subsubsection{Weekly status reports for the advisor meetings}
\begin{itemize}
\item{}Date
\item{}Project name
\item{}Calling by
\item{}Time and date 
\item{}Place 
\item{}Attendees 
\item{}Referent
\end{itemize}

\begin{enumerate}
\item{}Activities done this week
\item{}Activities planned next week
\item{}Challenges ahead
\item{}Status summary
\item{}Milestones
\end{enumerate}

\subsubsection{Minutes for meetings}

\paragraph{Minutes for advisor meetings} \hfill
\newline
\begin{itemize}
\item{}Date
\item{}Project name
\item{}Calling by
\item{}Time and date 
\item{}Place 
\item{}Attendees 
\item{}Referent
\end{itemize}

\begin{enumerate}
\item{}Agenda approved
\item{}Minutes of meeting from last advisor meeting approved 
\item{}Comments to the minutes from last customer meeting
\item{}Approval of the status report
\item{}Summary
\begin{enumerate}
\item{}Work done in this period
\item{}Status of the documents that are being created
\end{enumerate}
\item{}Meetings
\item{}Other activities
\end{enumerate}

\paragraph{Minutes for customer meetings} \hfill
\newline
\begin{itemize}
\item{}Date
\item{}Project name
\item{}Calling by
\item{}Time and date 
\item{}Place 
\item{}Attendees 
\item{}Referent
\end{itemize}

\begin{enumerate}
\item{}Topics of the meeting.
\item{}Decisions made.
\item{}Actions agreed upon.
\item{}Clarifications.
\item{}Time, date and place of next meeting
\end{enumerate}

\paragraph{Minutes for internal meetings} \hfill
\begin{itemize}
\item{}Date
\item{}Project name
\item{}Calling by
\item{}Time and date 
\item{}Place 
\item{}Attendees 
\item{}Referent
\end{itemize}

\begin{enumerate}
\item{}Agenda approved
\item{}Minutes of meeting from last internal, customer and advisor meeting approved
\item{}Case name
\begin{enumerate}
\item{}Sub item
\item{}Sub item
\end{enumerate}
\item{}Other issues
\item{}Next meeting
\end{enumerate}

\subsection{Standards}

\subsubsection{Organizing files}
paragraph{Google docs}\hfill
 We use Google Docs to organize our files so that our entire team can change and look at all the documents we have whenever they want. We have one folder that contains our entire project, and in that folder we have subfolders to make the best overview as possible. We try not to have too many files in each folder so that finding a specific document becomes as easy as possible.

\paragraph{Git}\hfill 
We also use git so that it is easy to share all the code as well. When using git all the team members have access to shange, read and use all the code files whenever they feel like it.

\paragraph{Saving in gitHub}
All documents and Java files must compile before they are pushed to the repository.\newline
\textbf{Code repository structure}
\begin{verbatim}
kpro-app/
  res/ -- application specific resources -- resources used in the application
    drawable/ -- images for the application
    layout/ -- layouts for the the activies
    values/ -- common values for the application
  src/ -- java source folder
    main/java/no/ntnu/kpro/app -- the difference activites
  target/ -- build folder
kpro-instrumentation/
  src/ -- java source folder
    main/java/no/ntnu/kpro
	    app/ -- gui tests folder
		  core/service/ -- service tests folder
  target/ -- build folder
kpro-lib/
  res/ -- application specific resources
  src/ -- java source folder
    main/
      java/no/ntnu/kpro/core
        model/ -- application specific models
        service/ -- service related package
          factories/ -- factory classes
          implementations/  -- implementating classes
          interfaces/ -- all relevant interfaces
        utilities -- common javabased utilities
    test/
      java/no/ntnu/kpro/core/
        service/
          implementation/
  target/ -- build folder

\end{verbatim}
\textbf{LaTeX repository structure}
\begin{verbatim}
Backlog/		-- contains the teams backlog
Meetings/		-- agendas, feedback and minutes from every meeting
  Internal/
    Agendas/
    Feedback/
    Minutes/
  Mohsen Anvaari/
    Agendas/
    Feedback/
    Minutes/
  Thales/
    Agendas/
    Feedback/
    Minutes/
Project report/		-- the tentative complete report
Templates and standards/ -- templates in use
Weekly status reports/ -- weekly timesheets

\end{verbatim}


\subsubsection{Naming of files}
All the files in our project have a name that starts with the date they were created so that they’ll be easier to find. The name also contains one or two words that explain what the documents consists.

\paragraph{Agendas}\hfill
\newline
"YYYY-MM-DD - Agenda - [Thales/Mohsen Anvaari/Internal]"

\paragraph{Minutes of meeting}\hfill
\newline
"YYYY-MM-DD - Minutes of Meeting - [Thales/Mohsen Anvaari/Internal]"

\subsubsection{Coding style}
See \ref{tab:namingconventions} on page \pageref{tab:namingconventions} for code naming conventions.
\begin{table}
\begin{tabular}{l|l|l}
\textbf{Element} & \textbf{Convention} & \textbf{Example} \\ \hline \hline
Button & btn & btnSubject \\ \hline
ImageButtonr & imb & imbSubject \\ \hline
TextView & lbl & txtSubject \\ \hline
ImageView & imv & imvSubject \\ \hline
EditText & txt & txtSubject \\ \hline
Spinner & spr & sprSubject \\ \hline
ListView & lst & lstSubject \\ \hline
CheckBox & chb & chbSubject \\ \hline
RadioButton & rbt & rbtSubject \\ \hline
ToggleButton & tbt & tbtSubject \\ \hline
Layout & lay & laySubject 
\end{tabular}
\caption{Code naming conventions - all conventions need to have + identifer (lower camel-case) in addition to its unique three letter convention }\label{tab:namingconventions}
\end{table}
