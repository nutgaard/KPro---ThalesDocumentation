\subsection{Templates and standards}
The group has decided to make templates and standards for the most relevant document types. Even though it will take some time to create these in the beginning, we believe it will benefit us over time.

\subsubsection{Templates}
We have made templates for agendas and minutes of meetings, as well as for the weekly documents. These are located in Appendix B. The template make the work for our agenda and minutes responsible a lot easier.

\subsubsection{Standards}

\paragraph{Organizing files}
paragraph{Google docs}\hfill
 We use Google Docs to organize our files so that our entire team can look at and change all the documents we have whenever they want. We have one folder that contains our entire project and in that folder we have subfolders to make the best overview possible. We try not to have too many files in each folder so that finding a specific document becomes as easy as possible.

\subparagraph{Git}\hfill 
We also use git so that it is easy to share all the code as well. When using git all the team members have access to change, read and use all the code files whenever they feel like it.

\subparagraph{Saving in gitHub}\hfill
All documents and Java files must compile before they are pushed to the repository.\newline
\textbf{Code repository structure}
\begin{verbatim}
kpro-app/
  res/ -- application specific resources -- resources used in the application
    drawable/ -- images for the application
    layout/ -- layouts for the the activies
    values/ -- common values for the application
  src/ -- java source folder
    main/java/no/ntnu/kpro/app -- the difference activites
  target/ -- build folder
kpro-instrumentation/
  src/ -- java source folder
    main/java/no/ntnu/kpro
	    app/ -- gui tests folder
		  core/service/ -- service tests folder
  target/ -- build folder
kpro-lib/
  res/ -- application specific resources
  src/ -- java source folder
    main/
      java/no/ntnu/kpro/core
        model/ -- application specific models
        service/ -- service related package
          factories/ -- factory classes
          implementations/  -- implementating classes
          interfaces/ -- all relevant interfaces
        utilities -- common javabased utilities
    test/
      java/no/ntnu/kpro/core/
        service/
          implementation/
  target/ -- build folder

\end{verbatim}
\textbf{LaTeX repository structure}
\begin{verbatim}
Backlog/		-- contains the teams backlog
Meetings/		-- agendas, feedback and minutes from every meeting
  Internal/
    Agendas/
    Feedback/
    Minutes/
  Mohsen Anvaari/
    Agendas/
    Feedback/
    Minutes/
  Thales/
    Agendas/
    Feedback/
    Minutes/
Project report/		-- the tentative complete report
Templates and standards/ -- templates in use
Weekly status reports/ -- weekly timesheets

\end{verbatim}


\paragraph{Naming of files}
All the files in our project have a name that starts with the date they were created so that they’ll be easier to find. The name also contains one or two words that explain what the documents consists.

sub\paragraph{Agendas}\hfill
\newline
"YYYY-MM-DD - Agenda - [Thales/Mohsen Anvaari/Internal]"

sub\paragraph{Minutes of meeting}\hfill
\newline
"YYYY-MM-DD - Minutes of Meeting - [Thales/Mohsen Anvaari/Internal]"

\paragraph{Coding style}
See \ref{tab:namingconventions} on page \pageref{tab:namingconventions} for code naming conventions.
\begin{table}
\begin{tabular}{l|l|l}
\textbf{Element} & \textbf{Convention} & \textbf{Example} \\ \hline \hline
Button & btn & btnSubject \\ \hline
ImageButtonr & imb & imbSubject \\ \hline
TextView & lbl & txtSubject \\ \hline
ImageView & imv & imvSubject \\ \hline
EditText & txt & txtSubject \\ \hline
Spinner & spr & sprSubject \\ \hline
ListView & lst & lstSubject \\ \hline
CheckBox & chb & chbSubject \\ \hline
RadioButton & rbt & rbtSubject \\ \hline
ToggleButton & tbt & tbtSubject \\ \hline
Layout & lay & laySubject 
\end{tabular}
\caption{Code naming conventions - all conventions need to have + identifer (lower camel-case) in addition to its unique three letter convention }\label{tab:namingconventions}
\end{table}
