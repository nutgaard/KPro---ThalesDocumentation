\chapter{Project directive}

This chapter is about how we planned our project. The purpose of this chapter is to explain how our team is organized, who we are, why we do this project and how do it.

\section{Overall project plan}

\subsection{Project mandate}
The purpose of this project was to create a handheld version of the already existing XOmail system. Today’s system only works at computers, and is limited to use in offices and alike. The task we were assigned was to make a simplified version of XOmail optimized to use at handheld devices such as mobile phones.
\newline
\newline
The tile of the project was decided to be XOXOmail since it was to become a new and improved version of the already existing XOmail. XOXO is a common ending of letters and mails and is used to express affection or friendship, made worldwide famous by the tv series “Gossip Girl”. The expression is an abbreviation for the term hugs and kisses.

\subsection{The client}
The customer for this project is Thales Norway AS. Thales is a leading international electronics and systems group, focusing on defense, aerospace and security markets worldwide. The cutting edge technology in use at Thales offers capabilities unmatched in Europe for the development and deployment of mission-critical information systems proven in the field. The group’s civil and military business areas develop in parallel and share a common base of technologies to serve a single objective: the security of people, property and nations.
\newline
\newline
After Thales’ 50 years of industrial activity in Norway it is now one of Norway’s largest industrial centers of expertise for mission-critical IT and telecommunications solutions and one of the principal supplies of military communication systems to the Norwegian Armed Forces. [1]
\newline
\newline
See table \ref{tab:customer} at page \pageref{tab:customer}
\begin{table}
\begin{tabular}{l|l|l|l}
\textbf{Name} & \textbf{Office} & \textbf{Phone nr} & \textbf{E-mail} \\ \hline \hline
Sølve Conradi Olsen & Thales Norway & 907 80 179 & solve.olsen@thalesgroup.com \\ \hline
Christian Tellefsen & Thales Norway & 959 98 765 & christian.tellefsen@thalesgroup.com \\ \hline
Stig Bjørlykke & Thales Norway & 982 29 806 & stig.bjorlykke@thalesgroup.com
\end{tabular}
\caption{The customer representatives} \label{tab:customer}
\end{table}

\subsection{Involved Parties}
In this project there are only three involved parties: a) the customer b) the project team and c) the advisor. The customer, Thales, described in the section above was represented by Christian Tellefsen and Stig Bjørlykke. The project team consists of 6 students from the Department of Computer and Information Science (IDI) at the Norwegian University of Science and Technology (NTNU).The advisor was Mohsen Anvaari who was assigned to our team to guide and help us during the project period.
\newline
\newline
See table \ref{tab:projectgroup} at page \pageref{tab:projectgroup}
\begin{table}
\begin{tabularx}{\linewidth}{>{\setlength\hsize{.52\hsize}}X|>{\setlength\hsize{0.5\hsize}}X|>{\setlength\hsize{.3\hsize}}X|>{\setlength\hsize{.5\hsize}}X}
\textbf{Name} & \textbf{Address} & \textbf{Phone nr} & \textbf{E-mail} \\ \hline \hline
Ida Thoresen & Klæbuveien 143, 7031 & 936 68 688 & idakatt@stud.ntnu.no\\ \hline
Kristin Tønnesen & Lars Onsagersvei 12, 7030 & 986 28 958 & kristonn@stud.ntnu.no \\ \hline
Lars Høysæter & Innherredsveien 2a, 7014 & 900 31 814 & larssmor@stud.ntnu.no\\ \hline
Nicklas Utgaard & Odd Brochmanns veg 47, 7030 & 976 87 790 & nicklau@stud.ntnu.no\\ \hline
Magnus Ulstein & & 472 31 418 & magnuul@stud.ntnu.no\\ \hline
Aleksander Sjåfjell & Odd Brochmanns veg 60, 7030 & 456 01 212 & aleksasj@stud.ntnu.no
\end{tabularx}
\caption{The project group} \label{tab:projectgroup}
\end{table}

See table \ref{tab:advisor} at page \pageref{tab:advisor}
\begin{table}
\begin{tabular}{l|l|l|l}
\textbf{Name} & \textbf{Work} & \textbf{Phone nr} & \textbf{E-mail} \\ \hline \hline
Mohsen Anvaari & PhD-stud. NTNU & 405 70 403 & mohsena@idi.ntnu.no
\end{tabular}
\caption{Our advisor} \label{tab:advisor}
\end{table}

\subsection{Background for the project}
This project was given to us as an assignment in the course TDT4290 Customer Driven Project. Since the society is constantly changing, the technology must follow along with the enormous changes we see today. Because of these changes Thales found that they needed a handheld android version of their existing XOmail system. This new handheld system would help the users of XOmail so that they could also use it in the field and not only at the office. When able to use this mailing system not only in contact with a computer, the regular working day of Thales customers would be a lot easier, especially since their primary customer is the Norwegian military.

\subsection{Project objective}
The objective from the customer was twofold: The first part was to develop a functioning prototype to demonstrate a possible way of doing formal and secure messaging on a mobile platform. The second part was to explore possibilities for further development, e.g. what solutions already exists and what is required to implement the level of security that must be present in military systems. The application should include functionality much like existing mail applications, but the security aspects are quite different and must be investigated and documented, as there probably will not be time enough to implement all the features.
The objective from NTNU’s point of view was to give the students practical experience with being involved in all the phases of a large IT project and also experience in teamwork. 
The goal of the team was to become more experienced with working with a more real development project and get a glance into how the real world works. It was also important to develop an application that the customer could be satisfied with. 


\subsection{Duration}
The course staff has suggested a estimated 25 hour week for each student. This will give us a total of 1950 working hours since we are 6 students in our team and the project is distributed over 13 weeks.

\begin{itemize}
\item{}Project start: 21.08-2012
\item{}Project finish: 22.11-2012
\end{itemize}

