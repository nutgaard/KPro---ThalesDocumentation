\chapter{Project directive}

This chapter is about how we planned our project. The purpose of this chapter is to explain how our team was organized, who we are, why we conducted this project and how we did it.

\section{Overall project plan}

\subsection{Project mandate}
The purpose of this project was to create a handheld version of the already existing XOmail system, as this system only works at computers. The task we were assigned was to make a simplified version of XOmail optimized to be used at handheld devices such as mobile phones.
\newline
\newline 
The title of the project and the application was decided to be XOXOmail as we thought this was a fun name. It is a combination of the name of the existing system and a common way of ending letters and mails, namely "XOXO". The term is an abbreviation for the term hugs and kisses and is used to express affection or friendship and was made worldwide famous by the TV series "Gossip Girl".

\subsection{The client}
The customer for this project was Thales Norway AS, a subsidiary of the leading international electronics and systems group Thales. Thales' focus areas are aerospace, defence and security, and the company has positioned themself as one of the world leaders in mission-critical information systems. Thales Norway has had over 50 years of industrial activity in Norway and is one of the principal suppliers of a military communication system to the Norwegian Armed Forces.
\cite{bib:thales}.
\newline
\newline
For the contact info of the contact persons at Thales see table \ref{tab:customer} below.
\begin{table}[h!]
\begin{center}
\begin{tabular}{l|l|l|l}
\hline
\textbf{Name} & \textbf{Office} & \textbf{Phone nr} & \textbf{Email} \\ \hline \hline
Sølve Conradi Olsen & Thales Norway & 907 80 179 & solve.olsen@thalesgroup.com \\ 
Christian Tellefsen & Thales Norway & 959 98 765 & christian.tellefsen@thalesgroup.com \\ 
Stig Bjørlykke & Thales Norway & 982 29 806 & stig.bjorlykke@thalesgroup.com \\ \hline
\end{tabular}
\end{center}
\caption{The customer representatives} \label{tab:customer}
\end{table}

\subsection{Involved parties}
In this project there were three involved parties: a) the customer b) the project team and c) the advisor. The customer, Thales, described in the section above was represented by Christian Tellefsen and Stig Bjørlykke. The project team consisted of six students from the Department of Computer and Information Science (IDI) at the Norwegian University of Science and Technology (NTNU).The advisor was Mohsen Anvaari who was assigned to our team to guide and help us during the project period.
\newline
\newline
For contact info of the group, see table \ref{tab:projectgroup} below.
\begin{table}[h!]
\begin{center}
\begin{tabularx}{\linewidth}{>{\setlength\hsize{.52\hsize}}X|>{\setlength\hsize{0.5\hsize}}X|>{\setlength\hsize{.3\hsize}}X|>{\setlength\hsize{.5\hsize}}X}
\hline
\textbf{Name} & \textbf{Address} & \textbf{Phone nr} & \textbf{Email} \\ \hline \hline
Ida Thoresen & Klæbuveien 143, 7031 & 936 68 688 & idakatt@stud.ntnu.no\\ 
Kristin Tønnesen & Lars Onsagersvei 12, 7030 & 986 28 958 & kristonn@stud.ntnu.no \\ 
Lars Høysæter & Innherredsveien 2a, 7014 & 900 31 814 & larssmor@stud.ntnu.no\\ 
Nicklas Utgaard & Odd Brochmanns veg 47, 7051 & 976 87 790 & nicklau@stud.ntnu.no\\ 
Magnus Ulstein & & 472 31 418 & magnuul@stud.ntnu.no\\ 
Aleksander Sjåfjell & Odd Brochmanns veg 60, 7051 & 456 01 212 & aleksasj@stud.ntnu.no\\ \hline
\end{tabularx}
\end{center}
\caption{The project group} \label{tab:projectgroup}
\end{table}
\newline
\newline
For contact info of the advisor, see table \ref{tab:advisor} below.
\begin{table}[h!]
\begin{center}
\begin{tabular}{l|l|l|l} \hline
\textbf{Name} & \textbf{Work} & \textbf{Phone nr} & \textbf{Email} \\ \hline \hline
Mohsen Anvaari & PhD-stud. NTNU & 405 70 403 & mohsena@idi.ntnu.no \\ \hline
\end{tabular}
\end{center}
\caption{Our advisor} \label{tab:advisor}
\end{table}

\subsection{Background for the project}
This project was given as an assignment in the course TDT4290 Customer Driven Project. Since the society is constantly changing, the technology must follow along. Because of these changes Thales found that they wanted a prototype of an Android version of their existing XOmail system. This new handheld system could help the users of XOmail so that they could also use it on mobile devices and not only at desktop computers. When able to use this mailing system not only in contact with a computer, the regular working day of Thales customers could be a lot easier, especially for their primary customer, the Norwegian Armed Forces.

\newpage

\subsection{Project objective}
The objective from the customer was twofold: The first part was to develop a functioning prototype to demonstrate a possible way of implementing formal and secure messaging on a mobile platform. The second part was to explore possibilities for further development, e.g. what solutions already exist and what is required to implement the level of security that must be present in military systems. The application should include functionality much like existing mail applications, but the security aspects are quite different and must be investigated and documented, as there probably will not be time enough to implement all the desired features.
\newline
\newline
The objective from NTNU’s point of view was to give the students practical experience with being involved in all the phases of a large \gls{it} project and also experience in teamwork. 
\newline
\newline
The goal of the team was to become more experienced with working in a real development project and get a glance into how development with real customers works. It was also important for us to develop an application that the customer could be satisfied with. 


\subsection{Duration}
The course staff suggested an estimated 25 hour week for each student. This gave us a total of 1950 working hours since we were six students in our team and the project was distributed over 13 weeks.

\begin{itemize}
\item{}Project start: 21.08.2012
\item{}Project finish: 22.11.2012
\end{itemize}

