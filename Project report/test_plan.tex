\chapter{Test plan}

The main part of project testing will be done as we go along. Using unit tests, each part of the code will be thoroughly tested before it is put together. Larger scale integration and instrumentation tests are done largely the same way, compounded by informal testing by group members. The sending and receiving of emails will be tested using Greenmail, while Mockito is used to handle not yet implemented classes as well as emulator free testing for Android specific classes. All of these tools are described in the section below. Towards the end of the project period, including large parts of the fourth sprint, the focus will shift from producing additional code to finding and repairing bugs. While our plan is vaguely based upon, and in large part inspired by, \gls{ieee829} \cite{bib:ieee} we are too informal about it to be able to use such grand claims. In short, our testing plan is to test each public method both for valid and invalid input, and then run larger scale integration tests between modules.
By using dependency injection and Mockito, we will also be able to test top level modules without testing the building blocks of that module. 
\newline
\newline
The overall goal for the test plan is that every part of our code related to the backend service is tested thoroughly, both for valid and invalid input. Though the same would be good for the frontend,
it is harder to do this automaticly, and therefore it will only be tested whether or the functionality works or not. The optimal goal would be 100\% code coverage, but this is rarely feasible, hence
we will settle for anything between 60-80\%. 

\section{Methods for testing}
In order to ensure the greatest possibility of correctness for our code, each member of the team will be responsible for writing test for their code before it is submitted to our common repository.
Changes in the user interface should also be tested, though if there is no automatic way of doing it, one should tests all functionality related to the changes one have made in the code. 

\section{Test tools}
\subsubsection{JUnit}
JUnit is the industry standard for test driven development. It works by creating a separate test method for each method in the class that is to be tested. In this way one can test large portions of code independent of the other code sections. For us, it provides many benefits, especially that most of us have dealt with it before and are familiar with the concept.

\subsection{Cobertura}
Cobertura is a tool for calculating the code covered by our tests, and generate a report based on this. While all other testing tools are automatically run at each build cycle, this is not the case with
Cobertura, which need to be invoked by a seperate command. 

\subsection{Mockito}
Mockito is a simple and elegant \gls{api} for mocking classes. With a few lines of code an unimplemented class can be instructed to provide a valid output, allowing other parts of the code to be tested without concerns.

\subsection{Greenmail}
Greenmail is an open source mail server package for testing in Java. It provides all the server side software needed to test simple sending and receiving of mail without having an actual server. 
It does, however, not supprt the \gls{pami} IDLE, which in turn forced us to test this feature manually.

\section{Design Verification}
In order to accommodate the agile development plan used by the team, our early design verification is necessarily somewhat vague. In the early stages of the project we had limited knowledge of how the final product would be structured and which features it would involve.
\newline
\newline
Early in the first sprint we decided upon a set of basic interfaces, which neatly split up the project into manageable parts. We began early by writing up tests for future classes that implemented these interfaces, allowing low level testing of individual methods. At the time of writing these tests we had no idea how each method was going to be implemented. This enabled us to write generic black box tests.
\newline
\newline
Later in the project this method level testing will augmented by white box and stress testing; essentially us deliberately trying to crash the system with strange inputs or action combinations. 

\section{Why we selected these test tools}
The choice of tools was mainly based on previous experience inside the group, the framework provided by and the demands of the platform. 
\newline
\newline
\gls{junit}t was chosen because of our previous experience and the fact that the Android instrumentation framework is based on \gls{junit}. Bundled with Mockito it also gives us the possibility to test otherwise Android dependent modules in a Java environment with the Android specifics mocked up. This can greatly reduce the build cycle time, since the tests no longer need to be pushed onto a device in order to run. Mockito also provides another benefit, in that you can test a component without testing its dependencies. This allows for highly specific testing of just one class, component or module.
\newline
\newline
Android instrumentation was chosen as a way to enable testing of \gls{gui}; provided with the \gls{kds} and recommended by Google, it was the only tool providing us with the functionality we needed.
\newline
\newline
Greenmail was chosen based on the need to test the mail module in our project, ease of use and fast response time. If the code passes the Greenmail tests, the code is then tested against gmail’s mail server to ensure that functions with external servers as well. 

\section{Templates for testing}
\begin{table}
\begin{tabular}{l|p{10cm}}
Test case ID & id \\ \hline
Name & Shortname for the test\\ \hline
Requirement & Requirement reference\\ \hline
Description & Description of test\\ \hline
Preconditions & The conditions of the application before the tests are started\\ \hline
Flow of events & The list of steps necessary to complete the test \\ \hline
Expected results & The end result if everything works as expected\\ \hline \hline
\end{tabular}
\caption{Test case template} \label{tab:casetemp}
\end{table}
\begin{table}
	\begin{tabular}{l|p{10cm}}
		Test case ID & id \\ \hline
		Actual results & The end result\\ \hline
		Comments & What went wrong, what could been the cause\\ \hline
		Status & 
	\end{tabular}
	\caption{Test report template} \label{tab:casereporttemp}
\end{table}