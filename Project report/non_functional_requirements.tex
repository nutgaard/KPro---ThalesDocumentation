\section{Non-functional requirements}
This section will explain the requirements for the application that do not add specific functionality to the product.

\subsection{Quality requirements}
The requirements for quality attributes will be listed and explained below.

\paragraph{Usability}\hfill
\newline
Usability explains how we made our application effective to learn and use.
\newline
\newline
\textit{\textbf{U1 Ease of use}}
\newline
The user should be able to use the application with very few mistakes, such as pressing the wrong button because button labels are ambiguous or too small. The scenario is described in table \ref{tab:easeofuse}.
\begin{table}[h!]
\begin{center}
\begin{tabularx}{\linewidth}{>{\setlength\hsize{.3\hsize}}X|>{\setlength\hsize{0.7\hsize}}X} \hline
\textbf{Portion of scenario} & \textbf{Values} \\ \hline \hline
Source & End user \\ \hline
Stimulus & Use system without problems \\ \hline
Artifact & System \\ \hline
Environment & At run time \\ \hline
Response & Provide clean and understandable interface \\ \hline
Response measure & Maximum 5\% of user operations should be mistakes (another action than the user wanted) \\ \hline
\end{tabularx}
\end{center}
\caption{U1 Ease of use} \label{tab:easeofuse}
\end{table}

\textit{\textbf{U2 Efficent use}}
\newline
The user should be able to use the main features of the application with as few screen interactions as possible. Streamlined design is therefore of great importance. The scenario is described in table \ref{tab:efficentuse}.
\begin{table}[hbt]
\begin{center}
\begin{tabularx}{\linewidth}{>{\setlength\hsize{.6\hsize}}X|>{\setlength\hsize{1.4\hsize}}X} \hline
\textbf{Portion of scenario} & \textbf{Values} \\ \hline \hline
Source & End user \\ \hline
Stimulus & Use system efficiently \\ \hline
Artifact & System \\ \hline
Environment & At run time \\ \hline
Response & Organize application so that important functions are easy to reach from the main menu \\ \hline
Response measure & A user with only a quick (5 min) training course should never spend more than one minute finding what he looks for \\ \hline
\end{tabularx}
\end{center}
\caption{U2 Efficent use} \label{tab:efficentuse}
\end{table}

\newpage

\paragraph{Performance}

\textit{\textbf{P1 Latency}}
\newline
The user should be able to read the message quickly after receival of a new message. The scenario is described in table \ref{tab:performance}.
\begin{table}[h!]
\begin{center}
\begin{tabularx}{\linewidth}{>{\setlength\hsize{.6\hsize}}X|>{\setlength\hsize{1.4\hsize}}X}\hline
\textbf{Portion of scenario} & \textbf{Values} \\ \hline \hline
Source & End user \\ \hline
Stimulus & Wants to open message \\ \hline
Artifact & System \\ \hline
Environment & Under normal operations \\ \hline
Response & The operation is performed \\ \hline
Response measure & With a latency of maximum 3 seconds, the user should be able to read the message after it is received. This is the latency when we subtract the download time of message, which is dependent of the network connection \\ \hline
\end{tabularx}
\end{center}
\caption{P1 Latency} \label{tab:performance}
\end{table}

\paragraph{Security}
\textit{\textbf{S1 Data access}}
\newline
Data should not be exposed to unauthorized users or other application. The scenario is described in table \ref{tab:s1}.
\begin{table}[h!]
\begin{center}
\begin{tabularx}{\linewidth}{>{\setlength\hsize{.6\hsize}}X|>{\setlength\hsize{1.4\hsize}}X} \hline
\textbf{Portion of scenario} & \textbf{Values} \\ \hline \hline
Source & Unauthorized user or app \\ \hline
Stimulus & Wants to access data saved by XOXOmail \\ \hline
Artifact & System \\ \hline
Environment & Under normal operations \\ \hline
Response & User is blocked from accessing data by the Android OS \\ \hline
Response measure & No data should be exposed to the unauthorized user or application \\ \hline
\end{tabularx}
\end{center}
\caption{S1 Accessing locally stored data outside of application} \label{tab:s1}
\end{table}

\newpage

\textit{\textbf{S2 Unprivileged use}}
\newline
Users that try to use the application without the correct privileges should be denied. The scenario is described in table \ref{tab:s2}.
\begin{table}[h!]
\begin{center}
\begin{tabularx}{\linewidth}{>{\setlength\hsize{.6\hsize}}X|>{\setlength\hsize{1.4\hsize}}X}\hline
\textbf{Portion of scenario} & \textbf{Values} \\ \hline \hline
Source & Unauthorized user \\ \hline
Stimulus & Wants to use application features \\ \hline
Artifact & System \\ \hline
Environment & Under normal operations \\ \hline
Response & User is blocked from using functions \\ \hline
Response measure & No features should be exposed to the user, the user should be stopped by the login screen\\ \hline
\end{tabularx}
\end{center}
\caption{S2 Trying to use application with wrong privileges} \label{tab:s2}
\end{table}

\textit{\textbf{S3 External data traffic}}
\newline
External data traffic should not be exposed to unauthorized users or application or someone else sniffing network packages. The scenario is described in table \ref{tab:s3}.
\begin{table}[h!]
\begin{center}
\begin{tabularx}{\linewidth}{>{\setlength\hsize{.6\hsize}}X|>{\setlength\hsize{1.4\hsize}}X}\hline
\textbf{Portion of scenario} & \textbf{Values} \\ \hline \hline
Source & Unauthorized user or application, or someone sniffing network packages \\ \hline
Stimulus & Wants to access data stream \\ \hline
Artifact & System \\ \hline
Environment & Under normal operations \\ \hline
Response & Is unable to get useful information because of SSL encryption \\ \hline
Response measure & No useful data exposed to the user\\ \hline
\end{tabularx}
\end{center}
\caption{S3 Trying to access the application's external data traffic} \label{tab:s3}
\end{table}

\newpage

\subsection{Component and technology related constraints}
Vanilla Android is the platform of choice, compatible down to Android API 8.

\paragraph{Programming language} \hfill
\newline
Developing for Android will in this project means that code will be written in Java for the Dalvik Virtual Machine. This means that there is no support for multiple inheritance, and this limits design choices to some degree. It prevents us from playing with some more complicated architectures and inheritance schemes.

\paragraph{Mobile devices} \hfill
\newline
Android is a mobile OS, so the application will run on cell phones and potentially tablets. 

\paragraph{Input methods} \hfill
\newline
Android phones have small screens and usually no keyboard. This means that the user interface must be designed to work with touch interactions as the only input method.

\paragraph{Memory}\hfill
\newline
Phones have limited memory, typically 256-512 \gls{mb}. The application will probably never be close to breaking this limit, as the most memory consuming components are string based messages and potentially an attachment or two; very little of which has to remain in memory at the same time.

\paragraph{Screen resolution/aspect ratio} \hfill
\newline
As the application will support different Android devices, it also has to support different resolutions as well as aspect ratios. The \gls{gui} must be programmed to scale to different formats without degrading quality or proportions.

\paragraph{Bandwidth} \hfill
\newline
Since the application is network based, it is important to pay attention to bandwidth usage. We want to be able to wait for incoming messages and receive them as soon as the server does without having to waste bandwidth with constant polling, especially as the application is expected to see some use outside of 3G net areas.

\paragraph{Commercial off-the-shelf (\gls{cots})} \hfill
\newline
This project has security requirements that require special consideration with regards to \gls{cots}. We have to be able to trust the underlying software if it ever touches unencrypted data or has access to a memory partition (either legitimately or not) that contains unencrypted data. This means some care has to be taken in regards to what \gls{cots} solutions we can use and where we can use them. 

\newpage

\section{Stakeholders and their concerns}
For an overview of the stakeholders and what concerns these have in the project, see table \ref{tab:stakeholders}.

\begin{table}[h!]
\begin{tabular}{p{3.5cm}|p{11.5cm}} \hline
\textbf{Stakeholder} & \textbf{Concern} \\ \hline \hline
User & 
\begin{itemize}
\item{} Easy to understand the flow of the application
\item{} Installation should be easy
\item{} User friendly user interface
\end{itemize}\\ \hline
Developers & 
\begin{itemize}
\item{}Easy to understand the goal of the application
\item{}Easy to extend and change the application
\item{}Want to use technology they are familiar with
\item{}Easy to understand the requirements.
\end{itemize}\\ \hline
Course advisor & 
\begin{itemize}
\item{}Effective and healthy communication with the group
\item{}Easy to read documentation
\item{}Gets all deliveries on schedule
\end{itemize}\\ \hline
Customer & 
\begin{itemize}
\item{}Effective and healthy communication with the group
\item{}A working prototype that gives a possible solution to the requirements
\item{}Wants to see what is possible on the Android platform
\item{}Understandable requirements and architectural description documents.
\end{itemize}\\ \hline
Graphic designers & 
\begin{itemize}
\item{}Understanding of how the \gls{gui} should be
\item{}Understand the limitation of graphics due to screen size
\end{itemize} \\ \hline
\end{tabular}
\caption{Stakeholders and their concerns} \label{tab:stakeholders}
\end{table}