\subsection{Securing user credentials}
In the application we are developing we need some way of authenticating users and granting them access to the features of the application. We want the users to log in using a username and password and also allowing them to save the password so that the user does not have to retype the password every time they open the application. This feature creates certain security issues. 

\subsubsection{Local storage with username and password}
One solution is to store the username and password locally on the phone. In android applications this is usually done in SharedPreferences. SharedPreferences is sandboxed and thus preventing other applications from accessing the values stored there. If you also encrypt the credentials instead of just storing them as cleartext it would add even more security. 
\newline
\newline
The problem with this solution is if an attacker should in some way gain physical access to your phone. If you are already logged in, the perpetrator could just open up the app and view the date he wants. Even if you weren’t logged in the attacker could root the phone and gain access to the values saved in SharedPreferences. Even if the values are encrypted it would not help, because an attacker that has access to the values in SharedPreferences is also likely to have access to the applications binary, and thus the keys to decrypt the password.

\subsubsection{Creating a custom account type}
Another solution is to create an online user account and add this to the centralized AccountManager registry. The user can then type in his username and password once, and gain access to the various resources online. The credentials are here authenticated on the server and we can therefore store the credentials as a cryptographically secure token on the device. This would make sure that an attacker would not get access to your actual username and password. 
\newline
\newline
This solution also suffers from the fact that if the user is already logged in, the attacker could view all the information. One could solve this by implementing a timeout on the application (If the user does not perform an action after x minutes, the app is terminated and the user gets disconnected).

\subsubsection{Symmetric encryption}
Symmetric encryption is also a possible solution. If you are going to use this type of encryption you can’t store the keys because of the lacking general purpose, system-level secure storage. A way to solve this is to derive the keys from a user-entered password. To make sure the keys derived are random and hard to brute-force we need to use a standard PBE key derivation method. For android applications this is called PBKDF2WithHmacSHA1.
\newline
\newline
This could be a suitable solution, at least for the parts of the application that does not need to happen instantaneous. 