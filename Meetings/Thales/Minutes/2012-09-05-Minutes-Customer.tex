\documentclass[a4paper,12pt]{article}
\usepackage{array}
\begin{document}
\title{Minutes of Meeting for Customer Meeting \#3}
\maketitle
\begin{tabular}{>{\bfseries}l l}	
Project name&Formal and secure messaging on a mobile platform\\
Calling by&Group 15\\
Time and date&2012-09-05 12:15-13:00\\
Place&Thales' offices at Lerkendal\\
Attendees&Christian and Stig from Thales\\
& Aleksander, Ida, Kristin, Lars, Magnus and Nicklas from Group 15 \\
Referent&Kristin from Group 15\\
\end{tabular}
\section{Agenda approved}
\section{Minutes of meeting from last customer meeting approved}
Thales had some questions about the minutes, but they were resolved through mail exchange with the referent.
\section{Sprint planning}
\subsection{Comments to JIRA}
Thales had sent some comments to group 15 about the content in JIRA prior to the meeting. Their main concern was that they could not see a general, overall plan of how group 15 wants to reach the goals of the project. Thales also thought that tasks concerning pre-studies/analysis were missing in JIRA. Group 15 has to make decisions about choice of technology, use of third party code and design choices, at least at the end of sprint 1.
\subsection{Backlog document}
The discussion continued to the backlog document from Group 15. Thales wanted Group 15 to decide on the goal with each sprint, as both the developers and customer then has something concrete to relate to.\\
Thales wanted BL-5 (Message template) to be given less priority.
\subsection{Estimate of tasks}
Thales thought the estimates for the tasks might be a little optimistic. Group 15 will change to using story points instead of time estimates, as it is probably easier to estimate tasks in complexity instead of in hours, and it is also closer to the Scrum methodology. Group 15 will therefore make an updated version of the backlog and send this to Thales by October 13th.\\
Thales’ first impression of the backlog document is that it seems OK, and will send further comments by e-mail.
\subsection{Presentation of architecture}
Nicklas presented the current design, including interfaces and patterns. Thales will provide further comments on mail.
\section{Other issues}
\subsection{XOmail demonstration}
Thales showed a demo of their XOmail client. Group 15 found it interesting to see the complete version of the XOmail. However, the demo was very quick and the program complex and with a lot of functionality, and Group 15 does not see how we can use this experience at the moment.
\subsection{Password storing}
The discussion continued with details about storing passwords. Group 15 needs to find a smart way of storing passwords, e.g. find ways to get the OS to support it.
\section{Decisions}
\begin{enumerate}
\item
Further questions and comments will be exchanged by e-mail until the next meeting.
\end{enumerate}
\section{Actions}
\begin{enumerate}
\item
Group 15 will send digital versions of the backlog and architecture document to Thales.
\item
Thales will send further comments about the backlog and architecture document to Group 15
\item
Group 15 will send an updated version of the product backlog with priorities by October 13th.
\end{enumerate}
\section{Next meeting}
Wednesday 2012-09-19 12:15 at Lerkendal
\end{document}